\renewcommand{\chaptermark}[1]{\markboth{Appendix \thechapter\relax:\thinspace\relax#1}{}}
\chapter{Binary representation of FFS images and Classes}
\label{app:binary_rep}
This appendix visualizes the binary structures produced when serializing the \texttt{InodeTable}, the \texttt{InodeEntry}, and the \texttt{Directory} objects, and the binary structure of the encoded FFS images.

\section{Serialized C++ objects}
\begin{figure}[!ht]
	\centering
    \textbf{\texttt{InodeTable}}\par\medskip

	% bitwidth = 1/16 * textwidth
	\begin{bytefield}[bitwidth=0.0625\textwidth,endianness=big]{16}
		\bitbox{16}{\#~Inode Entries}\\
		\bitbox{4}{Inode~1} & \bitbox{12}{Inode Entry~1} \\
		\bitbox{4}{Inode~2} & \bitbox{12}{Inode Entry~2} \\
		\wordbox[]{1}{$\vdots$} \\[1ex]
		\bitbox{4}{Inode~N} & \bitbox{12}{Inode Entry~N} \\
	\end{bytefield}
	\caption[Binary representation of the serialization of a \texttt{InodeTable} object]{\texttt{\#~Inode Entries} is a 4-byte unsigned integer representing the amount of inode entries the inode table contains. Following is \texttt{\#~Inode Entries} entries of a 4-byte unsigned integer value representing the inode of the inode entry, and the serialization of the \texttt{InodeEntry} object}
\end{figure}

\section{FFS Images}
\begin{figure}[!ht]
	\centering
    \textbf{FFS Header}\par\medskip

	% bitwidth = 1/16 * textwidth
	\begin{bytefield}[bitwidth=0.0625\textwidth,endianness=big]{16}
		\bitheader{0,1,2,3,4,12,15} \\
		\bitbox{4}{Data length} & \bitbox{8}{Timestamp} & \bitbox{1}{V} &
		\bitbox{1}{'F'} & \bitbox{1}{'F'} & \bitbox{1}{'S'}
	\end{bytefield}
	\caption[Binary representation of the FFS image header]{\texttt{'F'} and \texttt{'S'} are the literal letters F and S in ASCII code. \texttt{V} is an integer representing the version of the FFS image produced. \texttt{Timestamp} is an unsigned integer representing the number of milliseconds since Unix epoch when the image was encoded. \texttt{Data length} is an unsigned integer representing the number of bytes stored after the header. Following the heder is \texttt{Data length} bytes, containing the actual data stored in the image}
\end{figure}