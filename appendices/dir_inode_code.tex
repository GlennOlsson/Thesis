\renewcommand{\chaptermark}[1]{\markboth{Appendix \thechapter\relax:\thinspace\relax#1}{}}
\chapter{Directory, InodeTable and InodeEntry class and attributes representation}
\label{app:inode_dir_code}
\begin{lstlisting}[language=c++, caption={The attributes classes representing directories and the inode table in FFS}, label=lst:dir_itable_classes,breaklines=true]
/**
* @brief Describes a directory in FFS. Keeps track of the filename and inode of each file
*/
class Directory {
public:
	/**
	* @brief Map of (filename, inode id) describing the content of the directory
	*/
	std::map<std::string, inode_id> entries;
	
	/**
	* @brief Returns the size of the directory object in terms of bytes
	* 
	* @return uint32_t the amount of bytes required by object
	*/
	uint32_t size();
};

/**
* @brief Describes and entry in the inode table, representing a file or directory
*/
class InodeEntry {
public:
	/**
	* @brief The size of the file (not used for directories) 
	*/
	uint32_t length;

	/**
	* @brief True if the entry describes a directory, false if it describes a file
	*/
	uint8_t is_dir;

	/**
	* @brief A list representing the posts of the file or directory. 
	*/
	std::vector<post_id> post_blocks;
	
	/**
	 * @brief Returns the size of the object in terms of bytes
	 * 
	 * @return uint32_t the amount of bytes occupied by object
	 */
	uint32_t size();
};

/**
* @brief Describes the inode table of the filesystem. The table consists of multiple inode entries
*/
class InodeTable {
public:
	/**
	* @brief Map of (inode id, Inode entry) describing the content of the inode table
	*/
	std::map<inode_id, InodeEntry> entries;

	/**
	 * @brief Returns the size of the object in terms of bytes
	 * 
	 * @return uint32_t the amount of bytes occupied by object
	 */
	uint32_t size();
};

\end{lstlisting}