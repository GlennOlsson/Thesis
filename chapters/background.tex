\chapter{Background}

\label{ch:background}
% \todo[inline, backgroundcolor=kth-lightblue]{Bakgrund}

% \todo[inline]{When you do your literature study, you should have a nearly complete Chapters 1 and 2.\\
% You may also find it convenient to introduce the future work section into your report early – so that you can put things that you think about but decide not to do now into this section.\\
% Note that later you can move things between this future work section and what you have done as you may change your mind about what to do now versus what to put off to future work.
% }
% \todo[inline]{What does a reader (another x student -- where x is your study line) need to know to understand your report?
% What have others already done? (This is the “related work”.) Explain what and
% how prior work / prior research will be applied on or used in the degree
% project /work (described in this thesis). Explain why and what is not used in
% the degree project and give valid reasons for rejecting the work/research.}

% This chapter provides basic background information about xxx. Additionally, this chapter describes xxx. The chapter also describes related work xxxx.



% \todo[inline, backgroundcolor=kth-lightblue]{Vilken viktig litteratur och
  % (forsknings-)artiklar har du studerat inom området (litteraturstudie)? }

This chapter presents concepts and theory that is relevant for understanding, implementing, and evaluating \gls{FFS}. Data storage principles of filesystems and the idea of \mbox{inode-based} filesystems is presented in \Cref{sec:fs_data_store}. \Cref{sec:fuse} introduces the \gls{FUSE} library that will be used to implement \gls{FFS}. \Cref{sec:ows} presents background information about \glspl{OWS}, such as Twitter. \Cref{sec:back_crypto} presents theory about the cryptographic methodologies used by \gls{FFS} to encrypt and decrypt the data. \Cref{sec:introTreats} presents some threats, including potential adversaries of \gls{FFS}. 

\section{Filesystems and data storage}
This section presents how certain filesystems used today are structured. We present the idea of inode-based filesystems and distributed filesystems. Following, we describe how data is stored in a storage system and how this information can be used in FFS.

\subsection{Unix filesystems}
A Unix filesystem uses a data structure called an \textit{inode}. The inodes are found in an inode table and each inode keeps track of the size, blocks used for the file's data, and metadata for the files in the filesystem. A directory simply contains the file names and each file or directory's inode id. The system can with an inode id find information about the file or directory using the inode table. Each inode can contain any metadata that might be relevant for the system, such as creation time and last update time. 

Figure~\ref{fig:inode_diag} shows how example inode filesystem and how it can be visualized. The blocks of an inode entry are where in the storage device the data is stored, each block is often defined as a certain amount of bytes. Listing~\ref{lst:inode_fs} describes a simple implementation of an inode, an inode table, and directory entries. 

% TODO: Add info about how FFS will only implement some metadata? And/or add that to delimitations too

\begin{figure}[!ht]
	\begin{center}
	  \includegraphics[width=0.5\textwidth]{figures/inode_diagram.png}
	\end{center}
	\caption{Basic structure of inode-based filesystem}
	\label{fig:inode_diag}
\end{figure}

\begin{minipage}{\linewidth}
\begin{lstlisting}[language=c, caption={Pseudocode of a minimalistic inode filesystem structure}, label=lst:inode_fs]
struct inode_entry {
	int 	length
	int[]	blocks
	// Metadata attributes are defined here
}

struct directory_entry {
	char*   filename
	int     inode
}

// Maps inode_id to a inode_entry
map<int, inode_entry> inode_table

\end{lstlisting}
\end{minipage}

Different filesystems provide different features and limitations. Extended Filesystem (ext) exists in four different versions: ext, ext2, ext3, and ext4. This filesystem is often used on Unix systems. Each iteration brings new features and changes the limitations. For instance, comparing the two latest iterations, ext3 and ext4, ext4 can theoretically store files up to \SI{16}{\tebi\byte} while ext3 can store files up to \SI{2}{\tebi\byte}\,\cite{salterUnderstandingLinuxFilesystems2018}. Additionally, ext4 supports timestamps in units of nanoseconds while et3 only supports timestamp with resolution of one second. The Zettabyte filesystem (ZFS) introduces features that no version of ext supports, such as block-level cryptographic checksumming\,\cite{salterUnderstandingLinuxFilesystems2018}.
% FIXME: Why is only ZFS mentioned? Mention more? Motivate why? IDK

% TODO: Chip: "I would suggest a table of file system operations by different file systems and which one you choose to implement in FFS. Think about tables such as https://nuetzlich.net/gocryptfs/comparison/ to summarize information about alternatives."
% 	Do here as it is apparently part of the pre-study, or in summary section


\subsection{Distributed filesystems}
Filesystems are used to store data on for instance a hard drive of a computer locally or in the cloud. For example, Google Drive is a filesystem that enables users to save their data online with up to \SI{15}{\giga\byte} for free\,\cite{CloudStorageWork} using Google's clusters of distributed storage devices, meaning that the data is saved on Google's servers which can be located wherever they have data centers\,\cite{DistributedStorageWhat}. Paying customers can have a greater amount of storage using the service. Apple's iCloud and Microsoft's OneDrive are two additional examples of distributed filesystems where users have the option of free-tier and paid-tier storage.

\subsection{Data storage and encoding}
Different file types have different protocols and definitions of how they should be encoded and decoded, for instance a JPEG and a PNG file can be used to display similar content but the data they store is different. At the lowest level, storage devices often represent files as a string of binary digits no matter the file type (however there are non-binary storage devices\,\cite{MultistateDataStorage2020}, but this is outside the scope of this thesis). If one would represent an arbitrary file of $X$ bytes, each byte (0x00 - 0xFF) can be represented as a character such as the Extended ASCII (EASCII) keyset and we can therefore decode this file as $X$ different characters. Using the same set of characters for encoding and decoding we can get a symmetric relation for representing a file as a string of characters. EASCII is only one example of such a set of characters, any set of strings with $256$ unique symbols can be used to create such a symmetric relation, for instance, $256$ different emojis or a list of $256$ different words. However, if we are using a set of words we could also have to introduce a unique separator so that the words can be distinguished. If we would use a single space character as the separator, we could make the encoded text look like a text document; however, with random words one after another leading to a high probability of creating an unstructured text document. Further, if punctuation is introduced, for instance as part of some words, the text document could look like it contains random and unstructured sentences.

% FIXME: Chip: "There is a citation example shown on the page: ARC Centre of Excellence in Future Low-Energy Electronics Technologies. "Multi-state data storage leaving binary behind: Stepping 'beyond binary' to store data in more than just 0s and 1s." ScienceDaily. ScienceDaily, 12 October 2020. <www.sciencedaily.com/releases/2020/10/201012115937.htm>.More over, you could actually cite the journal paper: Qiang Cao, Weiming Lü, X. Renshaw Wang, Xinwei Guan, Lan Wang, Shishen Yan, Tom Wu, Xiaolin Wang. Nonvolatile Multistates Memories for High- Density Data Storage. ACS Applied Materials & Interfaces, 2020; 12 (38): 42449 DOI: 10.1021/acsami.0c10184"

This string of $X$ bytes can also be used as the data in an image. An image can be abstracted as a $h * w$ matrix, where each element is a pixel of a certain color. In an image with 8-bit Red-Green-Blue (RGB) color depth, each pixel consists of three 8-bit values, i.e. three bytes. One can therefore imagine that we can use this string of $X$ bytes to assign colors in this pixel matrix by assigning the first three bytes as the first pixel's color, the next three bytes as the following pixel's color, and so forth. This means that $X$ bytes of data can be represented as 
$$ceil(\frac{X}{3})$$ 
pixels, where $ceil$ rounds a float to the closest larger integer. For a file of \SI{1}{\mega\byte}, i.e. $X = 1\,000\,000$ we need $333\,334$ pixels in an image with 8-bit RGB color depth. The values of $h$ and $w$ are arbitrary but if we for instance want a square image we can set $ h\,=\,w\,=\,578$ which means that there will be $334\,084$ pixels in total, and the remaining $750$ pixels will just be fillers to make the image a reasonable size. Using filler pixels requires us to keep track of the number of bytes that we store in the image so that we do not read the filler bytes when the image is decoded. However, we could choose $h = 1$ and $w = 333\,334$ which would mean a very wide image but would not require filler pixels. 

This means that we can represent any file as a string of bytes which can then be encoded into text or as an image, which can be posted on for instance social media. However, there is a possibility that the social media services compress the images uploaded which could lead to data loss in the image, which would mean that the decoded data would be different from the encoded data. In this case we would not be able to retrieve the original data that was stored.

\section{FUSE}
\gls{FUSE} is a library that provides an interface to create filesystems in userspace rather than in kernel space which is otherwise often considered the standard when writing commercial filesystems\,\cite{Libfuse2021}. The reason to implement a filesystem in kernel space is that it leads to faster system calls than when writing a filesystem in userspace. However, while filesystems written with \gls{FUSE} are generally slower than \mbox{kernel-based} filesystems, using \gls{FUSE} simplifies the process of creating filesystems. macFUSE is a port of \gls{FUSE} that operates on Apple's macOS operating system and it extends the \gls{FUSE} \gls{API}\,\cite{HomeMacFUSE}. macFUSE provides an \gls{API} for C and Objective C.
% TODO: MAYBE add about this "If you use these extensions, then how is it portable to \mbox{non-MAC} systems?". Cannot find anything online though

Figure~\ref{fig:fuse_desc} presents an overview how \gls{FUSE} works. \gls{FUSE} consists of a kernel space part and a userspace part that perform different tasks\,\cite{vangoorFUSENotFUSE2017}. The kernel part of \gls{FUSE} operates with the \gls{VFS} which is a layer in both the Linux kernel and the macOS kernel that exposes a filesystem interface for userspace applications\,\cite{goochOverviewLinuxVirtual, singhMacOSInternals2006}. The \gls{VFS} interface is independent of the underlying filesystem and is an abstraction of the underlying filesystem operations which can be used on any filesystem the \gls{VFS} supports. The userspace part of \gls{FUSE} communicates with the kernel space part through a block device. Operations on a mounted \gls{FUSE} filesystem are sent to the \gls{VFS} from the user application, which is then sent to the kernel part of \gls{FUSE}. If needed, the operations are transmitted to the userspace part of \gls{FUSE} where the operation is handled and a response is sent back to the \gls{VFS} and the user application through the \gls{FUSE} kernel module. However, some actions can be handled by the \gls{FUSE} kernel module directly, such as if the file is cached in the kernel part of \gls{FUSE}\,\cite{vangoorFUSENotFUSE2017}. The response is then sent back to the user application from the kernel module through the \gls{VFS}.

\begin{figure}[!ht]
	\begin{center}
	  \includegraphics[width=0.5\textwidth]{figures/fuse_description.png}
	\end{center}
	\caption{Simple visualization of how \gls{FUSE} operations are executed}
	\label{fig:fuse_desc}
\end{figure}

\section{Online web services}
\label{sec:ows}
This section presents two online web services (\gls{OWS}s), Twitter and Flickr, where one can create free-tier accounts. On both of these \gls{OWS}s, free-tier accounts can make numerous posts for free. The \gls{OWS}s provide free-to-use Application Programming Interfaces (APIs) for non-commercial development. 

\subsection{Twitter}
\label{subsec:ows_twitter}
Twitter is a micro-blog online where users can sign up for a free account and create public posts (tweets) using text, images, and videos. Each post has a unique id associated with it\,\cite{twitterTwitterIDs}. Text posts are limited to $280$ characters while images can be up to \SI{5}{\mega\byte} and videos up to \SI{512}{\mega\byte}\,\cite{MediaBestPractices}. A post with images can contain up to 4 images in one post. There is also a possibility to send private messages to other accounts, where each message can contain up to $10\,000$ characters and the same limitations on files. However, direct messages older than $30$ days are not possible to retrieve through Twitter's API\,\cite{RetrievingOlder302018}. It is possible to create threads of Twitter posts where multiple tweets can be associated in chronological order.

Twitter's API defines technical limits of how many times certain actions can be executed by a user\,\cite{UnderstandingTwitterLimits}. A maximum of $2\,400$ tweets can be sent per day, and the limit is further broken down into smaller limits at semi-hourly intervals. Hitting a limit means that the user account no longer can perform the actions that the limit represents until the time period has elapsed.

\subsection{Flickr}
\label{subsec:ows_flickr}
Flickr is a public image and video hosting service used to store and share photos and videos. Unlike Twitter, a post on Flickr is based on an image or video. The post can, optionally, have a title, a description, or both. However, the post must have exactly one photo or video. Flickr supports multiple image- and video formats, including PNG and MP4\,\cite{FlickrUploadRequirements2022}. Restrictions are set for each post, depending on the media type. Images uploaded to Flickr can be a maximum of \SI{200}{\mega\byte} and a video can be a maximum of \SI{1}{\giga\byte}. Further, free-tier accounts can only have a total of 1\,000 photos or videos on their account. A Flickr Pro account has unlimited storage on Flickr but is still subject to the per-item limit of \SI{200}{\mega\byte} and \SI{1}{\giga\byte} for images and videos, respectively\,\cite{flickrinc.UpgradeEverythingYou}. Flickr Pro costs between 7.49€ to 5.49€ per month, depending on the subscription time the user signs up for. The description of a post has a limit of 65535 characters according to Shhexy Corin\,\cite{FlickrHelpForum2009}. This has been verified through testing. The title of a post has also been discovered to have a limit of 255 characters through testing.

The images and videos uploaded to Flickr are stored in their original form \textbf{without any compression} and can be downloaded by the user as the same file as was uploaded\cite{flickrinc.DownloadPermissions}. Flickr also stores other formats of the file, such as thumbnails. User accounts can restrict who, other than themselves, can download the original image. The original video can only be downloaded by the user\,\cite{flickrinc.DownloadPermissions}. Flickr does not state if it will always be possible to download the original versions of the file. Further, Flickr states that it retains the right to remove user content from the service at any time\,\cite{flickrinc.FlickrTermsConditions2020}.

The Flickr API defines a query limit of 3\,600 requests per hour, per application, across all API calls\,\cite{flickrinc.FlickrFlickrDeveloper}. However, according to Sam Judson in 2013, this is not a hard limit\,\cite{WhatAreAPI2013}. There is no official information from Flickr about what happens if you break the hourly request limit. The Flickr API states that the API is monitored on other factors as well\,\cite{flickrinc.FlickrFlickrDeveloper}. If abuse is detected, Flickr reserves the right to revoke API keys.

\section{Cryptography}
\label{sec:back_crypto}
\gls{AES} is a symmetric key encryption standard established by the \gls{NIST}, specifying the Rijndael block cipher\,\cite{kumarvermaPerformanceAnalysisRC62012}. \gls{AES} is a symmetrical cipher, meaning that the same key is used for encryption and decryption. \gls{AES} is used to make the data confidential so that no one except the person with the key can access the unencrypted data. \gls{AES} produces \mbox{128-bit} encrypted cipher blocks and supports key sizes of 128 bits, 192 bits, or 256 bits. The security of \gls{AES} has been heavily researched since its introduction in the early 2000s, and literature has found it is well resistant to quantum attacks as well\,\cite{bonnetainQuantumSecurityAnalysis2019}.

While \gls{AES} is a good standard for the confidentiality of the data, confidentiality is often not enough to secure the data\,\cite{rosswallrabensteinWhenItComes2021}. The importance of ensuring the authenticity of the data is also high. This means that we want to know that the data has not been modified since it was encrypted. This problem can be solved by using authenticated encryption\,\cite{khovratovichAnswerWhyShould2013}. \gls{GCM} is a block cipher mode of operation which provides authenticated encryption\,\cite{mcgrewGaloisCounterMode2004}. \gls{GCM} can be used together with \gls{AES} to provide secure, authenticated encryption of data. To encrypt using \gls{GCM}, the encryption function requires a key, a randomized \gls{IV}, and the data to encrypt. The output is the encrypted cipher text and an authentication tag. The decryption function of \gls{GCM} requires the same key and \gls{IV} as was used as input in the encryption function, as well as the authentication tag and the cipher text received as output by the encrypting function. Further, both the encryption function and the decryption support \gls{ADD} to be provided. \gls{ADD} is data that should be authenticated, but not encrypted. If \gls{ADD} is provided to the encryption function, it must also be provided to the decryption function.

The key used when encrypting using \gls{AES} can be derived from a password that the user provides. A \gls{PBKDF} is a function that can be used to derive a key to be used for, for instance, \gls{AES}. The input to a \gls{PBKDF} is a secret, such as a password\,\cite{kodwaniSecurityKeyDerivation2021}. An example of a \gls{PBKDF} schema is the \gls{HKDF} presented by \citeauthor{krawczykCryptographicExtractionKey2010}\,\cite{krawczykCryptographicExtractionKey2010, krawczykHMACbasedExtractandExpandKey2010} which utilizes a hashing algorithm that provide a \mbox{pseudo-random} key. \gls{HKDF} supports multiple hashing algorithms. The security of \gls{HKDF} is partially dependent on the security of the hashing algorithm used. A \mbox{well-defined} suit of hashing algorithms is the \gls{SHA}, which covers, among other hash functions, \gls{SHA}-256 \cite{hansenUSSecureHash2011}. \gls{SHA}-256 is a cryptographic hash function that outputs a \mbox{256-bit} \mbox{pseudo-random} cipher from its input, which can, for instance, be a password. Further, \gls{HKDF} uses a salt to improve the security of the provided secret. The salt is random data used to further diffuse the produced key, making two keys with the same secret but different salts, different\,\cite{ariasAddingSaltHashing2021}. The salt does not have to be secret and is sometimes stored with the produced cipher so that the decryption function can easily \mbox{re-use} the salt when deriving the decryption key. If the key used for encryption and the key used for decryption are derived using different salts, the keys will differ and the cipher text cannot be decrypted.

An alternative encryption solutions is \gls{RSA}. \gls{RSA} is an asymmetrical cipher, meaning that it uses a public key and a private key for encryption and decryption. According to \citeauthor{mahajanStudyEncryptionAlgorithms2013}, asymmetric encryption techniques are more computationally intensive than symmetrical encryption techniques and are almost 1\,000 times slower than symmetrical techniques\,\cite{mahajanStudyEncryptionAlgorithms2013}. \citeauthor{mahajanStudyEncryptionAlgorithms2013} found that \gls{AES} is a faster algorithm for encryption and decryption than \gls{RSA}, while maintaining very good security.

\section{Threats}
To consider a filesystem secure it is important to imagine different potential adversaries who might attack the system. Considering that FFS has no real control of the data stored on the different services, all the data must be considered to be stored in an insecure system. Even if we could hide the posts made on for instance Twitter by making the profile private, we must still consider that Twitter themselves could be an adversary or that they could potentially give out information, such as tweets or direct messages, to entities such as the police. Twitter's privacy policy mentions that they may share, disclose, and preserve personal information and content posted on the service, even after account deletion for up to $18$ months\,\cite{TwitterPrivacyPolicy}. Therefore, to achieve security the data stored must always be encrypted. We assume that an adversary has access to all knowledge about FFS, including how the data is converted, encrypted, and posted. We also assume they know which websites and accounts could post data from the filesystem - but we assume they do \textbf{not} have the decryption key. There are multiple secure ways of encrypting data, including AES which is one of the faster and more secure encryption algorithms\,\cite{mahajanStudyEncryptionAlgorithms2013}. However, even though the data is encrypted, other properties such as your IP address can be compromised which can expose the user's identity. The problem of these other sources of information external to FFS is not addressed in FFS but remains for future work.

Other than adversaries for FFS, we might also imagine that the underlying services might face attacks that can potentially harm the security of the system or even cause the service to go offline, potentially indefinitely. One solution is to use redundancy - by duplicating the data over multiple services, we can more confidently believe that our data will be accessible as the probability of all services going offline at the same time is lower.

The deniability of FFS is an important aspect of the filesystem. Potential threat adversaries are agents that the user is trying to hide the data from, such as governing states. For the system to be deniable, an adversary should not be able to gain any information about anything about the potential data in the system, this includes even the existence of data. When the filesystem is unmounted there should be no trace of the filesystem ever being present in the device. We will assume that an adversary is competent and can analyze the software and hardware completely.