\section{Major background area 1}
% \todo[inline, backgroundcolor=kth-lightblue]{Viktigt bakgrundsområde 1}
There are xxx characteristics that distinguish yyy from other information and communication technology (ICT) system, as shown in Figure~\ref{fig:lotsofstars}. Table \ref{tab:tablecaracteristics} summarizes these characteristics.

 
\begin{figure}[!ht]
  \begin{center}
    \includegraphics[width=0.5\textwidth]{figures/lots_of_stars.png}
  \end{center}
  \caption{Lots of stars  (Inspired by Figure x.y on page z of [xxx])}
  \label{fig:lotsofstars}
\end{figure}
% \todo[inline, backgroundcolor=kth-lightblue]{Massor av stärnor (Inspirerad av figur x.y på sidan z i [xxx])}


\begin{table}[!ht]
  \begin{center}
    \caption{xxx characteristics}
    \label{tab:tablecaracteristics}
    \begin{tabular}{l|S[table-format=4.6]} % <-- Alignments: 1st column left, 2nd middle, with vertical lines in between
      \textbf{Characteristics} & \textbf{Description}\\
      $\alpha$ & $\beta$ \\
      \hline
      1 & 1110.1\\
      2 & 10.1\\
      3 & 23.113231\\
    \end{tabular}
  \end{center}
\end{table}
% \todo[inline, backgroundcolor=kth-lightblue]{Egenskaper}
% \todo[inline, backgroundcolor=kth-lightblue]{ Beskrivning}

\subsection{Subarea 1.1}
Entangled states are an important part of quantum cryptography, but also relevant in other domains. This concept might be relevant for neutrinos, see for example \cite{kim_small-mass_2016}.

\subsection{Subarea 1.1.2}
Computational methods are increasingly used as a third method of carrying out
scientific investigations. For example, computational experiments were used to
find the amount of wear in a polyethylene liner of a hip prosthesis in \cite{maguire_jr_new_2014}.
…

\subsection{Subarea 1.1.2}
Using the nearest data center may improve performance, see \cite{bogdanov_nearest_2015}


\subsection{Link layer Encapsulation}
\label{sec:llencap}

% See Figure~\ref{fig:ieee8023-data-packet} which uses the \textsf{bytefield}  \LaTeX\ package. 


% \begin{figure}[!ht]
% 	\centering
% \begin{bytefield}{21}
% \bitbox[]{7}{} & \bitbox[]{3}{\tiny octets:} & \bitbox[]{4}{\tiny 6} & \bitbox[]{4}{\tiny 6} & \bitbox[]{3}{\tiny 2} & \bitbox[]{5}{\tiny 46 to 1500} & \bitbox[]{3}{\tiny 0 to 46} & \bitbox[]{2}{\tiny 4}\\ 

% \bitbox[]{8}{\textbf{ETHERNET \\[-1ex] \tiny{data link-layer}}} & \bitbox[]{2}{} & 

% \bitbox{4}{\tiny Destination Address} & \bitbox{4}{\tiny Source Address} & \bitbox{3}{\tiny Length/ Type} & 
% \bitbox{5}{\tiny Data Payload} & \bitbox{3}{\tiny Padding} &
% \bitbox{2}{\tiny CRC} \\

% \bitbox[]{1}{} &\bitbox[]{3}{\tiny octets:} & \bitbox[]{4}{\tiny 7} & \bitbox[]{2}{\tiny 1} & \bitbox[]{0}{$\vdots$ \\[1ex]} & \bitbox[]{16}{} & \bitbox[]{0}{$\vdots$ \\[1ex]} & \bitbox[]{5}{} & \bitbox[]{4}{\tiny Variable}\\

% \bitbox[]{4}{\textbf{MAC \\[-1ex] \tiny{packet}}} & \colorbitbox{lightgray}{4}{\tiny Preamble} & \colorbitbox{lightgray}{2}{\tiny SFD} & \colorbitbox{lightgray}{16}{\tiny MAC Client Data} & \colorbitbox{lightgray}{3}{\tiny Padding} &
% \colorbitbox{lightgray}{2}{\tiny CRC} & \colorbitbox{lightgray}{4}{\tiny Extension}
% \end{bytefield}
%      \label{fig:ieee8023-data-packet}
%      \caption{Ethernet data link layer protocol encapsulated into a IEEE~802.3 MAC packet}
% \end{figure}

\subsection{IP packet headers}
\label{sec:ipheaders}
% The data link layer will receive a packet from the IP layer. The layout of
% an IPv4 packet is shown in Figure~\ref{fig:ipv4-header}. This should be
% contrasted with the IPv6 header shown in Figure~\ref{fig:ipv6-header}.

%
% IPv4 packet header
%
% \begin{figure}[!ht]
% 	\centering
% \begin{bytefield}{32}
% \bitheader{0-31} \\
% \bitbox{4}{\footnotesize{Version}} & \bitbox{4}{IHL} & \bitbox{6}{\tiny{Type of Service}} & \bitbox{2}{{\scriptsize ECN}} \bitbox{16}{Total Length}\\
% \bitbox{16}{Identification} & \bitbox{3}{Flags} & \bitbox{13}{Fragment Offset}\\
% \bitbox{8}{Time to Live} & \bitbox{8}{Protocol} & \bitbox{16}{Header Checksum}\\
% \wordbox{1}{Source Address}\\
% \wordbox{1}{Destination Address}\\
% \colorbitbox{lightgray}{24}{Options} & \colorbitbox{lightgray}{8}{Padding}
% \end{bytefield}
%      \label{fig:ipv4-header} 
%      \caption[IPv4 datagram header]{IPv4 datagram header. Light grey coloured fields are optional.}
% \end{figure}

%
% IPv6 packet header
%
% \begin{figure}[!ht]
% 	\centering
% \begin{bytefield}{32}
% \bitheader{0-31} \\
% \bitbox{4}{\footnotesize{Version}} & \bitbox{8}{Traffic Class} & \bitbox{20}{Flow Label}\\
% \bitbox{16}{Payload Length} & \bitbox{8}{Next Header} & \bitbox{8}{Hop Limit}\\
% \wordbox{4}{Source Address}\\
% \wordbox{4}{Destination Address}\\
% \end{bytefield}
%      \label{fig:ipv6-header}
%      \caption{IPv6 datagram header}
% \end{figure}

\subsection{Test for accessibility of formulas}

As can be seen in these equations:
$c=2 \cdot \pi \cdot r$ or \[ \int_{a}^{b} x^2 \,dx \] a chemical formula: $(C_5O_2H_8)_n$
...