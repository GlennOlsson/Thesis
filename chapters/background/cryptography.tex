\section{Cryptography}
\label{sec:back_crypto}
\gls{AES} is an encryption standard established by the \gls{NIST}, more specificity specifying the Rijndael block cipher\,\cite{kumarvermaPerformanceAnalysisRC62012}. \gls{AES} is a symmetrical cipher, meaning that the same key is used for encryption and decryption. \gls{AES} is used to make the data confidential so that no one except the person with the key can access the unencrypted data. \gls{AES} produces 128-bit encrypted cipher blocks and supports key sizes of 128 bits, 192 bits, or 256 bits. The security of \gls{AES} has been heavily researched since its introduction in the early 2000s, and literature has found it is well resistant to quantum attacks as well\,\cite{bonnetainQuantumSecurityAnalysis2019}.

While \gls{AES} is a good standard for the confidentiality of the data, confidentiality is often not enough to secure the data\,\cite{rosswallrabensteinWhenItComes2021}. The importance of ensuring the authenticity of the data is also high. This means that we want to know that the data has not been modified since it was encrypted. This problem can be solved by using authenticated encryption\,\cite{khovratovichAnswerWhyShould2013}. \gls{GCM} is a block cipher mode of operation which provides authenticated encryption\,\cite{mcgrewGaloisCounterMode2004}. \gls{GCM} can be used with \gls{AES} to provide secure, authenticated encryption of data. To encrypt using \gls{GCM}, the encryption function requires a key, a randomized \gls{IV} and the data to encrypt. The output is the encrypted cipher text and an authentication tag. The decryption function of \gls{GCM} requires the same key and \gls{IV} as was used as input in the encryption function, as well as the authentication tag and the cipher text received as output by the encrypting function. Further, both the encryption function and the decryption support \gls{ADD} to be provided. \gls{ADD} is data that should be authenticated, but not encrypted. If \gls{ADD} is provided to the encryption function, it must also be provided to the decryption function.

The key used when encrypting using \gls{AES} is often derived from a password that the user provides. \gls{PBKDF}s are functions that can be used to derive a key used for, for instance, \gls{AES}. The input to a \gls{PBKDF} is a secret, such as a password\,\cite{kodwaniSecurityKeyDerivation2021}. An example of a \gls{PBKDF} schema is the \gls{HKDF} presented by \citeauthor{krawczykCryptographicExtractionKey2010}\,\cite{krawczykCryptographicExtractionKey2010}\cite{krawczykHMACbasedExtractandExpandKey2010} which utilizes a hashing algorithm that provide a pseudo-random key. \gls{HKDF} supports multiple hashing algorithms. The security of \gls{HKDF} is partially dependent on the security of the hashing algorithm used. A well-defined suit of hashing algorithms is the \gls{SHA}, which covers, among other hash functions, \gls{SHA}-256 \cite{hansenUSSecureHash2011}. \gls{SHA}-256 is a cryptographic hash function that outputs a 256-bit pseudo-random cipher from its input, which can, for instance, be a password. Further, \gls{HKDF} uses a salt to improve the security of the provided secret. The salt is random data used to further diffuse the produced key, making two keys with the same secret but different salts, different\,\cite{ariasAddingSaltHashing2021}. The salt does not have to be secret and is sometimes stored with the produced cipher so that the decryption function easily can re-use the salt when deriving the decryption key. If the key used for encryption and the key used for decryption are derived using different salts, the keys will differ and the cipher cannot be decrypted.

Alternative encryption solutions are, among others, \gls{RSA} and \gls{DES}. \gls{RSA} is an asymmetrical cipher, meaning that it uses a public key and a private key for encryption and decryption. According to \citeauthor{mahajanStudyEncryptionAlgorithms2013}, asymmetric encryption techniques are more computationally intensive than symmetrical encryption techniques and are almost 1\,000 times slower than symmetrical techniques\,\cite{mahajanStudyEncryptionAlgorithms2013}. \citeauthor{mahajanStudyEncryptionAlgorithms2013} found that \gls{AES} is the fastest algorithm for encryption and decryption between \gls{RSA}, \gls{DES}, and \gls{AES} while maintaining very good security. This further proves \gls{AES} to be a good choice as the cryptography technique for \gls{FFS}.