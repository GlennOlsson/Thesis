\section{Cryptography}
\label{sec:back_crypto}
The Advanced Encryption Standard (\gls{AES}) is a encryption standard established by the U.S. National Institute of Standards and Technology (\gls{NIST}), more specificity specifying the Rijndael block cipher\,\cite{kumarvermaPerformanceAnalysisRC62012}. AES is a symmetrical cipher, meaning that the same key is used for encryption and decryption. AES is used to make the data confidential, so that no one except the person with the key can access the unencrypted data. AES produces 128-bit encrypted cipher blocks, and supports key sizes of 128 bits, 192 bits, or 256 bits. The security of AES has been heavily researched since its introduction in the early 2000s, and literature has found it is well resistant to quantum attacks as well\,\cite{bonnetainQuantumSecurityAnalysis2019}.

While AES is a good standard for the confidentiality of the data, confidentiality is often not enough to secure the data\,\cite{rosswallrabensteinWhenItComes2021}. Importance of ensuring the authenticity of the data is also high. This means that we want to know that the data has not been modified since it was encrypted. This problem can be solved by using authenticated encryption\,\cite{khovratovichAnswerWhyShould2013}. The Galois/counter mode (\gls{GCM}) is a block cipher mode of operation which provides authenticated encryption\,\cite{mcgrewGaloisCounterMode2004}. GCM can be used with AES to provide secure, authenticated encryption of data. To encrypt using GCM, the encryption function requires a key, a randomized Initialization Vector (\gls{IV}) and the data to encrypt. The output is the encrypted cipher text and an authentication tag. The decryption function of GCM requires the same key and IV as was used as input in the encryption function, as well as the authentication tag and the cipher text received as output by the encrypting function. Further, both the encryption function and the decryption support Additional authentication data (\gls{ADD}) to be provided. ADD is data that should be authenticated, but not encrypted. If ADD is provided to the encryption function, it must also be provided to the decryption function.

The key used when encrypting using AES is often derived from a password that the user provides. Password-Based Key Derivation Functions (\gls{PBKDF}s) are functions that can be used to derive a key used for, for instance, AES. The input to a PBKDF is a secret, such as a password\,\cite{kodwaniSecurityKeyDerivation2021}. An example of a PBKDF schema is the hashed message authentication code (\gls{HMAC}) based key derivation functions (\gls{HKDF}) presented by \citeauthor{krawczykCryptographicExtractionKey2010}\,\cite{krawczykCryptographicExtractionKey2010}\cite{krawczykHMACbasedExtractandExpandKey2010} which utilizes a hashing algorithm that provide a pseudo-random key. HKDF supports multiple hashing algorithms. The security of HKDF is partially dependent on the security of the hashing algorithm used. A well-defined suit of hashing algorithms is the Secure Hash Algorithms (\gls{SHA}), which covers, among other hash functions, SHA-256 \cite{hansenUSSecureHash2011}. SHA-256 is a cryptographic hash function which outputs a 256-bit pseudo-random cipher from its input, which can, for instance, be a password. Further, HKDF uses a salt to improve the security of the provided secret. The salt is random data used to further diffuse the produced key, making two keys with the same secret but different salts, different\,\cite{ariasAddingSaltHashing2021}. The salt does not have to be secret, and is sometimes stored with the produced cipher so that the decryption function easily can re-use the salt when deriving the decryption key. If the key used for encryption and the key used for decryption are derived using different salts, the keys will differ and the cipher cannot be decrypted.
