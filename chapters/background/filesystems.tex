\section{Filesystems and data storage}
This section presents how certain filesystems used today are structured. We present the idea of \mbox{inode-based} filesystems and distributed filesystems. Following, we describe how data is stored in a storage system and how this information can be used in \gls{FFS}.

\subsection{Unix filesystems}
A Unix filesystem uses a data structure called an \textit{inode}. The inodes are found in an inode table and each inode keeps track of the size, blocks used for the file's data, and metadata for the files in the filesystem. A directory simply contains the filenames and each file or directory's inode id. The system can with an inodeIDfind information about the file or directory using the inode table. Each inode can contain any metadata that might be relevant for the system, such as creation time and last update time. 

Figure~\ref{fig:inode_diag} shows an example inode filesystem and how it can be visualized. The blocks of an inode entry are where in the storage device the data is stored, each block is often defined as a certain number of bytes. Listing~\ref{lst:inode_fs} describes a simple implementation of an inode, an inode table, and directory entries. 

\begin{figure}[!ht]
	\begin{center}
	  \includegraphics[width=0.5\textwidth]{figures/inode_diagram.png}
	\end{center}
	\caption{Basic structure of \mbox{inode-based} filesystem}
	\label{fig:inode_diag}
\end{figure}

\begin{minipage}{\linewidth}
\begin{lstlisting}[language=c, caption={Pseudocode of a minimalistic inode filesystem structure}, label=lst:inode_fs]
struct inode_entry {
	int 	length
	int[]	blocks
	// Metadata attributes are defined here
}

struct directory_entry {
	char*   filename
	int     inode
}

// Maps inode_id to an inode_entry
map<int, inode_entry> inode_table

\end{lstlisting}
\end{minipage}

Different filesystems provide different features and limitations. The Extended Filesystem (ext) exists in four different versions: ext, ext2, ext3, and ext4. This filesystem is often used on Unix systems. Each iteration brings new features and changes the limitations. For instance, comparing the two latest iterations, ext3 and ext4, ext4 can theoretically store files up to \SI{16}{\tebi\byte} while ext3 can store files up to \SI{2}{\tebi\byte}\,\cite{salterUnderstandingLinuxFilesystems2018}. Additionally, ext4 supports timestamps in units of nanoseconds while et3 only supports timestamps with a resolution of one second. Additionally. ext4 natively supports encryption at the directory level through the use of the fscrypt \gls{API}\,\cite{FscryptArchWiki}.

\gls{APFS} is a modern filesystem that is used on iPhones and Macs and can store files with a size up to \SI{9}{\exa\byte}\,\cite{igotofferAPFSAppleFile2017}. It supports timestamps in units of nanoseconds and is built to be used on \gls{SSD}\,\cite{nelsonWhatAPFSDoes}. It also supports modern features that its predecessor Mac OS Extended (HFS+) does not support, such as Snapshots and Space Sharing. \gls{APFS} natively supports encryption of the filesystem volume\,\cite{appleinc.FileSystemFormats}.

\subsection{Distributed filesystems}
Filesystems are used to store data, for instance locally on a hard drive of a computer, or in the cloud. Google Drive is an example of a filesystem that enables users to save their data online with up to \SI{15}{\giga\byte} for free\,\cite{CloudStorageWork} using Google's clusters of distributed storage devices, meaning that the data is saved on Google's servers which can be located wherever they have data centers\,\cite{DistributedStorageWhat}. Paying customers can have a greater amount of storage using the service. Apple's iCloud and Microsoft's OneDrive are two additional examples of distributed filesystems where users have the option of \mbox{free-tier} and \mbox{paid-tier} storage.

Cloud-based filesystems, as opposed to a filesystem on a physical disk, are accessible from multiple computers and devices without requiring the user to connect a physical disk to the computer. Instead, as the filesystem is accessible through the internet, it can be accessed regardless of the user's location and on multiple devices, as long as a connection to the filesystem can be established. Thus, even if the user would lose their computer or if it would malfunction, the data on the \mbox{cloud-based} filesystem can still be accessed which means that the data could still be recovered. These filesystems are often owned by companies, such as Google Drive and Apple's iCloud, as they are big companies that can provide reliable storage. This also means that they have their own agenda and policies, and as they are hosting the data they have the possibility of accessing your data. The data is often encrypted, but in the case of Google Drive, they have access and control of the encryption and decryption keys which in turn means that they have access and control of the data stored\,\cite{johnsonGoogleDriveSecure2021}. While they mention in their Terms of Service that the user retains ownership of the data\,\cite{googleGoogleDriveTerms}, they also mention that they can disclose your data for legal reasons and that they retain the right to review the content uploaded by users\,\cite{googleGoogleTermsService}. Controlling the encryption and decryption keys also enables the possibility of hackers gaining access to your data by attacking Google. iCloud uses \mbox{end-to-end} encryption for some parts of the service, but not for the whole suite\,\cite{appleinc.ICloudSecurityOverview}. For instance, backup data and iCloud drive are not \mbox{end-to-end} encrypted while the Keychain and Memoji data are.

\subsection{Data storage and encoding}
\label{sec:data_storage}
Different file types have different formats that describe of how they should be encoded and decoded, for instance, a JPEG and a PNG file can be used to display similar content but the data they store is different. At the lowest level, storage devices often represent files as a string of binary digits no matter the file type (however, there are \mbox{non-binary} storage devices\,\cite{MultistateDataStorage2020}, but this is outside the scope of this thesis).

A file as (string of bytes) can be encoded into text or as an image by representing the bytes as pixel color data and adding a suitable image header for a given image format. This image can in turn be posted on, for instance, social media. However, there is a possibility that the social media services compress the uploaded images which could lead to data loss in the image, which would mean that the decoded data would be different from the encoded data. In this case, we would not be able to retrieve the original data that was stored unless we use methods such as \mbox{error-correcting} codes. The \mbox{error-correcting} codes would have to be stored in an ensured lossless format. For instance, text posts on the \gls{OWS} can be used as long as the posts are not removed or the text is modified.