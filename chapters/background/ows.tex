\section{Online web services}
\label{sec:ows}
This section presents two \gls{OWS}s, Twitter and Flickr, where one can create \mbox{free-tier} accounts. On both of these \gls{OWS}s, \mbox{free-tier} accounts can make numerous posts for free. The \gls{OWS}s each provide a \mbox{free-to-use} \gls{API} for \mbox{non-commercial} development. 

\subsection{Twitter}
\label{subsec:ows_twitter}
Twitter is a \mbox{micro-blog} online where users can sign up for a free account and create public posts (tweets) using text, images, and videos. Each post has a unique id associated with it\,\cite{twitterTwitterIDs}. Text posts are limited to $280$ characters while images can be up to \SI{5}{\mega\byte} and videos up to \SI{512}{\mega\byte}\,\cite{MediaBestPractices}. A post with images can contain up to 4 images in one post. There is also a possibility to send private messages to other accounts, where each message can contain up to $10\,000$ characters and the same limitations on files. However, direct messages older than $30$ days are not possible to retrieve through Twitter's \gls{API}\,\cite{RetrievingOlder302018}. It is possible to create threads of Twitter posts where multiple tweets can be associated in chronological order.

Twitter's \gls{API} defines technical limits of how many times certain actions can be executed by a user\,\cite{UnderstandingTwitterLimits}. A maximum of $2\,400$ tweets can be sent per day, and the limit is further broken down into smaller limits at \mbox{semi-hourly} intervals. Hitting a limit means that the user account no longer can perform the actions that the limit represents until the time period has elapsed.

\subsection{Flickr}
\label{subsec:ows_flickr}
Flickr is a public image and video hosting service used to store and share photos and videos. Unlike Twitter, a post on Flickr is based on an image or video. The post can, optionally, have a title, a description, or both. However, the post must have exactly one photo or video. Flickr supports multiple image- and video formats, including PNG and MP4\,\cite{FlickrUploadRequirements2022}. Size restrictions are set for each post, depending on the media type. Images uploaded to Flickr can be a maximum of \SI{200}{\mega\byte} and a video can be a maximum of \SI{1}{\giga\byte}. Further, \mbox{free-tier} accounts can only have a total of 1\,000 photos or videos on their account. A Flickr Pro account has unlimited storage on Flickr but is still subject to the \mbox{per-item} limit of \SI{200}{\mega\byte} and \SI{1}{\giga\byte} for images and videos, respectively\,\cite{flickrinc.UpgradeEverythingYou}. Flickr Pro costs between EUR~7.49 to EUR~5.49 per month, depending on the subscription time the user signs up for. The description of a post has a limit of 65535 characters according to Shhexy Corin\,\cite{FlickrHelpForum2009}. This has been verified through testing. The title of a post has also been discovered through testing to have a limit of 255 characters.

The images and videos uploaded to Flickr are stored in their original form \textbf{without any compression} and can be downloaded by the user as the same file as was uploaded\cite{flickrinc.DownloadPermissions}. Flickr also stores other formats of the file, such as thumbnails. User accounts can restrict who, other than themselves, can download the original image. Restricting who can download the file helps ensure that \mbox{no-one} else can read the original file data, but also requires the user to authenticate with Flickr to download the image meaning it is not possible to anonymously download the image data. However, a restricted original image can still be downloaded by a knowledgeable person\,\cite{FlickrHelpForum2020}. The original video can only be downloaded by the user\,\cite{flickrinc.DownloadPermissions}. Flickr does not state if it will always be possible to download the original versions of the file. Further, Flickr states that it retains the right to remove user content from the service at any time\,\cite{flickrinc.FlickrTermsConditions2020}.

The Flickr \gls{API} defines a query limit of 3\,600 requests per hour, per application, across all \gls{API} calls\,\cite{flickrinc.FlickrFlickrDeveloper}. However, according to Sam Judson in 2013, this is not a hard limit\,\cite{WhatAre\gls{API}2013}. There is no official information from Flickr about what happens if you break the hourly request limit. The Flickr \gls{API} states that the \gls{API} is monitored on other factors as well\,\cite{flickrinc.FlickrFlickrDeveloper}. If abuse is detected, Flickr reserves the right to revoke \gls{API} keys.