\section{Online web services}
\label{sec:ows}
This section presents two online web services (OWSs), Twitter and Flickr, where one can create free-tier accounts. On both of these OWSs, free-tier accounts can make numerous of posts for free. The OWSs provide free-to-use Application Programming Interfaces (APIs) for non-commercial development. 

\subsection{Twitter}
\label{subsec:ows_twitter}
Twitter is a micro-blog online where users can sign up for a free account and create public posts (tweets) using text, images, and videos. Each post has a unique id associated with it\,\cite{twitterTwitterIDs}. Text posts are limited to $280$ characters while images can be up to \SI{5}{\mega\byte} and videos up to \SI{512}{\mega\byte}\,\cite{MediaBestPractices}. An post with images can contain up to 4 images in one post. There is also a possibility to send private messages to other accounts, where each message can contain up to $10\,000$ characters and the same limitations on files. However, direct messages older than $30$ days are not possible to retrieve through Twitter's API\,\cite{RetrievingOlder302018}. It is possible to create threads of Twitter posts where multiple tweets can be associated in chronological order.

Twitter's API defines technical limits of how many times certain actions can be executed by a user\,\cite{UnderstandingTwitterLimits}. A maximum of $2\,400$ tweets can be sent per day, and the limit is further broken down into smaller limits at semi-hourly intervals. Hitting a limit means that the user account no longer can perform the actions that the limit represents until the time period has elapsed.

\subsection{Flickr}
\label{subsec:ows_flickr}
Flickr is a public image and video hosting service, used to store and share photos and videos. Unlike Twitter, a post on Flickr is base d on the image or video. The post can, optionally, have a title, a description, or both. However, the post must have exactly one photo or video. Flickr supports multiple image and video formats, including PNG and MP4\,\cite{FlickrUploadRequirements2022}. Restrictions are set for each post, depending on the media type. Images uploaded to Flickr can be a maximum of \SI{200}{\mega\byte} and a video can be maximum of \SI{1}{\giga\byte}. Further, free-tier accounts can only have total of 1\,000 photos or videos on their account. A Flickr Pro account has unlimited storage on Flickr, but is still subject to the per-item limit of \SI{200}{\mega\byte} and \SI{1}{\giga\byte} for images and videos, respectively\,\cite{flickrUpgradeEverythingYou}. Flickr Pro costs between 7.49€ to 5.49€ per month, depending on the subscription time the user signs up for. The title of a post can be a maximum of 255 characters, and the description has a limit of 65535. \textbf{DO I NEED TO REFERENCE THIS? I HAVE A FORUM POST FROM A USER STATING THE DESCRIPTION LIMIT, AND THROUGH MY OWN TESTING I CAN CONCLUDE THESE LIMITS. SHOULD I MENTION THIS IN METHODOLOGY INSTEAD?}

The images and videos uploaded to Flickr is stored in its original form without any compression, and can be downloaded by the user as the same file as was uploaded\cite{flickrDownloadPermissions}. Flickr also stores other formats of the file, such as thumbnails. User accounts can restrict who, other than themselves, can download the original image. The original video can only be downloaded by the user\,\cite{flickrDownloadPermissions}.

The Flickr API defines a query limit of 3\,600 requests per hour, per application, across all API calls\,\cite{flickrFlickrFlickrDeveloper}. However, according to the user Sam Judson in 2013, this is not a hard limit\,\cite{WhatAreAPI2013}. There is no official information from Flickr of what happens if you break the request hourly limit. The Flickr API states that the API is monitored on other factors as well\,\cite{flickrFlickrFlickrDeveloper}. If abuse is detected, Flickr reserves the right to revoke API keys.