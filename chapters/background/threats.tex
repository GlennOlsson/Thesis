\section{Threats}
To consider a filesystem secure it is important to imagine different potential adversaries who might attack the system and. Considering that FFS has no real control of the data stored on the different services, all the data must be considered to be stored in an on an unsecure system. Even if we could hide the posts made on for instance twitter by making the profile private, we must still consider that twitter could be an adversary and therefore the data stored must always be unreadable without the correct authentication. We assume that any adversary has access to all knowledge about FFS, including how the data is converted, encrypted and posted. There are multiple secure ways of encrypting data, including AES which is one of the faster and secure encryption algorithms\cite{mahajanStudyEncryptionAlgorithms2013}.

Other than adversaries for just FFS, we might also imagine that the underlying services might receive attacks that can potentially harm the security of the system or even have it go offline indefinitely. One solution is to use redundancy - by duplicating the data over multiple services we can more confidently believe that our data will be accessible as all services will probably not go offline. 