\section{Threats}
To consider a filesystem secure it is important to imagine different potential adversaries who might attack the system. Considering that \gls{FFS} has no real control of the data stored on the different services, all the data must be considered to be stored in an insecure system. Even if we could hide the posts made on the online web service, for instance, Twitter, by making the profile private, we must still consider that Twitter themselves could be an adversary or that they could potentially give out information, such as tweets or direct messages, to entities such as the police. Twitter's privacy policy mentions that they may share, disclose, and preserve personal information and content posted on the service, even after account deletion for up to $18$ months\,\cite{TwitterPrivacyPolicy}. Therefore, to achieve security the data stored must always be encrypted. We assume that an adversary has access to all knowledge about \gls{FFS}, including how the data is converted, encrypted, and posted. We also assume they know which websites and accounts could host data from the filesystem - but we assume they do \textbf{not} have the decryption key. However, even though the data is encrypted, other properties such as your IP address can be known which can expose the user's identity. The problem of these other sources of information external to \gls{FFS} is not addressed in \gls{FFS} but remains for future work.

Other than adversaries for \gls{FFS}, we might also imagine that the underlying services might face attacks that can potentially harm the security of the system or even cause the service to go offline, potentially indefinitely. One solution is to use redundancy - by duplicating the data over multiple services, we can more confidently believe that our data will be accessible as the probability of all services going offline at the same time is lower.

The deniability of \gls{FFS} is an important aspect of the filesystem. Potential threat adversaries are agents that the user is trying to hide the data from, such as governing states. For the system to be completely deniable, an adversary should not be able to gain any information about the potential data in the system, this includes even the existence of data. When \gls{FFS} is unmounted there should be no trace of \gls{FFS} ever being present in the device. We will assume that an adversary is competent and can analyze the software and hardware completely. We assume that the adversary can gain access to the user's computer where \gls{FFS} has been mounted previously, but that they do not have access to the machine while \gls{FFS} is mounted. It is assumed that the adversary might have snapshots of the user's computer before and after \gls{FFS} was mounted, but that no snapshots were taken while \gls{FFS} was mounted. For instance, a country's border agents might take a snapshot of the computer's storage device every time the user passes through the border, but the user might mount \gls{FFS} during the time inside the country.