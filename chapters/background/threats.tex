\section{Threats}
To consider a filesystem secure it is important to imagine different potential adversaries who might attack the system. Considering that FFS has no real control of the data stored on the different services, all the data must be considered to be stored in an insecure system. Even if we could hide the posts made on for instance Twitter by making the profile private, we must still consider that Twitter could be an adversary or that they could potentially give out information such as tweets or direct messages to entities such as the police. In fact, Twitter's privacy policy mentions that they may share, disclose and preserve personal information and content posted on the service\cite{TwitterPrivacyPolicy}. Therefore the data stored must always be unreadable without the correct authentication. We assume that an adversary has access to all knowledge about FFS, including how the data is converted, encrypted, and posted. There are multiple secure ways of encrypting data, including AES which is one of the faster and more secure encryption algorithms\cite{mahajanStudyEncryptionAlgorithms2013}. However, even though the data is encrypted, other properties such as your IP address can be compromised which can expose the user's identity. This problem is however not addressed in FFS but is something for future work.

Other than adversaries for just FFS, we might also imagine that the underlying services might receive attacks that can potentially harm the security of the system or even have it go offline indefinitely. One solution is to use redundancy - by duplicating the data over multiple services we can more confidently believe that our data will be accessible as all services will probably not go offline. 

% TODO: Mention possibility of twitter giving out IP etc. to police?