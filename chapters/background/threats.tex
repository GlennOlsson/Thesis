\section{Threats}
To consider a filesystem secure it is important to imagine different potential adversaries who might attack the system. Considering that FFS has no real control of the data stored on the different services, all the data must be considered to be stored in an insecure system. Even if we could hide the posts made on for instance Twitter by making the profile private, we must still consider that Twitter themselves could be an adversary or that they could potentially give out information, such as tweets or direct messages, to entities such as the police. In fact, Twitter's privacy policy mentions that they may share, disclose, and preserve personal information and content posted on the service, even after account deletion for up to $18$ months\,\cite{TwitterPrivacyPolicy}. Therefore, to achieve security the data stored must always be encrypted. We assume that an adversary has access to all knowledge about FFS, including how the data is converted, encrypted, and posted. We also assume they know which websites and accounts that could post data from the filesystem - but we assume they do \textbf{not} have the decryption key. There are multiple secure ways of encrypting data, including AES which is one of the faster and more secure encryption algorithms\,\cite{mahajanStudyEncryptionAlgorithms2013}. However, even though the data is encrypted, other properties such as your IP address can be compromised which can expose the user's identity. The problem of these other sources of information external to FFS is not addressed in FFS but remains for future work.

Other than adversaries for FFS, we might also imagine that the underlying services might face attacks that can potentially harm the security of the system or even cause the service to go offline, potentially indefinitely. One solution is to use redundancy - by duplicating the data over multiple services, we can more confidently believe that our data will be accessible as the probability of all services going offline at the same time is lower.

The deniability of FFS is an important aspect of the filesystem. Potential threat adversaries are agents that the user is trying to hide the data from, such as governing states. For the system to be deniable, an adversary should not be able to gain any information about anything about the potential data in the system, this includes even the existence of data. When the filesystem is unmounted there should be no trace of the filesystem ever being present in the device. We will assume that an adversary is competent and can analyze the software and hardware completely.