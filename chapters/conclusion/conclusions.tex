
\section{Conclusions}
\label{sec:conclusions}
% \todo[inline, backgroundcolor=kth-lightblue]{Slutsatser}
% \todo[inline]{Describe the conclusions (reflect on the whole introduction given in Chapter 1).}
  
% \todo[inline]{Discuss the positive effects and the drawbacks.\\
% Describe the evaluation of the results of the degree project.\\
% Did you meet your goals?\\
% What insights have you gained?\\
% What suggestions can you give to others working in this area?\\
% If you had it to do again, what would you have done differently?}

% \todo[inline, backgroundcolor=kth-lightblue]{Uppfyllde du dina mål?\\
% Vilka insikter har du fått?\\
% Vilka förslag kan du ge till andra som arbetar inom detta område?
% Om du skulle göra detta igen, vad skulle du ha gjort annorlunda?}

FFS is a cryptographic and deniable cloud-based filesystem with free storage through exploiting online web services. Compared to other filesystems, \gls{FFS} is slow and is not suitable as a multi-purpose filesystem, for instance as a hard drive for a computer. It performed poorly even compared to another cloud-based filesystem, \gls{GCSF}. However, one key difference between these two filesystems is that \gls{FFS} manages the cryptography of the filesystem, while \gls{GCSF} delegates this task to Google Drive. This provides security benefits for \gls{FFS}, but might also contribute to the slower computation time. The results also show that even when removing the dependency of an internet connection is \gls{FFS} performing poorly, especially for the read operations compared to \gls{GCSF} and \gls{APFS}. The write operations of \gls{FFFS} perform better than \gls{GCSF} and \gls{FFS}. The read operations of \gls{FFS} and \gls{FFFS} are more similar than the write operations, however, \gls{FFFS} outperforms \gls{FFS} at every read operation as well leading to the conclusion that the internet connection and the \gls{OWS} influence the file operations significantly. With better read performance than write performance, \gls{FFS} is best suited as a many-read-few-write filesystem.

While the filesystem is slow, it provides security aspects such as end-to-end encryption and deniability. As long as the filesystem is not mounted to the computer, it is not possible to prove how much data is stored on \gls{FFS}, or even prove that data is stored on \gls{FFS}. End-to-end cryptography provides the user with confidential data. Further, by using authenticated encryption, \gls{FFS} provides the user with proof of the authenticity of the data it stores. 