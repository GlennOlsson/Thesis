
\section{Conclusions}
\label{sec:conclusions}
% \todo[inline, backgroundcolor=kth-lightblue]{Slutsatser}
% \todo[inline]{Describe the conclusions (reflect on the whole introduction given in Chapter 1).}
  
% \todo[inline]{Discuss the positive effects and the drawbacks.\\
% Describe the evaluation of the results of the degree project.\\
% Did you meet your goals?\\
% What insights have you gained?\\
% What suggestions can you give to others working in this area?\\
% If you had it to do again, what would you have done differently?}

% \todo[inline, backgroundcolor=kth-lightblue]{Uppfyllde du dina mål?\\
% Vilka insikter har du fått?\\
% Vilka förslag kan du ge till andra som arbetar inom detta område?
% Om du skulle göra detta igen, vad skulle du ha gjort annorlunda?}

\gls{FFS} is a cryptographic and deniable \mbox{cloud-based} filesystem with free storage through exploiting an \gls{OWS}. Compared to other filesystems, \gls{FFS} is slow and is unsuitable as a \mbox{multi-purpose} filesystem, for instance as replacement for a hard drive for a computer. It performs poorly even when compared to another \mbox{cloud-based} filesystem, \gls{GCSF}. However, one key difference between these two filesystems is that \gls{FFS} manages the cryptography of the filesystem, while \gls{GCSF} delegates this task to Google Drive. This provides security benefits for \gls{FFS}, but might contribute to its lower performance. The benchmarking results also show that even when removing the dependency on an internet connection, \gls{FFS} performs poorly, especially for read operations when compared to \gls{GCSF} and \gls{APFS}. Not surprisingly, the write operations of \gls{FFFS} perform better than \gls{GCSF} and \gls{FFS} - as there is not network latency or remote service time. The read operations of \gls{FFS} and \gls{FFFS} are more similar than the write operations; however, \gls{FFFS} outperforms \gls{FFS} for every read operation test leading to the conclusion that the internet connection and the \gls{OWS} greatly influence the performance of the file operations. With better read performance than write performance, \gls{FFS} is best suited as a \mbox{many-read-few-write} filesystem.

While the filesystem is slow, it provides security aspects such as \mbox{end-to-end} encryption and deniability. As long as the filesystem is not mounted on the computer, it is not possible to prove how much data is stored via \gls{FFS}, or even prove that data is stored on \gls{FFS}. However, it is possible to estimate an upper limit on the amount of data stored. \mbox{End-to-end} cryptography provides the user with confidential data. Further, by using authenticated encryption, \gls{FFS} provides the user with proof of the authenticity of the data it stores. 

%TODO: Possible to get an estimate of how much data is stored?