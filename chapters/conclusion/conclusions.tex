
\section{Conclusions}
\label{sec:conclusions}
% \todo[inline, backgroundcolor=kth-lightblue]{Slutsatser}
% \todo[inline]{Describe the conclusions (reflect on the whole introduction given in Chapter 1).}
  
% \todo[inline]{Discuss the positive effects and the drawbacks.\\
% Describe the evaluation of the results of the degree project.\\
% Did you meet your goals?\\
% What insights have you gained?\\
% What suggestions can you give to others working in this area?\\
% If you had it to do again, what would you have done differently?}

% \todo[inline, backgroundcolor=kth-lightblue]{Uppfyllde du dina mål?\\
% Vilka insikter har du fått?\\
% Vilka förslag kan du ge till andra som arbetar inom detta område?
% Om du skulle göra detta igen, vad skulle du ha gjort annorlunda?}

FFS is a cryptographic and deniable cloud-based filesystem with free storage through exploiting online web services. Compared to other filesystems, FFS is slow and is not suitable as a multi-purpose filesystem, for instance as a hard drive for a computer. It performed poorly even compared to another cloud-based filesystem, GCSF. However, one key difference between these two filesystems is that FFS manages the cryptography of the filesystem, while GCSF delegates this task to Google Drive. This provides security benefits for FFS, but might also contribute to the slower computation time. The results also show that, even when removing the dependency of an internet connection is FFS performing poorly, especially for the read operations compared to GCSF and APFS. The write operations of FFFS perform better than GCSF and FFS. The read operations of FFS and FFFS are more similar than the write operations, however, FFFS outperforms FFS at every read operation as well leading leading to the conclusion that the internet connection and the OWS influences the file operations significantly. With better read performance that write performance, FFS is best suited as a many-read-few-write filesystem.

While the filesystem is slow, it provides security aspects such as end-to-end encryption and deniability. As long as the filesystem is not mounted to the computer, it is not possible to prove how much data is stored on FFS, or even prove that data is stored on FFS. The end-to-end cryptography provides the user with confidential data. Further, by using authenticated encryption, FFS provides the user with proof of authenticity of the data it stores. 