
\section{Future work}
\label{sec:futureWork}
As mentioned previously, \gls{FFS} does not implement all of the features that the POSIX standard defines. Future development for \gls{FFS} could be to implement more of these functions, such as links and file permissions. This could make \gls{FFS} further resemble a regular filesystem. Another improvement could be to move from userspace using \gls{FUSE}, to kernel space. This could speed up filesystem operations. Another feature that could be interesting to evaluate is the possibility to share files with other users, similar to Google Drive.

Although the files are encrypted so that the data is confidential, further research could include hiding the user's online activity through the use of a distributed service to anonymize traffic origins, for instance, Tor. Currently, the anonymity of the user is not considered but for \gls{FFS} to further increase plausibly deniable, this should be addressed as the user could otherwise be identified by based upon their IP address and other online fingerprints that could be provided by the \glspl{OWS} and \glspl{ISP}.

To improve the dependability and increase the storage capacity of \gls{FFS}, support for multiple \glspl{OWS} could be implemented. For instance, GitHub provides free user accounts with many gigabytes of storage. Even \mbox{free-tier} distributed filesystems, such as Google Drive, could be utilized. If multiple user accounts are used in coordination over multiple \glspl{OWS}, \gls{FFS} could achieve even more storage. Future work includes comparing such a filesystem with the current state of \gls{FFS}.

To further increase the storage capacity, \gls{FFS} could take advantage of storing videos on the \gls{OWS} as well. Flickr allows videos up to \SI[per-mode = symbol]{1}{\giga\byte} on its service. Future work could include researching how much steganographic data can be stored in videos and how efficient a filesystem using encoded videos could be.

If the \gls{OWS} would pursue identifying \gls{FFS} images stored on their service to remove them, this would be a problem for \gls{FFS}. Even removing a single image could remove the full functionality of the filesystem. Future work should include finding evasion techniques to hide the encoded data even further. For instance, hiding less data per image is a possibility that could enable the images to look like actual images, such as photographs. This is similar to the idea of CovertFS\,\cite{baliga2007web} where a maximum of \SI[per-mode = symbol]{4}{\kilo\byte} per image would be used. However, this would significantly decrease the storage capacity of \gls{FFS}. Part of future work is to explore if more data could be stored in the images, or if multiple \glspl{OWS} could be used to overcome the decreased storage capacity. 

% TODO: To get bandwidth of each filesystem: 
%	Record all requests
%	When filtering, look at all DNS requests to google.com/flickr.com
%	Only look at packets to/from these IPs
%	Need a lot of memory/save the file multiple to record all packets
%	Also takes a lot of time to filter all packets, millions if not billions
%
%