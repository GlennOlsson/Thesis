
\section{Future work}
\label{sec:futureWork}
% TODO: Sharable files?

As mentioned previously, FFS does not implement all features that the POSIX standard defines. Future development for FFS could be to implement more of these functions, such as links and file permissions. This could make the filesystem resemble a regular filesystem further. Another improvement could be to move from userspace using FUSE, to kernel space. This could speed up filesystem operations. Another feature that could be interesting to evaluate is the possibility to share files with other users, similar to Google Drive.

Even though the files are encrypted so that the data is confidential, further research could include hiding the user's online activity through the use of for instance Tor. Currently, the integrity of the user is not considered but for the filesystem to be further plausibly deniable, this should be addressed as the user could otherwise be identified by its IP address and other online fingerprints that could be provided by the online web services.

To improve the dependability of the filesystem, support for more online web services could be implemented. For instance, Github provides free user accounts with many gigabytes of data. Even free-tier distributed filesystems, such as Google Drive, could be utilized. If multiple user accounts are used in coordination over multiple services, the filesystem could achieve even more storage.

% TODO: Add about multi-user support. You mount FFS with your credentials, but the FFS stack might have multiple users, you can only see your own files obviously

% TODO: Add about if FFS becomes big, twitter will probably ban the accounts. Instead hide in better steganographic ways, such as small amounts of data in real images. More reliable towards detection for new accounts

% \todo[inline, backgroundcolor=kth-lightblue]{Vad du har kvar ogjort?\\
% Vad är nästa självklara saker som ska göras?\\
% Vad tips kan du ge till nästa person som kommer att följa upp på ditt arbete?
% \todo[inline]{Describe valid future work that you or someone else could or should do.\\
% Consider: What you have left undone? What are the next obvious things to be done? What hints can you give to the next person who is going to follow up on your work?
% }

% }

% Due to the breadth of the problem, only some of the initial goals have been
% met. In these section we will focus on some of the remaining issues that
% should be addressed in future work. ...

% \subsection{What has been left undone?}
% \label{what-has-been-left-undone}

% The prototype does not address the third requirment, i.e., a yearly
% unavailability of less than 3 minutes, this remains an open problem. ...

% \subsubsection{Cost analysis}

% The current prototype works, but the performance from a cost perspective makes
% this an impractical solution. Future work must reduce the cost of this
% solution, to do so a cost analysis needs to first be done. ...

% \subsubsection{Security}

% A future research effort is needed to address the security holes that results
% from using a self-signed certificate. Page filling text mass. Page filling
% text mass. ...


% \subsection{Next obvious things to be done}

% In particular, the author of this thesis wishes to point out xxxxxx remains as
% a problem to be solved. Solving this problem is the next thing that should be
% done. ...