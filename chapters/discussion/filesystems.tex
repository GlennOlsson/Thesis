\section{Filesystems}
\label{sec:dis_fs}
Figure~\ref{fig:bench_ffs_read}, Figure~\ref{fig:bench_ffs_re_read}, and Figure~\ref{fig:bench_ffs_rnd_read} show that \gls{FFS} performs poorly for Read operations with a small buffer size. Beginning at \SI{4}{\kilo\byte} buffer size, the performance in general increases with the first few buffer sizes. This indicates that the overhead of the \gls{FFS} read operation is high as the performance gets better when it reads fewer buffers. Overhead of the read operation includes, among other things, the time to fetch the image from Flickr if it is not in the cache, and decrypting the image which is required even if the image is cached. Further, it is expected that \mbox{Re-Read} performs better than Read when the file size is small enough to fit in the cache. This is also a conclusion that can be drawn from the result. However, looking at Figure~\ref{fig:bench_ffs_read}, and Figure~\ref{fig:bench_ffs_re_read}, we can also see that \mbox{Re-Read} overall performs better for file sizes bigger than the \SI{5}{\mega\byte} cache file size limit as well. Especially for the file sizes \SI{32\,768}{\kilo\byte} to \SI{262\,144}{\kilo\byte}, the performance of the Read test is in general very low. As Figure~\ref{fig:res_box_read} and Figure~\ref{fig:res_box_reread} shows, the average performance of \mbox{Re-Read} is more than double than the average performance of Read for \gls{FFS}. It is expected that the cache will increase the performance of the filesystem. The reason that \mbox{Re-Read} performs better than the Read test for files bigger than the cache size limit could be due to a cache in Flickr. It is possible that Flickr prepares often-requested images in a cache that serves the image faster than less-requested images. This could influence the time required to get the image from Flickr. It is also possible that IOZone does not close the file before it is read again, meaning that the file can be kept in memory as \gls{FFS} caches open files that have been read. IOZone does not specify when the file is closed.  Random Read also performs better than the Read test for big file sizes. This could also be because the file has not been closed since the file was read in to memory, enabling \gls{FFS} to provide the requested data faster. However, if the file would not be closed after a write test, the write test of \gls{FFS} should have the same performance as the write test of \gls{FFFS} as neither of the two filesystems actually save the data in their storage medium until the file has been closed. As the write tests of \gls{FFS} and \gls{FFFS} are not very similar, it is improbable that the file is not closed after the write tests. Furthermore, as the write tests are performed before the read tests, there is no possibility that the file was kept open after a read test either.

While the \mbox{Re-Read} and Random Read tests increases in performance for the first buffer sizes, the performance also decreases eventually. Looking at the data presented in Figure~\ref{fig:bench_ffs_re_read} and Figure~\ref{fig:bench_ffs_rnd_read}, the buffer sizes \SI[per-mode = symbol]{4\,096}{\kilo\byte} to \SI[per-mode = symbol]{16\,384}{\kilo\byte} have, in general, lower performance than the buffer sizes \SI[per-mode = symbol]{256}{\kilo\byte} to \SI[per-mode = symbol]{2\,048}{\kilo\byte}, for the same file size. This indicates that the optimal buffer size for \gls{FFS} read operations on previously read files is not the biggest possible buffer size, but rather around \SI[per-mode = symbol]{512}{\kilo\byte}, depending on the file size. The biggest file size has its best performance for a buffer size of \SI[per-mode = symbol]{512}{\kilo\byte}. Looking at the \SI[per-mode = symbol]{131\,072}{\kilo\byte} file size, it peaks for both the \mbox{Re-Read} test and the Random read test at \texttt{buffer size = 128\,kB}, and its performance for that buffer size for both tests are higher than any other performance of any file size or buffer size in both tests, for the filesystem. This is interesting because the \SI[per-mode = symbol]{131\,072}{\kilo\byte} file size does not always outperform the other file sizes. Looking at the \mbox{Re-Read} test, the \SI[per-mode = symbol]{262\,144}{\kilo\byte} has the best performance for seven of the 13 tests while the \SI[per-mode = symbol]{131\,072}{\kilo\byte} file size has the best performance for five buffer sizes, namely the first one, the last three and for \texttt{buffer size = 128\,kB}. The \SI[per-mode = symbol]{16\,384}{\kilo\byte} file size has the best performance of the test once, for \texttt{buffer size = 1\,024\,kB}. Considering how fast the operations actually are, it is possible to understand why the values can fluctuate. For instance, the \mbox{Re-Read} test for \gls{FFS} using \texttt{file size = 16\,384\,kB, buffer size = 1\,024\,kB} has a performance of \SI[per-mode = symbol]{7\,599\,353}{\kilo\byte\per\second}. Transferring \SI[per-mode = symbol]{16\,284}{\kilo\byte} at \SI[per-mode = symbol]{7\,599\,353}{\kilo\byte\per\second} takes \SI[per-mode = symbol]{2.14}{\milli\second}. If it would take 10\% more time, the performance would instead be under \SI[per-mode = symbol]{7\,000\,000}{\kilo\byte\per\second}, meaning that the \SI[per-mode = symbol]{262\,144}{\kilo\byte} file size would have better performance for the same buffer size. However, if it instead would take 10\% less time to perform the \texttt{file size = 16\,384\,kB, buffer size = 1\,024\,kB} \mbox{Re-Read} test, it would reach a performance of \SI[per-mode = symbol]{8\,443\,725.6}{\kilo\byte\per\second} which would be the highest performance of the test on \gls{FFS} of all file sizes and buffer sizes. Small time differences can significantly affect the performance of the tests. The time of the filesystem operations can be fluctuated by process scheduling and the internet connection, among other things.

The performance of the write operations of \gls{FFS} are highly influenced by the file size. As shown in Figure~\ref{fig:bench_ffs_write}, Figure~\ref{fig:bench_ffs_re_write}, and Figure~\ref{fig:bench_ffs_rnd_write}, bigger file sizes implicates better performance for the write operations, generally. The best performing file size was most often the largest file size, \SI[per-mode = symbol]{262\,144}{\kilo\byte}. Furthermore, the biggest file size of the tests perform better for the Write test than for the \mbox{Re-Write} and Random Write test. This is interesting as the Write test includes the overhead of creating the files before writing to them, which \mbox{Re-Write} and Random Write does not.

One interesting comparison is between the benchmark results of \gls{FFS} and \gls{GCSF}. Both filesystems are \mbox{cloud-based} \gls{FUSE} filesystems dependent on an internet connection to their respective storage servers. Looking at the box plots in Section~\ref{sec:res_bench}, we can see that \gls{GCSF} outperforms \gls{FFS} in both average performance and median performance for all tests. However, \gls{GCSF} does not have any data for the biggest file size while \gls{FFS} has data for it. Looking at Figure~\ref{fig:bench_ffs_re_read} and Figure~\ref{fig:bench_gcsf_re_read}, we can see that \gls{FFS} performs better than \gls{GCSF} for many of the bigger file sizes for the \mbox{Re-Read} test. It is possible that \gls{FFS} would perform better than \gls{GCSF} for the \SI{262\,144}{\kilo\byte} file size test if \gls{GCSF} could run the test. However, even if that would be the case, it is also possible that \gls{GCSF} would still perform better overall. One reason that \gls{GCSF} generally outperforms \gls{FFS} could be because the \gls{FFS} cache stores the encrypted version of the image, meaning that before the data is read, the image must first be decrypted and decoded. As Google Drive provides the raw data of the file stored, \gls{GCSF} can store the raw data in its cache meaning that the data in the read operation can be returned faster. If Google Drive caches the raw file data as well, it does not have to decrypt the data when serving it to \gls{GCSF}. \gls{GCSF} also outperforms \gls{FFS} in all the write tests. The reason could be that \gls{GCSF} does not have to encrypt the data nor encode it as images. This could save significant computation time. The average (reference point) bandwidth measured when the two filesystem benchmarks were run are similar, indicating that there was not a big difference in the internet connection to the measurement servers. However, as this does not measure the actual internet connection to the \gls{OWS}, the actual bandwidth of the filesystem could be different from this value. However, even assuming that the bandwidth of the internet connections of \gls{GCSF} and \gls{FFS} are equal, \gls{GCSF} can still benefit from fewer REST \gls{API} calls. As Google Drive is a filesystem, the inode table of the filesystem, or however the filesystem is organized, can be stored on the service without exposing it to the user. For instance, assuming that Google Drive uses an inode table like \gls{FFS}, the inode table would never have to be downloaded and uploaded by \gls{GCSF}. By simply uploading a file and specifying what its path and filename is, the inode table does not have to be modified by the user but can be handled by Google Drive in the background, potentially after the request has completed requiring less time for the file upload request. Meanwhile, \gls{FFS} has to upload the inode table after every file modification. Additionally, the old version of the file and inode table must be removed. This requires \gls{FFS} to perform at least four requests for all modifications:
\needspace{5\baselineskip}
\begin{itemize}
	\item Upload a new image with the new file content,
	\item Upload a new image with the new inode table content,
	\item Remove the old file, and,
	\item Remove the old inode table
\end{itemize}
Although, removing the old images is performed on another thread. Meanwhile, uploading a modified file to Google Drive requires one \gls{API} request using the file's ID\,\cite{FilesUpdateDrive2022} which will perform the same functionality as the four requests required for \gls{FFS}. The ID of the file could be stored locally in memory by \gls{GCSF} to be able to serve file ID's quickly, but this data structure does not have to be uploaded to Google Drive. Furthermore, when downloading a file, parts of the file can be downloaded rather than the full file\,\cite{googleDownloadFilesDrive2022}. This can reduce the time as the full file does not have to be downloaded every time a file is read. Even if we could download parts of a file from Flickr for \gls{FFS}, it would not make sense as we need the full file content to decrypt it. Futhermore, with Google Drive's 800 million daily users\,\cite{lardinoisGoogleUpdatesDrive2017} versus Flickr's 60 million monthly users\,\cite{campbellFlickrStatistics20222022}, Google Drive is a much bigger service. This could mean that it has better infrastructure which can process uploaded data faster than Flickr can.

Certain data points in the graphs presented in Section~\ref{sec:res_bench} are outliers from groups of data points. For instance, looking at the \mbox{Re-Read} test for \gls{GCSF} in Figure~\ref{fig:bench_gcsf_re_read} for \texttt{file size = 32\,768\,kB, buffer size = 256\,kB}, the test data point has much lower performance than the the other data points for similar buffer sizes in the same test with the same file size. There are many possible reasons behind this drop in performance. One reason could be a slow internet connection to Google Drive in the point of time when the specific data point was benchmarked, for instance due to a higher load of other user requests to the service. Due to the \gls{OWS} being an external service that other users use at the same time, it is always possible that the \gls{OWS} of the \mbox{cloud-based} filesystems experiences a high \mbox{user-demand} at any time. Another reason of the data outlier can also be because the operating system scheduler scheduled the \gls{GCSF} process unfavorable at that time. This is an especially possible reason to the outlier data points for the \mbox{non-cloud-based} filesystems \gls{FFFS} and \gls{APFS}. For instance, Figure~\ref{fig:bench_apfs_rnd_read} shows two outliers for \texttt{file size = 131\,072\,kB}, namely \texttt{buffer size = 4\,096\,kB} and \texttt{buffer size = 8\,192\,kB}. They could also have lower performance due to disk scheduling and cache management. The cached files could be removed from the cache if other processes are reading other files from the disk at the same time, invalidating the cache of the benchmark file.

Other outlier data points have much higher performance than the other data points in a test. For instance, looking at the Write test for \gls{FFFS} in Figure~\ref{fig:bench_fffs_write}, there are data points for \texttt{file size = 8\,192\,kB}, \texttt{file size = 16\,384\,kB}, and \texttt{file size = 32\,768\,kB} that have much higher performance than the other data points. While most data points are approximately between \SI[per-mode = symbol]{6\,500}{\kilo\byte\per\second} to \SI[per-mode = symbol]{8\,300}{\kilo\byte\per\second}, two of these file sizes have one outlier, and one has two outliers, of approximately \SI[per-mode = symbol]{100\,000}{\kilo\byte\per\second}. Outliers can also be seen in Figure~\ref{fig:bench_fffs_re_write} and Figure~\ref{fig:bench_fffs_rnd_write} for the \mbox{Re-Write} and Random Write tests on \gls{FFFS}. \gls{FFFS} is not a cloud-based filesystem and is therefore not affected by a fluctuating internet connection. Rather, this is possibly the result of favorable process scheduling and disk operation scheduling. As both file sizes exceed the cache limit of \gls{FFFS}, the cache of the filesystem should not affect the value. However, it is a possibility that IOZone did not close the file or that the \texttt{close} operation was not performed correctly for the preceding test, which would keep the open file in memory regardless of size. This would result in faster file operation as the file would not have to be read from the disk. Furthermore, if the \texttt{close} operation was not called for a test, the filesystem would not write the data to the disk resulting in shorter execution time, leading to a higher performance. Therefore, if there was a missed \texttt{close} operation, the following test or preceding test should also have higher performance than the other data points. However, looking at, for instance, the preceding Read test for \texttt{file size = 32\,768\,kB}, buffer size = 2\,048\,kB in Figure~\ref{fig:bench_fffs_read} and the subsequent \mbox{Re-Read} test for \texttt{file size = 32\,768\,kB, buffer size = 2\,048\,kB} in Figure~\ref{fig:bench_fffs_re_read}, those data points are not outliers which indicates that it the outlier in the Write test probably is not due to a missed \texttt{close} call.

Comparing \gls{FFFS} benchmarking results against the \gls{APFS} benchmarking results, we can compare the theoretical best performance of \gls{FFS} against a \mbox{general-purpose} \mbox{widely-used} filesystem. Furthermore, we can compare \gls{FFFS} against the underlying filesystem in which it is storing its data. In Figure~\ref{fig:bench_fffs_read} and Figure~\ref{fig:bench_apfs_read} we can see that the read operation perform similarly for \gls{FFFS} and \gls{APFS}, where \gls{APFS} is in general faster than \gls{FFFS}. However, for certain data points, such as \texttt{file size = 131\,072\,kB, buffer size = 256\,kB}, \gls{FFFS} has higher performance than \gls{APFS} with \SI[per-mode = symbol]{7\,525\,973}{\kilo\byte\per\second} for \gls{FFFS} and \SI[per-mode = symbol]{5\,927\,107}{\kilo\byte\per\second} for \gls{APFS}. As the file size is bigger than \SI{5}{\mega\byte} it was not stored in the cache of \gls{FFFS} and the filesystem therefore had to read it from \gls{APFS}. The reason why \gls{FFFS} outperformed \gls{APFS} at this data point is therefore most likely due to process scheduling or because the buffer size used by the test is less efficient for \gls{APFS} than the one called on \gls{APFS} by \gls{FFFS}. The buffer size used by \gls{FFFS} to read the file from \gls{APFS} is not certain to be the same as \gls{FFFS} was called with. The buffer size used by \gls{FFFS} on \gls{APFS} is set by the implementation of \texttt{basic\_filebuf}\,\cite{cppreference.comStdBasicFilebuf2020}.

The cache of a filesystems can generally influence the performance of the read operations. However, there is no significant drop in performance for \gls{FFFS} when reading a file that fits in the \gls{FFFS} cache, and one that does not. In fact, the performance of the Read test for \texttt{file size = 8\,192} is in general better than for \texttt{file size = 4\,096} and \texttt{file size = 2\,048} on \gls{FFFS}, even though the \SI{8\,192}{\kilo\byte} file cannot fit in the \gls{FFFS} cache while the other file sizes can, as long as the encrypted data and the PNG attributes do not make the file bigger than \SI[per-mode = symbol]{5}{\mega\byte}. The reason why these files can be provided fast is most probably due to the kernel caching of the files. However, the performance does drop significantly for \texttt{file size = 262\,144\,kB} compared to the previous file sizes, indicating that these files were not be cached by the kernel. The reason to this could be that \gls{FFFS} has to save the data as two images rather than one as the image size limit of \gls{FFFS} is the same as the limit for \gls{FFS}. All files that are read from \gls{FFFS} that are not in any cache are read from \gls{APFS}. However, when \gls{APFS} provides data, the data could also come from its kernel cache, or any other cache implemented by that filesystem. If the data is not in the any cache for \gls{APFS}, at least one \gls{APFS} read operation is performed. While the \gls{APFS} read operation called might not be called with the same buffer size as the read operation called by IOZone on \gls{FFFS}, the performance of the \gls{FFFS} read operation cannot exceed the \gls{APFS} read operation. However, the similarity of the performance between \gls{FFFS} and \gls{APFS} indicates that \gls{FFS} implements small read operation overhead, and that the read operation performance of \gls{FFS} depends to a great extent on the internet bandwidth and latency to the \gls{OWS}, as well as the \gls{OWS}'s data processing performance.

The only implementation difference between \gls{FFS} and \gls{FFFS} is that \gls{FFS} stores the produced images on Flickr while \gls{FFFS} stores the produced images on the local filesystem. Therefore, the time difference of an \gls{FFS} operation compared to an \gls{FFFS} operation should only depend on the internet connection to Flickr and how fast Flickr can process the requests. For instance, looking at the Write tests for the two filesystems, \gls{FFFS} outperforms \gls{FFS} significantly. Looking at one file size, for instance, \texttt{file size = 8\,192} \gls{FFFS} has an average performance of approximately \SI[per-mode = symbol]{15\,800}{\kilo\byte\per\second} and \gls{FFS} has an average performance of approximately \SI[per-mode = symbol]{1\,050}{\kilo\byte\per\second} for the Write test. This means that the test took on average \SI{518}{\milli\second} for \gls{FFFS} and \SI{7\,802}{\milli\second} for \gls{FFS}. The same test for the same file size for \gls{APFS} had an average performance of \SI[per-mode = symbol]{612\,000}{\kilo\byte\per\second}, meaning it took on average \SI[per-mode = symbol]{133}{\milli\second} for \gls{APFS} to save the \SI[per-mode = symbol]{8\,192}{\kilo\byte} file. Subtracting this value from the test time of \gls{FFFS}, we get the average overhead of the Write test for both \gls{FFS} and \gls{FFFS} as they have the same overhead. The time the test takes for \gls{FFFS} is the same as the time the test takes for the Write overhead plus the time \gls{APFS} takes to save the file. The average overhead time of the Write test for \gls{FFS} and \gls{FFFS} is therefore \SI[per-mode = symbol]{385}{\milli\second}, meaning that the requests and the request's overhead by \gls{FFS} took on average $7\,802 - 385 =$ \SI{7417}{\milli\second} which is about 95\% of the computation time for this test. Assuming the upload bandwidth to Flickr is the measured reference bandwidth of \SI[per-mode = symbol]{92.95}{\mega\byte\per\second}, uploading \SI[per-mode = symbol]{8\,192}{\kilo\byte} would take \SI[per-mode = symbol]{705}{\milli\second}. This means that the remaining \SI[per-mode = symbol]{6\,712}{\milli\second} were used for request overhead, such as preparing for the request, waiting for Flickr to process the data, and receiving the response from Flickr, including the post ID. This indicates that with faster data processing by Flickr, \gls{FFS} could potentially be faster. Furthermore, it indicates that the bandwidth of the internet connection to the \gls{OWS} is not the most important factor of the performance of \gls{FFS}. Even if uploading the file over the internet to the \gls{OWS} would be instant, it would reduce the file operation time by less than 10\%. To improve the filesystem operation performance, using a \gls{OWS} which can process the data faster is of more importance. The calculations above assume that the bandwidth to Flickr was approximately the same as the bandwidth to the measurement servers. It is possible that the bandwidth to Flickr was much lower, which would mean that the bandwidth has more impact of the filesystem operation time.

Looking at the same file size for the Read test, \gls{FFFS} has an average performance of approximately \SI[per-mode = symbol]{4\,804\,000}{\kilo\byte\per\second} and \gls{FFS} has an average performance of approximately \SI[per-mode = symbol]{4\,234\,000}{\kilo\byte\per\second}. These performances are similar to each other, and notably, the performance of \gls{FFS} is much higher than the reference download bandwidth, and higher than what a normal internet connection usually supports, they are often limited to $\text{\SI[per-mode = symbol]{1}{\giga\bit\per\second}} = \text{\SI[per-mode = symbol]{125\,000}{\kilo\byte\per\second}}$ by the ISP, depending on the subscription. This indicates that this data was rather provided by the kernel cache than by the filesystem itself. This is probably the case for most of the \mbox{high-performing} read operations on \gls{FFS} for files bigger than the \gls{FFS} cache limit. The \mbox{high-performing} reads on \gls{FFFS} could be provided by either the kernel cache for \gls{FFFS} or from \gls{APFS}.

While the values of the read operation for \gls{FFFS} and \gls{APFS} are comparable to each other, this is not the case for all tests. For instance the write operation of \gls{FFFS} is much slower than the write operation of \gls{APFS}, as can be seen in Table~\ref{tbl:data_fejk-ffs_write} and Table~\ref{tbl:data_local_write}. The write operation performance average of \gls{FFFS} is about 1.5\% of the average performance of the write operation on \gls{APFS}. The reason for this could be the fact that \gls{FFFS} has to encrypt the data stored, including creating all the cryptographic variables such as the salt and the \gls{IV}. While \gls{APFS} is also an encrypted filesystem, it is possible that the filesystem prepares the cryptographic variables before they are needed. For instance, the next cryptographic key and its salt could be derived while the filesystem is idle as it does not depend on the content of the stored data. It is also possible that \gls{APFS} is using multiple threads to write the data which could speed up the operation. For instance, it is possible that \gls{FFFS} requires multiple \texttt{write} calls to the underlying \gls{APFS} filesystem. If each \texttt{write} call to \gls{APFS} uses a different thread, the multiple \texttt{write} calls could possibly be completed faster rather than if they were completed sequentially. 

\gls{FFFS} and \gls{GCSF} are comparable in some tests, which is interesting as \gls{GCSF} is dependent on an internet connection while \gls{FFFS} is not. The median performance of the \mbox{Re-Read} test on \gls{GCSF} is slightly worse than the medium performance of the \mbox{Re-Read} test on \gls{FFFS}. Meanwhile, the median Read performance of \gls{GCSF} is significantly less than the median Read performance of \gls{FFFS}. This indicates that \gls{GCSF} implements a fast cache. However, the data of \gls{GCSF} does not include the \SI{262\,144}{\kilo\byte} file size. The Write, \mbox{Re-Write}, and Random Write tests on \gls{FFFS} outperform the same tests on \gls{GCSF}. This is reasonable as the data written to \gls{GCSF} must be uploaded to Google Drive, while the data written to \gls{FFFS} is stored on the local disk. Uploading \SI{16}{\mega\byte} of data with the average (reference point) upload speed of \SI[per-mode = symbol]{91.83}{\mega\bit\per\second} would take about \SI{1.4}{\second}. Meanwhile, we can see in Figure~\ref{fig:bench_apfs_write} that \gls{APFS} can write \SI{16}{\mega\byte} of data as fast as \SI[per-mode = symbol]{6\,921\,222}{\kilo\byte\per\second} = \SI[per-mode = symbol]{6\,921.2}{\mega\byte\per\second}, meaning it would take about \SI{2}{\milli\second} to write the data. Meanwhile, the maximum Write performance of \gls{FFFS} is \SI[per-mode = symbol]{101\,677}{\kilo\byte\per\second} according to Figure~\ref{fig:bench_fffs_write}, meaning that \gls{FFFS} can write \SI{16}{\mega\byte} in about \SI{157}{\milli\second}. With this data, we can see that \gls{FFFS} can write the \SI{16}{\mega\byte} of data about 7\,800\% slower than \gls{APFS} can, and that \gls{GCSF} can write the \SI{16}{\mega\byte} of data about 800\% slower than \gls{FFFS}. 

It is easy to see, and it is not unexpected, that \gls{APFS} outperforms \gls{FFS} in performance. As a professional local filesystem, \gls{APFS} will always have better performance than FFS. Further, like \gls{FFFS}, the performance of \gls{FFS} depends on the performance of \gls{APFS} as the file which is uploaded to Flickr first needs to be saved on disk. This dependency could be removed, for instance by providing the temporary file to the FlickCURL library via a \gls{FUSE} filesystem. Further, the median performance of the \mbox{Re-Read} test on \gls{FFS} is about 72\% of the performance of the \mbox{Re-Read} test on \gls{APFS}. With higher bandwidth and with another \gls{OWS}, it is possible that \gls{FFS} could increase its performance. In contrast, the median performance of the \mbox{Re-Read} test on \gls{FFS} is about 76\% of the median performance of the same test on \gls{GCSF}.

Generally, the figures of the benchmark data presented in Section~\ref{sec:res_bench} have visibly similar patterns for the Write tests and similar patterns the Read tests, per filesystem. For instance, the patterns of the \gls{FFS} Read, \mbox{Re-Read}, and Random Read follow a similar pattern with a similar curve of the data points, while the Write, \mbox{Re-Write}, and Random Write follow another distinct pattern. The Read, \mbox{Re-Read}, and Random Read figures of \gls{GCSF} follow another distinct pattern, as well as the three Write test data follow a fourth distinct pattern. The three write test patterns of \gls{FFS} are dissimilar to the patterns of the \gls{FFFS} write tests even though both filesystems are implemented very similarly, other than the storage medium. However, some patters are similar even though they are from different filesystem, for instance, the Read and Random Read tests of \gls{FFS} and \gls{FFFS}. For instance, in the Read test certain filesystem data points are found on the lower spectrum of the plot, while the other file sizes follow a somewhat similar curve for both filesystems. However, the pattern of the \mbox{Re-Read} test data of \gls{FFS} and \gls{FFFS} differ significantly. It might be possible to use these distinct patterns to create a fingerprint of a filesystem. This could be used, for instance, to identify a filesystem based on its performance when the filesystem is unknown. This could be useful when the filesystem is masked by an overlaying filesystem such as Cryptfs. However, by benchmarking the overlying filesystem, the pattern of the underlying filesystem might be lost. Analyzing if it is possible to identify filesystems based on fingerprints and identifying underlying filesystems using this technique is part of future work.

% TODO: 
% - Analyze the time of "overhead", i.e. time of FFFS - time of APFS
%		Does not really make sense though as the buffer size to APFS is unknown, meaning it's hard
%		to know which buffer size to compare to. The highest performing? Maybe the lowest?