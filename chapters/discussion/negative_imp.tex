\section{Impact}
This section presents the impact \gls{FFS} could have. Section~\ref{subsec:imp_soc} presents societal impacts that \gls{FFS} could introduce. Section~\ref{subsec:imp_env} presents the environment impacts that \gls{FFS}.

\subsection{Societal impacts}
\label{subsec:imp_soc}
Secure and hidden data is not only for the better good. As the data stored on \gls{FFS} cannot be decrypted by bad guys no good guys, illegal data could be stored on the system without anyone knowing about it. It is known that end-to-end encryption does not only have a positive impact on society, for instance, terrorist organizations are also known to be using it to spread their messages across the internet\,\cite{ruddEncryptionCounterterrorismGetting2017}. \gls{FFS} could potentially provide secure storage for illegal groups such as terrorist organizations and child pornography rings. It is not possible to limit who uses \gls{FFS}, by other means than not publishing the source code of the filesystem. However, this does not prevent criminal organizations to use other end-to-end encrypted filesystems or develop their own. Some terrorist organizations consist of well-educated engineers who could develop similar technologies for their organization\,\cite{berrebyEngineeringTerror2010}.

\subsection{Environmental impact}
\label{subsec:imp_env}
FFS uses Flickr's data centers to store its data. Globally, data centers have a huge environmental impact. It has been estimated that they use over 2\% of the world's electricity\,\cite{mcleanDataCentersGenerate2020} and emit roughly the same amount of carbon dioxide as the global airline industry emits burning aircraft fuel\,\cite{pearceEnergyHogsCan}. These data centers are always on and are always consuming energy. When storing the data on a local filesystem instead, the device can be powered off while the filesystem is not in use, such as by detaching an external storage drive.

Further, as mentioned previously, encrypting and encoding the stored data as images requires more storage than the actual data stored. This means that more storage is required to store all the data in \gls{FFS}, as opposed to storing the same data on a local filesystem. It also means that the network request will carry more data than necessary, requiring more energy. This fact is also true when comparing \gls{FFS} to a cloud-based filesystem, such as Google Drive. While both Google Drive and \gls{FFS} store their data in data centers, the data that Google Drive stores can be less than the same file stored in \gls{FFS} due to the overhead of \gls{FFS}. Further, Google Drive can be optimized as a filesystem as it is the intention of Google Drive. Meanwhile, \gls{FFS} is exploiting Flickr's image storage and is not optimized as filesystem storage. For instance, Google Drive can optimize the cache of the filesystem to better reflect a filesystem cache, reducing power consumption. However, as Flickr handles large amounts of data, Flickr has probably implemented energy-efficient solutions for retrieving images. 

% Stored in data centers rather than locally
%	Always on, consuming energy
%	Requires networking to reach data, uses multiple routers between home and servers
% 	Less power efficient to make a network request than accessing local hard drive

% Stored on Flickr rather than on Google drive
% 	Takes more space, more bandwidth and storage needed
%	Good engineers who can optimize storage efficiency
%	Flickr has good engineers too, and needs to store its data good as well
