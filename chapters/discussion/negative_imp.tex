\section{Impact}
\label{sec:imp}
This section presents the impact \gls{FFS} could have. Section~\ref{subsec:imp_soc} presents societal impacts that \gls{FFS} could introduce. Section~\ref{subsec:imp_env} presents the environmental impacts of \gls{FFS}.

\subsection{Societal impacts}
\label{subsec:imp_soc}
Secure and hidden data is not only for the better good. As the data stored on \gls{FFS} cannot be decrypted by bad guys or good guys, illegal data could be stored on the system without anyone knowing about it. It is known that \mbox{end-to-end} encryption does not have only a positive impact on society, for instance, terrorist organizations are known to be using \mbox{end-to-end} encryption to spread their messages across the internet\,\cite{ruddEncryptionCounterterrorismGetting2017}. \gls{FFS} could potentially provide secure storage for illegal groups such as terrorist organizations and child pornography rings. It is not possible to limit who uses \gls{FFS}, by means other than not publishing the source code of the filesystem. However, this does not prevent criminal organizations from using other \mbox{end-to-end} encrypted filesystems or developing their own. Some terrorist organizations consist of \mbox{well-educated} engineers who could develop similar technologies for their organization\,\cite{berrebyEngineeringTerror2010}.

An ethical consideration of \gls{FFS} is that it is breaking Flickr's terms of service by exploiting its free storage. Flickr and many other \gls{OWS}s provide users with a lot of free storage. By exploiting this storage, the costs for the \glspl{OWS} could increase requiring them to charge users for the storage or decrease or eliminate the free storage quota. This would hurt honest users of the service who are following the guidelines. Content creators, such as photographers on Flickr, could have to pay money to use the previously free service which could deter the usage of the platform. If the \gls{OWS} implements detection techniques to combat the \gls{FFS} images, these could falsely mark legit images for removal which again could affect honest users of the service. Fewer users of the \gls{OWS} could eventually require the company to reduce their staff due to loss of revenue or potentially cease business altogether. 

\gls{FFS} provides free storage for all users. This benefits people who might not have money to spend on commercial \mbox{cloud-based} storage or physical hard disks. However, as \gls{FFS} requires the user to run macOS which natively only runs on Apple's computers which are often considered expensive, this might not benefit the poor. Furthermore, the cost of a hard disk or commercial \mbox{cloud-based} storage is often not expensive. The result is that arguing for \gls{FFS} based on economic arguments is a weak argument.

\subsection{Environmental impact}
\label{subsec:imp_env}
\gls{FFS} uses Flickr's data centers to store its data. Globally, data centers have a large environmental impact. It has been estimated that they use over 2\% of the world's electricity\,\cite{mcleanDataCentersGenerate2020} and emit roughly the same amount of carbon dioxide as the global airline industry emits burning aircraft fuel\,\cite{pearceEnergyHogsCan}. However, the emissions of the data centers depend on their location. For instance, some data centers in Sweden are powered with 100\% carbon-free energy\,\cite{cappellaSwedenSustainableData2022,unfcccEcoDataCenterSwedenUNFCCC}. The data centers Flickr are using and where they are located have not been found.

As mentioned previously, encrypting and encoding the data as images requires more storage than the actual data stored. This means that more storage is required to store all the data in \gls{FFS}, as opposed to storing the same data in a local filesystem. It also means that the network request will carry more data than necessary, requiring more energy. This fact is also true when comparing \gls{FFS} to a \mbox{cloud-based} filesystem, such as Google Drive. While both Google Drive and \gls{FFS} store their data in data centers, the data that Google Drive stores is less than the same file stored in \gls{FFS} due to only storing the encrypted file data rather than an encoded image. While the extra PNG data is expected to be less than 10\% of the total image size, 10\% of a large file can still be significant. Furthermore, as Google Drive is a filesystem, operations and data structures could be optimized for a filesystem whereas \gls{FFS} is a layered filesystem on top of Flickr. As mentioned in~\Cref{sec:dis_fs}, Google Drive could potentially implement a more efficient filesystem with its REST \gls{API} than Flickr does with its \gls{API}. \gls{FFS} was developed by one developer during a limited time while Google Drive is maintained by multiple teams and was released over ten years ago. Google Drive also has requirements for efficient power- and storage usage as it is a massive company where small improvements can save a lot of money. \gls{FFS} on the other hand has no such requirements and was developed only for this thesis.

Other than storing data in the cloud, data can is often stored locally on physical devices such as memory sticks or hard disks. \gls{FFS} provides \mbox{on-demand} storage when the user needs it, and can be forgotten when not in use. It does not require the user to purchase hardware whenever they need more storage, possibly just temporarily. Such hardware could produce litter if the user disposes of it after use. While the storage devices can be cheap, they can promote \mbox{single-use} of the devices which in turn could increase littering. However, with the low-performance of \gls{FFS} compared to a local filesystem, \gls{FFS} can not be used as a substitute for a physical storage device. While a filesystem on a portable storage device was not included in the analysis of the thesis, \gls{FFS} will probably perform worse than most portable storage devices. Many modern storage devices are based on solid-state disks which in general are very fast. Even hard-drive based storage devices are probably faster than a \mbox{cloud-based} filesystem in many cases.

% TODO: UN Environmental goals, connect?