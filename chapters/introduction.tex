
\chapter{Introduction}
\label{ch:introduction}
% 	todo[inline, backgroundcolor=kth-lightblue]{svensk: Introduktion}


% \todo[inline, backgroundcolor=kth-lightblue]{Ofta kommer problemet och problemägaren
%   från industrin där man önskar en specifik lösning på ett specifikt
%   problem. Detta är ofta ”för smalt” definierat och ger ofta en ”för smal”
%   lösning för att resultatet skall vara intressant ur ett mer allmänt
%   ingenjörsperspektiv och med ”nya” erfarenheter som resultat. Fundera
%   tillsammans med projektets intressenter (student, problemägare och akademi)
%   hur man skulle kunna använda det aktuella problemet/förslaget för att
%   undersöka någon ingenjörsaspekt och vars resultat kan ge ny eller
%   kompletterande erfarenhet till ingenjörssamfundet och vetenskapen.\\
  
%   Examensarbetet handlar då om att ta fram denna nya ”erfarenhet” och på köpet
%   löser man en del eller hela delen av det ursprungliga problemet.\\

%   Erfarenheten kommer ur en frågeställning som man i examensarbetet försöker
%   besvara med tidigare och andras erfarenhet, egna eller modifierade metoder som
%   ger ett resultat vilket kan användas för att diskutera ett svar på
%   undersökningsfrågan.\\

%   Detta stycke skall alltså, förutom det ursprungliga ”smala” problemet,
%   innehålla  vad som skall undersökas för att skapa ny ingenjörserfarenhet
%   och/eller vetenskap.
% }

% \todo[inline, backgroundcolor=kth-lightgreen]{The first paragraph after a heading is not indented, all of the
%   subsequent paragraphs have their first line indented.}
  
% This chapter describes the specific problem that this thesis addresses, the context of the problem, the
% goals of this thesis project, and outlines the structure of the thesis.\\

% \todo[inline]{Give a general introduction to the area. (Remember to use appropriate references in this and all other sections.)}

% One can use either biblatex or bibtex - set as the option for the document at the top of this file
  % \ifbiblatex
% \todo[inline, backgroundcolor=kth-lightgreen]{We use the \emph{biblatex} package to handle our references.  We
% use the command \texttt{parencite} to get a reference in parenthesis, like
% this \textbackslash parencite\{heisenberg2015\} resulting in \parencite{heisenberg2015}.  It is also possible to include the author as part of the sentence using \texttt{textcite}, like talking about the work of \textbackslash textcite\{einstein2016\} resulting in \textcite{einstein2016}.\\
% This also means that you have to change the include files to include biblatex and change the way that the reference.bib file is included.}
  % \else
% \todo[inline, backgroundcolor=kth-lightgreen]{We use the \emph{bibtex} package to handle our references.  We therefore
% use the command \textbackslash cite\{farshin\_make\_2019\}. For example, Farshin, \etal described how to improve LLC
% cache performance in \cite{farshin_make_2019} in the context of links running
% at \SI{200}{Gbps}.}
  % \fi

% \todo[inline, backgroundcolor=kth-lightgreen]{Use the glossaries package to help yourself and your readers.
% Add the acronyms and abbreviations to templates/kth/lib/acronyms.tex. Some examples are shown below:}
% In this thesis we will examine the use of \glspl{LAN}. In this thesis we will
% assume that \glspl{LAN} include \glspl{WLAN}, such as \gls{WiFi}.

Year after year, people increase their total data storage used for obvious reasons. Cameras increase their resolution leading to images and videos take even more space. With storage being cheap and easily usable, files do not needed to be deleted thus the data accumulates. % FIXME: CITATION NEEDED?
This means that users will require more and more storage throughout their lifetime, and even potentially beyond their lifetime if their decendants want to keep these files. System storage in our hardware devices often increases with new product cycles. Today you can keep hundreds of gigabytes in your pocket at a resonable cost. % FIXME: COMPARE iPHONE FROM LIKE 10 YEARS AGO AND TODAY - STORAGE AVAILIABLE, INCREASED. LOWEST TIER VS HIGHEST TIER
Along with increasing device storage cloud storage capacity is increasing. For instance Apple's iCloud service allows users to store up to \SI{2}{\tera\byte} of data in the cloud for a few U.S. Dollars per month. % FIXME: CITE??
Even though the cost per month is not a lot, after many months this cost accumulates and you as a user get more and more dependent on this storage, especially as you do not want to spend time looking through all your data and remove some files to save space. % FIXME: FIND SOMETHING THAT USERS NEVER DOWNGRADES THEIR STORAGE - MUST EXIST
With increased pricing or increased space, the total cost will be even higher.

Social media platforms such as Twitter, Flickr, and Facebook have many millions of daily users that post texts and images (for example, of their cats or funny videos). According to Henna Kermani at Twitter, they processed ~\SI{200}{\giga\byte} of image data every second in 2016\cite{MobileScaleLondona}. A single user posting a few images per day does not significantly change the amount of data processed or saved at all for these tech giants. % - a few gigabytes here or there will probably go unnoticed. % FIXME: PROBABLY?? IS THAT GOOD ENOUGH FOR A THESIS??
The difference between the photos posted on Twitter compared to the ones stored on cloud services such as iCloud is that the images on Twitter are stored for free for the user, for what seems to be indefinitely. While there is no obligation for these services to save it forever, and they do reserve the right to remove any content at any time, %TODO: Link TOS.
there is also no specified maximum lifespan of these posts. While iCloud and similar services often have a free-tier of storage, Twitter does not have a specified upper limit of how many images or tweets one can make, but such constraints can be inflicted on specific users whenever as according to their terms of service.

\section{Project Overview}

This project intends to create a filesystem called \textit{Fejk File System} (FFS) which takes advantage of online web services, such as Twitter, for the actual storage. The idea is to save the user's files by posting or sending an encrypted version of the file as posts or private messages on these web services. The intention is not to create a revolutionary fast and usable filesystem but instead to explore how well it is possible to utilize the storage that Twitter and similar services provides for free as a filesystem. The performance and limits of this filesystem will however be analyzed and compared to existing alternatives, such as Google Drive, to compare the benefits of this free storage compared to a professional system that might cost money.

% \section{Background}
\label{sec:background}
% \todo[inline, backgroundcolor=kth-lightblue]{svensk: Bakgrund}

% \todo[inline]{Present the background for the area. Set the context for your project – so that your reader can understand both your project and this thesis. (Give detailed background information in Chapter 2 - together with related work.)
% Sometimes it is useful to insert a system diagram here so that the reader
% knows what are the different elements and their relationship to each
% other. This also introduces the names/terms/… that you are going to use
% throughout your thesis (be consistent). This figure will also help you later
% delimit what you are going to do and what others have done or will do.}

Year after year, people increase their total data storage used for obvious reasons. Cameras get better leading to images and videos taking more space, and with storage being cheap and easily usable, files are not needed to be deleted meaning that the data usage accumulates (\textbf{CITATION NEEDED?}). This means that users will require more and more storage throughout their lifetime, and even potentially beyond their lifetime if dependants want to keep these files. System storage in our hardware devices often increases with new product cycles. Today you can keep hundreds of gigabytes in your pocket without spending a big fortune(\textbf{COMPARE iPHONE FROM LIKE 10 YEARS AGO AND TODAY - STORAGE AVAILIABLE, INCREASED. LOWEST TIER VS HIGHEST TIER}). Along with increasing device storage is cloud storage increasing. For instance Apple's service iCloud allows users to store up to 2TB of data in the cloud for a few bucks per month (\textbf{CITE??}). Even though the cost per month is not a lot, after many months this cost accumulates and you as a user get more and more dependent on this storage, especially as you don't want to spend time looking through all your data and maybe remove some to save space (\textbf{FIND SOMETHING THAT USERS NEVER DOWNGRADES THEIR STORAGE - MUST EXIST}). With increased pricing or necessary space upgrade, the cost will be even higher.

Social media platforms such as Twitter, Flickr, and Facebook have many millions of daily users that post anything from texts to images for their cats or funny videos. According to Henna Kermani at Twitter, they processed about 200GB of image data every second in 2016\cite{MobileScaleLondona}. A single user posting a few images per day does not significantly change the amount of data processed or saved at all for these tech giants - a few gigabytes here or there will probably go unnoticed (\textbf{PROBABLY?? IS THAT GOOD ENOUGH FOR A THESIS?}). (\textbf{MENTION HERE ABOUT POTENTIAL ANOMALY DETECTION?? OR LATER?+}). The difference between the photos posted on Twitter compared to the ones stored on cloud services such as iCloud is that the images on Twitter are stored for free for the users, indefinitely. While iCloud and similar services often have a free-tier of storage, Twitter does not have an upper limit of how many images or tweets one can make (\textbf{RIGHT?? I COULD NOT FIND ANYTHING WITH A QUICK GOOGLE SEARCH. LOOK AT TOS?})


\section{Problem}
\label{sec:problem}
% \todo[inline, backgroundcolor=kth-lightblue]{svensk: Problemdefinition eller Frågeställning\\
% Lyft fram det ursprungliga problemet om det finns något och definiera därefter
% den ingenjörsmässiga erfarenheten eller/och vetenskapen som kan komma ur
% projektet. }

Is it possible to create a steganographic, distributed filesystem that takes advantage of online services to store the data through the use of free user accounts? What are the drawbacks of such a filesystem compared to commercial available solutions in regards to speed, throughput and reliability? Are there more advantages than it being a free storage system?

% Longer problem statement\\
% If possible, end this section with a question as a problem statement.

% % Research Question
% \subsection{Original problem and definition}
% \label{sec:researchQuestion}
% % \todo[inline, backgroundcolor=kth-lightblue]{Ursprungligt problem och definition}
% Some text

% \subsection{Scientific and engineering issues}% \todo[inline, backgroundcolor=kth-lightblue]{Vetenskaplig och ingenjörsmässig frågeställning}
% some text


\section{Purpose and motivation}
% \todo[inline, backgroundcolor=kth-lightblue]{Syfte}
% \todo[inline, backgroundcolor=kth-lightblue]{Skilj på syfte och mål! Syfte är att förändra något till det bättre. I examensarbetet finns ofta två aspekter på detta. Dels vill problemägaren (företaget) få sitt problem löst till det bättre men akademin och ingenjörssamfundet vill också få nya erfarenheter och vetskap. Beskriv ett syfte som tillfredställer båda dessa aspekter.\\
% Det finns även ett syfte till som kan vara värt att beakta och det är att du som student skall ta examen och att du måste bevisa, i ditt examensarbete, att du uppfyller examensmålen. Dessa mål sammanfaller med kursmålen för examensarbetskursen. 
% }
% \todo[inline]{State the purpose  of your thesis and the purpose of your degree project.\\
% Describe who benefits and how they benefit if you achieve your goals. Include anticipated ethical, sustainability, social issues, etc. related to your project. (Return to these in your reflections in Section~\ref{sec:reflections}.)}

The purpose of this research is to explore the possibility to create a filesystem that stores data on online services and to compare the performance of such a filesystem to an actual distributed filesystem service. The interesting aspect of this is that services, such as social media, provide users with essentially an infinite amount of storage for free. Anyone can create any number of accounts on Twitter and Facebook without cost, and with enough accounts, one could potentially store all their data using such a filesystem. The thesis explores the use of such a filesystem despite potentially being slower and less dependable than filesystems that utilize other types of storage media, such as filesystems that costs a few dollars per month. Further, is it ethically defendable to create and use such a system?

% Being able to advantaging just a few services to create a useable filesystem that can store certain amount of data means that further work can be done to extend the filesystem with even more services and thus achieving even more storage.


\section{Goals}
% \todo[inline, backgroundcolor=kth-lightblue]{Mål}
% \todo[inline, backgroundcolor=kth-lightblue]{Skilj på syfte och mål. Syftet är att åstakomma en förändring i något. Målen är vad som konkret skall göras för att om möjligt uppnå den önskade förändringen (syfte). }

% \todo[inline]{State the goal/goals of this degree project.}

The project aims to create a secure, deniable filesystem that stores its data on online web services by taking advantage of the storage provided to its users. This can be split into the following subgoals:
\begin{enumerate}
\item to create a mountable filesystem where files and directories can be stored, read, and deleted,
\item for the filesystem to store all the data on online web services rather than on the local disk,
\item for the system to be secure in the sense that even with access to the uploaded files and the software, the \mbox{plain-text} data is unreadable without the correct decryption key, 
\item to provide the user of the filesystem with plausible deniability of its data in the sense that it is not possible to associate the user with \gls{FFS} if the filesystem is not mounted,
\item to analyze the write and read speed, storage capacity, and reliability of the filesystem and compare it to commercial \mbox{cloud-based} filesystems and local filesystems, and,
\item to analyze and discuss environmental and ethical aspects of the filesystem.
\end{enumerate}

% \todo[inline]{In addition to presenting the goal(s), you might also state what the deliverables and results of the project are.}



\section{Research Methodology}%\todo[inline, backgroundcolor=kth-lightblue]{Undersökningsmetod}
% \todo[inline, backgroundcolor=kth-lightblue]{Här anger du vilken vilken övergripande undersökningsstrategi eller metod du skall använda för att försöka besvara den akademiska frågeställning och samtidigt lösa det e v ursprungliga problemet. Ofta kan man använda ”lösandet av ursprungsproblemet” som en fallstudie kring en akademisk frågeställning. Du undersöker någon intressant fråga i ”skarpt” läge och samlar resultat och erfarenhet ur detta.\\
% Tänk på att företaget ibland måste stå tillbaka i sin önskan och förväntan på projektets resultat till förmån för ny eller kompletterande ingenjörserfarenhet och vetenskap (ditt examensarbete). Det är du som student som bestämmer och löser fördelningen mellan dessa två intressen men se till att alla är informerade. }
% \todo[inline]{Introduce your choice of methodology/methodologies and method/methods – and the reason why you chose them. Contrast them with and explain why you did not choose other methodologies or methods. (The details of the actual methodology and method you have chosen will be given in Chapter~\ref{ch:methods}. Note that in Chapter~\ref{ch:methods}, the focus could be research strategies, data collection, data analysis, and quality assurance.)\\
% In this section you should present your philosophical assumption(s), research method(s), and research approach(es).}

The filesystem created through this thesis will be developed on a Macbook laptop running macOS Monterey, version 12.3.1. It will be written in C++20 and use the Filesystem in Userspace (FUSE) MacOS library\,\cite{HomeMacFUSE} which enables the writing of a filesystem in userspace rather than in kernel space. FUSE is available on other platforms too, such as Linux, but the filesystem will be developed on a Macbook laptop thus macFUSE is chosen. C++ is chosen because the FUSE API is available in C, and C++ version 20 is well established and used. Further details about the development environment will be found in Section~\ref{sec:dev_env}.

The resulting filesystem will be evaluated against other filesystems, both commercial distributed systems, such as Google drive, and an instance of Apple File System (APFS)\,\cite{appleinc.AppleFileSystem} on the Macbook laptop referenced above. Quantitative data will be gathered from the different filesystems through the use of experiments with the filesystem benchmarking software IOzone\,\cite{IozoneFilesystemBenchmark}. IOzone was chosen because it is, compared to tools such as Fio and Bonnie++, simpler to use while still powerful\,\cite{agarwalComparingIOBenchmarks2018}. We will look at attributes such as the differences in read and write speeds between different filesystems, as well as the speed of random read and random write. However, according to \citeauthor{tarasovBenchmarkingFileSystem2011}, benchmarking filesystems using benchmarking tools is difficult to perform in a standardized way\,\cite{tarasovBenchmarkingFileSystem2011} which will be taken into consideration during the evaluation and when concluding the thesis. Further discussion about this will be found in Section~\ref{sec:iozone}.

\section{Delimitations} % \todo[inline, backgroundcolor=kth-lightblue]{Avgränsningar}
% \todo[inline]{Describe the boundary/limits of your thesis project and what you are explicitly not going to do. This will help you bound your efforts – as you have clearly defined what is out of the scope of this thesis project. Explain the delimitations. These are all the things that could affect the study if they were examined and included in the degree project.}

Due to limitations in time and as the system is only a prototype for a working filesystem and not a production filesystem, some features found in other filesystems are not going to be implemented in FFS. The focus will be to implement a subset of the POSIX standard functions, containing only crucial functions for a simple filesystem, specifically, the FUSE functions \textit{open}, \textit{read}, \textit{write}, \textit{mkdir}, \textit{rmdir}, \textit{readdir}, and \textit{rename}. However, file access control is not a necessity and will therefore not be implemented, thus functions such as \textit{chown} and \textit{chmod} are not going to be implemented. The reason is that the goal is to present and evaluate the possibility of creating a secure steganographic filesystem with a storage medium based on online web services and thus FFS will only aim to implement a minimal filesystem. 

There is also an argument that could be made that FFS should support multiple users so that anyone can mount FFS but only browse their files. However, as this project is only a proof-of-concept of the filesystem, this will not be implemented. Instead, FFS will be built for single-user support where only a password will unlock everything FFS is storing. This means that anyone who mounts FFS with the password will access everything that other users might have stored.

% FIXME: IS the last sentence true? Can't I / Must I not support multiple users for my deniability argument?

\section{Structure of the thesis} % \todo[inline, backgroundcolor=kth-lightblue]{ Rapportens disposition}
Chapter~\ref{ch:background} presents theoretical background information of filesystems and the basis of \gls{FFS} while Chapter~\ref{ch:related_work} mentions and analyzes related work. Chapter~\ref{ch:methods} describes the implementation and the design choices made for the system, along with the analysis methodology. Chapter~\ref{ch:results} presents the results of the analysis and Chapter~\ref{ch:discussion} discusses the findings and other aspects of the work. Lastly, Chapter~\ref{ch:conclusionsAndFutureWork} will state the conclusion of the thesis and discuss potential future work.