
\chapter{Introduction}
\label{ch:introduction}
% 	todo[inline, backgroundcolor=kth-lightblue]{svensk: Introduktion}


% \todo[inline, backgroundcolor=kth-lightblue]{Ofta kommer problemet och problemägaren
%   från industrin där man önskar en specifik lösning på ett specifikt
%   problem. Detta är ofta ”för smalt” definierat och ger ofta en ”för smal”
%   lösning för att resultatet skall vara intressant ur ett mer allmänt
%   ingenjörsperspektiv och med ”nya” erfarenheter som resultat. Fundera
%   tillsammans med projektets intressenter (student, problemägare och akademi)
%   hur man skulle kunna använda det aktuella problemet/förslaget för att
%   undersöka någon ingenjörsaspekt och vars resultat kan ge ny eller
%   kompletterande erfarenhet till ingenjörssamfundet och vetenskapen.\\
  
%   Examensarbetet handlar då om att ta fram denna nya ”erfarenhet” och på köpet
%   löser man en del eller hela delen av det ursprungliga problemet.\\

%   Erfarenheten kommer ur en frågeställning som man i examensarbetet försöker
%   besvara med tidigare och andras erfarenhet, egna eller modifierade metoder som
%   ger ett resultat vilket kan användas för att diskutera ett svar på
%   undersökningsfrågan.\\

%   Detta stycke skall alltså, förutom det ursprungliga ”smala” problemet,
%   innehålla  vad som skall undersökas för att skapa ny ingenjörserfarenhet
%   och/eller vetenskap.
% }

% \todo[inline, backgroundcolor=kth-lightgreen]{The first paragraph after a heading is not indented, all of the
%   subsequent paragraphs have their first line indented.}
  
% This chapter describes the specific problem that this thesis addresses, the context of the problem, the
% goals of this thesis project, and outlines the structure of the thesis.\\

% \todo[inline]{Give a general introduction to the area. (Remember to use appropriate references in this and all other sections.)}

% One can use either biblatex or bibtex - set as the option for the document at the top of this file
  % \ifbiblatex
% \todo[inline, backgroundcolor=kth-lightgreen]{We use the \emph{biblatex} package to handle our references.  We
% use the command \texttt{parencite} to get a reference in parenthesis, like
% this \textbackslash parencite\{heisenberg2015\} resulting in \parencite{heisenberg2015}.  It is also possible to include the author as part of the sentence using \texttt{textcite}, like talking about the work of \textbackslash textcite\{einstein2016\} resulting in \textcite{einstein2016}.\\
% This also means that you have to change the include files to include biblatex and change the way that the reference.bib file is included.}
  % \else
% \todo[inline, backgroundcolor=kth-lightgreen]{We use the \emph{bibtex} package to handle our references.  We therefore
% use the command \textbackslash cite\{farshin\_make\_2019\}. For example, Farshin, \etal described how to improve LLC

% at \SI{200}{Gbps}.}
  % \fi

% \todo[inline, backgroundcolor=kth-lightgreen]{Use the glossaries package to help yourself and your readers.
% Add the acronyms and abbreviations to templates/kth/lib/acronyms.tex. Some examples are shown below:}
% In this thesis we will examine the use of \glspl{LAN}. In this thesis we will
% assume that \glspl{LAN} include \glspl{WLAN}, such as \gls{WiFi}.

To keep files and data secure we often use encrypted filesystems. However, while these filesystems hide the content of the data, they often do not conceal the existence of data. For instance, using snapshots of the filesystems from different moments in time, it could be possible to notice a difference in the data stored and therefore that data exists and where it is located. Snapshots could even reveal user passwords\,\cite{hanMultiuserSteganographicFile2010}.

Deniable filesystems are intended to make the data deniable, meaning that the user is supposed to be able to plausibly deny the existence of data. This is often accomplished through the use of digital steganography. There are many reasons why this is important. For instance, in 2011, a Syrian man recorded videos of attacks on civilians carried out by Syrian security forces, which he wanted to share with the world\,\cite{westheadHowSyrianRefugee2012}. By cutting his arm, he was able to hide a memory card inside the wound and smuggled it out of the country. However, if he would have used methods such as an encrypted deniable filesystem, the border control may not have been able to discover even the existence of data, even if they would have found the memory card. By only encrypting the data, the border control would have been able see that he was trying to hide data and make him reveal the decryption key, either by legal measures or by force, which is why he smuggled it out.

There exists multiple deniable filesystems that are designed to combat this problem on physical devices, such as memory cards. However, even just carrying a memory card might subject you to suspicion of hiding data, no matter how the filesystem is designed. Another solution to hiding the data is therefore to hide it somewhere else, for instance online through the use of cloud-based filesystem service, such as Google Drive. Someone searching your body and devices, at for instance an airport or border control, might not realize that you are using a cloud-based filesystem service to hide your data. Although, more thorough investigations of a person might reveal user accounts used on the service, leading to legal processes where the service is forced to disclose your data. Even if you encrypt the data you upload to such a service, you can still be forced to reveal the decryption keys. What we want to achieve is a combination of a deniable filesystem and a cloud-based filesystem, where the data is stored using digital cryptographic and steganographic methods but without any company or person other than the user controlling the actual data. To accomplish this, we can store the data on online social media platforms.

Social media platforms such as Twitter and Flickr have many millions of daily users that post texts and images (for example, of their cats or funny videos). According to Henna Kermani at Twitter, they processed ~\SI{200}{\giga\byte} of image data every second in 2016\,\cite{MobileScaleLondona}. The photos posted on Twitter, as opposed to the ones stored on cloud services such as Google Drive, are stored for free on the service for the user, for what seems to be an indefinite period. There is also no specified limit of how many images or tweets one can make. Although, as stated in their terms of service, such limits can be imposed on specific users whenever Twitter wishes and tweets can be removed at any point in time\,\cite{twitterTwitterTermsService2021}.

This project intends to create a cryptographic and deniable cloud-based filesystem called the \textit{Fejk FileSystem} (FFS) which takes advantage of free online web services, such as Twitter and Flickr, for the actual storage. The idea is to save the user's files by posting an encrypted version of the file as images and text posts these web services. The intention is not to create a revolutionary fast and usable filesystem but instead to explore how well it is possible to utilize the storage that Twitter and similar services provide their users for free, as a cryptographic and deniable cloud-based filesystem. Additionally, the performance and limits of this filesystem will be analyzed and compared to alternative filesystems, such as Google Drive, to compare the advantages and disadvantages of the developed filesystem compared to professional filesystems. The security of the filesystem will also be discussed, as well as an analysis of the steganographic capability of the developed filesystem.

% \section{Project Overview}

This project intends to create a filesystem called \textit{Fejk FileSystem} (FFS) which takes advantage of online web services, such as Twitter, for the actual storage. The idea is to save the user's files by posting or sending an encrypted version of the file as posts or private messages on these web services. The intention is not to create a revolutionary fast and usable filesystem but instead to explore how well it is possible to utilize the storage that Twitter and similar services provides for free as a filesystem. However, the performance and limits of this filesystem will be analyzed and compared to existing alternatives, such as Google Drive, to compare the benefits of this free storage compared to a professional system that might cost money. The security of the filesystem will also be discussed, as well as an analysis of the steganographic capability of the developed filesystem. 
% Moved the content to this page instead


\section{Problem}
\label{sec:problem}
% \todo[inline, backgroundcolor=kth-lightblue]{svensk: Problemdefinition eller Frågeställning\\
% Lyft fram det ursprungliga problemet om det finns något och definiera därefter
% den ingenjörsmässiga erfarenheten eller/och vetenskapen som kan komma ur
% projektet. }

Is it possible to create a secure and deniable cloud-based filesystem that store the data on various online web services through the use of the free user accounts offered by these online web services? What are the drawbacks of such a filesystem compared to similar filesystem solutions with regards to write and read speed, storage capacity, and reliability? Are there advantages to such a system in regards to security and deniability? 

% Longer problem statement\\
% If possible, end this section with a question as a problem statement.

% % Research Question
% \subsection{Original problem and definition}
% \label{sec:researchQuestion}
% % \todo[inline, backgroundcolor=kth-lightblue]{Ursprungligt problem och definition}
% Some text

% \subsection{Scientific and engineering issues}% \todo[inline, backgroundcolor=kth-lightblue]{Vetenskaplig och ingenjörsmässig frågeställning}
% some text


\section{Purpose and motivation}
% \todo[inline, backgroundcolor=kth-lightblue]{Syfte}
% \todo[inline, backgroundcolor=kth-lightblue]{Skilj på syfte och mål! Syfte är att förändra något till det bättre. I examensarbetet finns ofta två aspekter på detta. Dels vill problemägaren (företaget) få sitt problem löst till det bättre men akademin och ingenjörssamfundet vill också få nya erfarenheter och vetskap. Beskriv ett syfte som tillfredställer båda dessa aspekter.\\
% Det finns även ett syfte till som kan vara värt att beakta och det är att du som student skall ta examen och att du måste bevisa, i ditt examensarbete, att du uppfyller examensmålen. Dessa mål sammanfaller med kursmålen för examensarbetskursen. 
% }
% \todo[inline]{State the purpose  of your thesis and the purpose of your degree project.\\
% Describe who benefits and how they benefit if you achieve your goals. Include anticipated ethical, sustainability, social issues, etc. related to your project. (Return to these in your reflections in Section~\ref{sec:reflections}.)}

The purpose of this research is to explore the possibility to create a secure, steganographic cloud-based filesystem that stores data on \gls{OWS}s and to compare the performance, benefits, and disadvantages of such a filesystem to existing steganographic filesystems and distributed filesystem services. A distributed filesystem service, such as Google Drive, provide data storage for users which can be both free and cost money. Even though Google Drive encrypts the user's data, they control the encryption and decryption keys, and the method of encryption\,\cite{johnsonGoogleDriveSecure2021}. This means that they can give out the user's files and data if faced with legal actions such as subpoenas. It also opens up the possibility of hackers gaining access to the files without the user having any way to control them.

% -- Even using a cloud-based filesystem service as a layer in a stacked filesystem with a cryptographic and deniable layer, the amount of data stored in the system would still be visible to the service provider even if some or even all of that data is noise. It is also an obvious way to store data on a service that provides conventional storage. 

The idea behind \gls{FFS} is to have a decentralized cloud-based filesystem where only the user has access to the unencrypted data. By encrypting and decrypting the files locally before uploading and after downloading them to these services (end-to-end encryption), it is possible to ensure that the user is the only one who has access to the encryption and decryption keys and therefore the unencrypted data. Even if the web service would look at the data uploaded by the user, it is unreadable without the decryption key. An interesting aspect of this is that online web services, such as social media, provide users with essentially an unbounded amount of storage for free. Anyone can create any number of accounts on Twitter and Facebook without cost, and with enough accounts, one could potentially store all their data using such a filesystem. We aim to exploit the storage web services give their users for free. As the file data is stored in the open but only accessible by the user, and as \gls{FFS} can be unmounted to hide its existence, it is steganographic. 

There are several steganographic filesystems available but these lack certain aspects that \gls{FFS} aims to solve. Some filesystems are based on the local disk of the device in use, such as the physical storage device on a computer or phone, or an external storage device connected to a computer or phone. While these filesystems have advantages compared to cloud-based solutions, such as latency, they lack accessibility as you need to have the device to access the content on it. It also means that when you want to share or transport the data, you must physically move the device which can mean problems as it could for instance be taken from you or be destroyed. Cloud-based solutions counter this by being available from any location that has internet access to the services used. However, existing cloud-based solutions introduce other disadvantages. One example is CovertFS\,\cite{baliga2007web} where data is stored in images posted on web services. The images are actual images representing something, meaning that there is a limit on how much steganographic data can be stored. CovertFS limit this to \SI{4}{\kilo\byte} which means that such a filesystem with a lot of data will require many images which could lead to suspicion from the owners of the web services. \gls{FFS} stores as much data as possible in the images, meaning that less images are needed to store a file bigger than \SI{4}{\kilo\byte}. It also means that the images produced by \gls{FFS} do not look like a normal image, but instead has seemingly randomly colored pixels. More examples of similar filesystems will be presented in Chapter~\ref{ch:related_work}. 


\section{Goals}
% \todo[inline, backgroundcolor=kth-lightblue]{Mål}
% \todo[inline, backgroundcolor=kth-lightblue]{Skilj på syfte och mål. Syftet är att åstakomma en förändring i något. Målen är vad som konkret skall göras för att om möjligt uppnå den önskade förändringen (syfte). }

% \todo[inline]{State the goal/goals of this degree project.}

The project aims to create a secure, mountable filesystem that stores its data via online web services by taking advantage of the storage provided to its users. This can be split into the following subgoals;
\begin{enumerate}
\item to create a free mountable filesystem where files can be stored, read, and deleted, % \todo[inline, backgroundcolor=kth-lightblue]{för att tillfredsställa problemägaren – industrin?}
\item for the system to be secure in the sense that even with access to the uploaded files and the software, the data is not readable without the correct decryption key, and, % \todo[inline, backgroundcolor=kth-lightblue]{för att tillfredsställa ingenjörssamfundet och vetenskapen – akademin) }
\item to analyze the write and read speed, storage capacity, and reliability of the filesystem and compare it to commercial distributed filesystems.
% \item to be able to store a useful\footnote{} amount of data % \todo[inline, backgroundcolor=kth-lightblue]{eventuellt, för att uppfylla kursmålen – du som student}
\end{enumerate}

A side effect of such a filesystem that creates posts that are, while encrypted, publicly available is a steganographic filesystem in the sense that the data is hidden in plain sight. Therefore, an additional subgoal is to achieve and analyze the deniability of the filesystem. This could make the filesystem useful for persons who need to hide their data, such as spies, journalists, and political actors where freedom of speech is non-existent.

% \todo[inline]{In addition to presenting the goal(s), you might also state what the deliverables and results of the project are.}



\section{Research Methodology}%\todo[inline, backgroundcolor=kth-lightblue]{Undersökningsmetod}
% \todo[inline, backgroundcolor=kth-lightblue]{Här anger du vilken vilken övergripande undersökningsstrategi eller metod du skall använda för att försöka besvara den akademiska frågeställning och samtidigt lösa det e v ursprungliga problemet. Ofta kan man använda ”lösandet av ursprungsproblemet” som en fallstudie kring en akademisk frågeställning. Du undersöker någon intressant fråga i ”skarpt” läge och samlar resultat och erfarenhet ur detta.\\
% Tänk på att företaget ibland måste stå tillbaka i sin önskan och förväntan på projektets resultat till förmån för ny eller kompletterande ingenjörserfarenhet och vetenskap (ditt examensarbete). Det är du som student som bestämmer och löser fördelningen mellan dessa två intressen men se till att alla är informerade. }
% \todo[inline]{Introduce your choice of methodology/methodologies and method/methods – and the reason why you chose them. Contrast them with and explain why you did not choose other methodologies or methods. (The details of the actual methodology and method you have chosen will be given in Chapter~\ref{ch:methods}. Note that in Chapter~\ref{ch:methods}, the focus could be research strategies, data collection, data analysis, and quality assurance.)\\
% In this section you should present your philosophical assumption(s), research method(s), and research approach(es).}

% TODO: Fix comments here

The filesystem created through this thesis be written in C++11 and the FUSE MacOS library\cite{HomeMacFUSE} which enables us to write a filesystem in user space rather than kernel space. The produced filesystem will be evaluated against other filesystems, both commercial distributed systems, such as Google drive, but also an APFS filesystem on a Macbook laptop. Quantitative data will be gathered from the different filesystems through the use of experiments with the filesystem benchmarking software Iozone\cite{IozoneFilesystemBenchmark}. We will look at attributes such as the difference in speed of read and write, as well as the speed of random read and random write. 

\todo[inline]{Do I need to motivate the use of Iozone, as compared to Fio or FFSB? Should the attributes I will look at be motivated as well?}

% Quantitative data will be gathered from the filsystem developed as a result of this thesis, as well as different existing filesystems it will be compared against, such as Google Drive. This data will be gathered by experiments using filesystem benchmarking tools, such as fio\footnote{\url{https://fio.readthedocs.io/en/latest/fio_doc.html}}, where different variables of the benchmarking tools can be tested. 

% TODO: WHAT DO ADD?? FUUUCK

\section{Delimitations} % \todo[inline, backgroundcolor=kth-lightblue]{Avgränsningar}
% \todo[inline]{Describe the boundary/limits of your thesis project and what you are explicitly not going to do. This will help you bound your efforts – as you have clearly defined what is out of the scope of this thesis project. Explain the delimitations. These are all the things that could affect the study if they were examined and included in the degree project.}

Due to limitations in time and as the system is only a prototype for a working filesystem and not a production filesystem, some features found in other filesystems are not going to be implemented in FFS. This includes for instance file access control and symbolic links. The reason is that the goal is to present and evaluate the possibility of creating such a filesystem with a variety of different storage subsystems. The features are more limited to those that are useful in a regular filesystem. However FFS will only aim to implement a minimalistic filesystem. %TODO: This includes reading, deleting and writing files and directories and support for multiple levels of directory hierarchy etc etc

\section{Structure of the thesis} % \todo[inline, backgroundcolor=kth-lightblue]{ Rapportens disposition}
Chapter~\ref{ch:background} presents relevant background information about xxx.  Chapter~\ref{ch:methods} presents the methodology and method used to solve the problem. …