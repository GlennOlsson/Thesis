
\chapter{Introduction}
\label{ch:introduction}

To keep files and data secure we often use encrypted filesystems. However, while these filesystems hide the content of the data, they often do not conceal the existence of data. For instance, using snapshots of the filesystems from different moments in time, it could be possible to notice a difference in the data stored and therefore that data exists and where it is located. Snapshots could even reveal user passwords\,\cite{hanMultiuserSteganographicFile2010}.

Deniable filesystems are intended to make the data deniable, meaning that the user is supposed to be able to plausibly deny the existence of data. This is often accomplished through the use of digital steganography. There are many reasons why this is important. For instance, in 2011, a Syrian man recorded videos of attacks on civilians carried out by Syrian security forces, which he wanted to share with the world\,\cite{westheadHowSyrianRefugee2012}. By cutting his arm, he was able to hide a memory card inside the wound and smuggled it out of the country. However, if he would have used methods such as an encrypted deniable filesystem, the border control may not have been able to discover even the existence of data, even if they would have found the memory card. By only encrypting the data, the border control would have been able to see that he was trying to hide data and make him reveal the decryption key, either by legal measures or by force, which is why he smuggled it out.

There exist multiple deniable filesystems that are designed to combat this problem on physical devices, such as memory cards. However, even just carrying a memory card might subject you to suspicion of hiding data, no matter how the filesystem is designed. Another solution to hiding the data is therefore to hide it somewhere else, for instance online through the use of a \mbox{cloud-based} filesystem service, such as Google Drive. Someone searching your body and devices, at for instance an airport or border control, might not realize that you are using a \mbox{cloud-based} filesystem service to hide your data. Although, more thorough investigations of a person might reveal user accounts used on the service, leading to legal processes where the service is forced to disclose your data. Even if you encrypt the data you upload to such a service, you can still be forced to reveal the decryption keys. What we want to achieve is a combination of a deniable filesystem and a \mbox{cloud-based} filesystem, where the data is stored using digital cryptographic and deniable methods but without any company or person other than the user controlling the actual data. To accomplish this, we can store the data on online social media platforms.

Social media platforms such as Twitter and Flickr have many millions of daily users that post texts and images (for example, of their cats or funny videos). According to Henna Kermani at Twitter, they processed ~\SI{200}{\giga\byte} of image data every second in 2016\,\cite{MobileScaleLondona}. The photos posted on Twitter, as opposed to the ones stored on cloud services such as Google Drive, are stored for free on the service for the user, for what seems to be an indefinite period. There is also no specified limit on how many images or tweets one can make. Although, as stated in their terms of service, such limits can be imposed on specific users whenever Twitter wishes, and tweets can be removed at any point in time\,\cite{twitterTwitterTermsService2021}.

This project created a cryptographic and deniable \mbox{cloud-based} filesystem called \gls{FFS} which takes advantage of free online web services, such as Twitter and Flickr, for the actual storage. The idea was to save the user's files by posting an encrypted version of the file as images and text posts on these web services. The intention was not to create a revolutionary fast and usable filesystem but instead explore how feasible it is to utilize the storage that Twitter and similar services provide their users for free, as a cryptographic and deniable \mbox{cloud-based} filesystem. Additionally, the performance and limits of this filesystem are analyzed and compared to alternative filesystems, such as Google Drive, to compare the advantages and disadvantages of the developed filesystem compared to professional filesystems. The security of the proposed filesystem is discussed and an analysis of the deniability of the developed filesystem is presented.


\section{Problem}
\label{sec:problem}
% \todo[inline, backgroundcolor=kth-lightblue]{svensk: Problemdefinition eller Frågeställning\\
% Lyft fram det ursprungliga problemet om det finns något och definiera därefter
% den ingenjörsmässiga erfarenheten eller/och vetenskapen som kan komma ur
% projektet. }

Is it possible to create a steganographic, distributed filesystem that takes advantage of online services to store the data through the use of free user accounts? What are the drawbacks of such a filesystem compared to commercial available solutions in regards to speed, throughput and reliability? Are there more advantages than it being a free storage system?

% Longer problem statement\\
% If possible, end this section with a question as a problem statement.

% % Research Question
% \subsection{Original problem and definition}
% \label{sec:researchQuestion}
% % \todo[inline, backgroundcolor=kth-lightblue]{Ursprungligt problem och definition}
% Some text

% \subsection{Scientific and engineering issues}% \todo[inline, backgroundcolor=kth-lightblue]{Vetenskaplig och ingenjörsmässig frågeställning}
% some text


\section{Purpose and motivation}
% \todo[inline, backgroundcolor=kth-lightblue]{Syfte}
% \todo[inline, backgroundcolor=kth-lightblue]{Skilj på syfte och mål! Syfte är att förändra något till det bättre. I examensarbetet finns ofta två aspekter på detta. Dels vill problemägaren (företaget) få sitt problem löst till det bättre men akademin och ingenjörssamfundet vill också få nya erfarenheter och vetskap. Beskriv ett syfte som tillfredställer båda dessa aspekter.\\
% Det finns även ett syfte till som kan vara värt att beakta och det är att du som student skall ta examen och att du måste bevisa, i ditt examensarbete, att du uppfyller examensmålen. Dessa mål sammanfaller med kursmålen för examensarbetskursen. 
% }
% \todo[inline]{State the purpose  of your thesis and the purpose of your degree project.\\
% Describe who benefits and how they benefit if you achieve your goals. Include anticipated ethical, sustainability, social issues, etc. related to your project. (Return to these in your reflections in Section~\ref{sec:reflections}.)}

The purpose of this research is to explore the possibility to create a filesystem that stores data on online services and to compare the performance of such a filesystem to an actual distributed filesystem service. The interesting aspect of this is that services, such as social media, provide users with essentially an infinite amount of storage for free. Anyone can create any number of accounts on Twitter and Facebook without cost, and with enough accounts, one could potentially store all their data using such a filesystem. The thesis explores the use of such a filesystem despite potentially being slower and less dependable than filesystems that utilize other types of storage media, such as filesystems that costs a few dollars per month. Further, is it ethically defendable to create and use such a system?

% Being able to advantaging just a few services to create a useable filesystem that can store certain amount of data means that further work can be done to extend the filesystem with even more services and thus achieving even more storage.


\section{Goals}
% \todo[inline, backgroundcolor=kth-lightblue]{Mål}
% \todo[inline, backgroundcolor=kth-lightblue]{Skilj på syfte och mål. Syftet är att åstakomma en förändring i något. Målen är vad som konkret skall göras för att om möjligt uppnå den önskade förändringen (syfte). }

% \todo[inline]{State the goal/goals of this degree project.}

The project aims to create a secure, deniable filesystem that stores its data on online web services by taking advantage of the storage provided to its users. This can be split into the following subgoals:
\begin{enumerate}
\item to create a mountable filesystem where files and directories can be stored, read, and deleted,
\item for the filesystem to store all the data on online web services rather than on the local disk,
\item for the system to be secure in the sense that even with access to the uploaded files and the software, the \mbox{plain-text} data is unreadable without the correct decryption key, 
\item to provide the user of the filesystem with plausible deniability of its data in the sense that it is not possible to associate the user with \gls{FFS} if the filesystem is not mounted,
\item to analyze the write and read speed, storage capacity, and reliability of the filesystem and compare it to commercial \mbox{cloud-based} filesystems and local filesystems, and,
\item to analyze and discuss environmental and ethical aspects of the filesystem.
\end{enumerate}

% \todo[inline]{In addition to presenting the goal(s), you might also state what the deliverables and results of the project are.}



\section{Research Methodology}%\todo[inline, backgroundcolor=kth-lightblue]{Undersökningsmetod}
% \todo[inline, backgroundcolor=kth-lightblue]{Här anger du vilken vilken övergripande undersökningsstrategi eller metod du skall använda för att försöka besvara den akademiska frågeställning och samtidigt lösa det e v ursprungliga problemet. Ofta kan man använda ”lösandet av ursprungsproblemet” som en fallstudie kring en akademisk frågeställning. Du undersöker någon intressant fråga i ”skarpt” läge och samlar resultat och erfarenhet ur detta.\\
% Tänk på att företaget ibland måste stå tillbaka i sin önskan och förväntan på projektets resultat till förmån för ny eller kompletterande ingenjörserfarenhet och vetenskap (ditt examensarbete). Det är du som student som bestämmer och löser fördelningen mellan dessa två intressen men se till att alla är informerade. }
% \todo[inline]{Introduce your choice of methodology/methodologies and method/methods – and the reason why you chose them. Contrast them with and explain why you did not choose other methodologies or methods. (The details of the actual methodology and method you have chosen will be given in Chapter~\ref{ch:methods}. Note that in Chapter~\ref{ch:methods}, the focus could be research strategies, data collection, data analysis, and quality assurance.)\\
% In this section you should present your philosophical assumption(s), research method(s), and research approach(es).}

The filesystem created through this thesis will be developed on a Macbook laptop running macOS Monterey, version 12.3.1. It will be written in C++20 and use the Filesystem in Userspace (FUSE) MacOS library\,\cite{HomeMacFUSE} which enables the writing of a filesystem in userspace rather than in kernel space. FUSE is available on other platforms too, such as Linux, but the filesystem will be developed on a Macbook laptop thus macFUSE is chosen. C++ is chosen because the FUSE API is available in C, and C++ version 20 is well established and used. Further details about the development environment will be found in Section~\ref{sec:dev_env}.

The resulting filesystem will be evaluated against other filesystems, both commercial distributed systems, such as Google drive, and an instance of Apple File System (APFS)\,\cite{appleinc.AppleFileSystem} on the Macbook laptop referenced above. Quantitative data will be gathered from the different filesystems through the use of experiments with the filesystem benchmarking software IOzone\,\cite{IozoneFilesystemBenchmark}. IOzone was chosen because it is, compared to tools such as Fio and Bonnie++, simpler to use while still powerful\,\cite{agarwalComparingIOBenchmarks2018}. We will look at attributes such as the differences in read and write speeds between different filesystems, as well as the speed of random read and random write. However, according to \citeauthor{tarasovBenchmarkingFileSystem2011}, benchmarking filesystems using benchmarking tools is difficult to perform in a standardized way\,\cite{tarasovBenchmarkingFileSystem2011} which will be taken into consideration during the evaluation and when concluding the thesis. Further discussion about this will be found in Section~\ref{sec:iozone}.

\section{Delimitations} % \todo[inline, backgroundcolor=kth-lightblue]{Avgränsningar}
% \todo[inline]{Describe the boundary/limits of your thesis project and what you are explicitly not going to do. This will help you bound your efforts – as you have clearly defined what is out of the scope of this thesis project. Explain the delimitations. These are all the things that could affect the study if they were examined and included in the degree project.}

Due to limitations in time and as the system is only a prototype for a working filesystem and not a production filesystem, some features found in other filesystems are not going to be implemented in FFS. The focus will be to implement a subset of the POSIX standard functions, containing only crucial functions for a simple filesystem, specifically, the FUSE functions \textit{open}, \textit{read}, \textit{write}, \textit{mkdir}, \textit{rmdir}, \textit{readdir}, and \textit{rename}. However, file access control is not a necessity and will therefore not be implemented, thus functions such as \textit{chown} and \textit{chmod} are not going to be implemented. The reason is that the goal is to present and evaluate the possibility of creating a secure steganographic filesystem with a storage medium based on online web services and thus FFS will only aim to implement a minimal filesystem. 

There is also an argument that could be made that FFS should support multiple users so that anyone can mount FFS but only browse their files. However, as this project is only a proof-of-concept of the filesystem, this will not be implemented. Instead, FFS will be built for single-user support where only a password will unlock everything FFS is storing. This means that anyone who mounts FFS with the password will access everything that other users might have stored.

% FIXME: IS the last sentence true? Can't I / Must I not support multiple users for my deniability argument?

\section{Structure of the thesis} % \todo[inline, backgroundcolor=kth-lightblue]{ Rapportens disposition}
Chapter~\ref{ch:background} presents theoretical background information of filesystems and the basis of \gls{FFS} while Chapter~\ref{ch:related_work} mentions and analyzes related work. Chapter~\ref{ch:methods} describes the implementation and the design choices made for the system, along with the analysis methodology. Chapter~\ref{ch:results} presents the results of the analysis and Chapter~\ref{ch:discussion} discusses the findings and other aspects of the work. Lastly, Chapter~\ref{ch:conclusionsAndFutureWork} will state the conclusion of the thesis and discuss potential future work.