\section{Background}
\label{sec:background}
% \todo[inline, backgroundcolor=kth-lightblue]{svensk: Bakgrund}

% \todo[inline]{Present the background for the area. Set the context for your project – so that your reader can understand both your project and this thesis. (Give detailed background information in Chapter 2 - together with related work.)
% Sometimes it is useful to insert a system diagram here so that the reader
% knows what are the different elements and their relationship to each
% other. This also introduces the names/terms/… that you are going to use
% throughout your thesis (be consistent). This figure will also help you later
% delimit what you are going to do and what others have done or will do.}

Year after year, people increase their total data storage used for obvious reasons. Cameras get better leading to images and videos taking more space, and with storage being cheap and easily usable, files are not needed to be deleted meaning that the data usage accumulates (\textbf{CITATION NEEDED?}). This means that users will require more and more storage throughout their lifetime, and even potentially beyond their lifetime if dependants want to keep these files. System storage in our hardware devices often increases with new product cycles. Today you can keep hundreds of gigabytes in your pocket without spending a big fortune(\textbf{COMPARE iPHONE FROM LIKE 10 YEARS AGO AND TODAY - STORAGE AVAILIABLE, INCREASED. LOWEST TIER VS HIGHEST TIER}). Along with increasing device storage is cloud storage increasing. For instance Apple's service iCloud allows users to store up to 2TB of data in the cloud for a few bucks per month (\textbf{CITE??}). Even though the cost per month is not a lot, after many months this cost accumulates and you as a user get more and more dependent on this storage, especially as you don't want to spend time looking through all your data and maybe remove some to save space (\textbf{FIND SOMETHING THAT USERS NEVER DOWNGRADES THEIR STORAGE - MUST EXIST}). With increased pricing or necessary space upgrade, the cost will be even higher.

Social media platforms such as Twitter, Flickr, and Facebook have many millions of daily users that post anything from texts to images for their cats or funny videos. According to Henna Kermani at Twitter, they processed about 200GB of image data every second in 2016\cite{MobileScaleLondona}. A single user posting a few images per day does not significantly change the amount of data processed or saved at all for these tech giants - a few gigabytes here or there will probably go unnoticed (\textbf{PROBABLY?? IS THAT GOOD ENOUGH FOR A THESIS?}). (\textbf{MENTION HERE ABOUT POTENTIAL ANOMALY DETECTION?? OR LATER?+}). The difference between the photos posted on Twitter compared to the ones stored on cloud services such as iCloud is that the images on Twitter are stored for free for the users, indefinitely. While iCloud and similar services often have a free-tier of storage, Twitter does not have an upper limit of how many images or tweets one can make (\textbf{RIGHT?? I COULD NOT FIND ANYTHING WITH A QUICK GOOGLE SEARCH. LOOK AT TOS?})