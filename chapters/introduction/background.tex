\section{Background}
\label{sec:background}
% \todo[inline, backgroundcolor=kth-lightblue]{svensk: Bakgrund}

% \todo[inline]{Present the background for the area. Set the context for your project – so that your reader can understand both your project and this thesis. (Give detailed background information in Chapter 2 - together with related work.)
% Sometimes it is useful to insert a system diagram here so that the reader
% knows what are the different elements and their relationship to each
% other. This also introduces the names/terms/… that you are going to use
% throughout your thesis (be consistent). This figure will also help you later
% delimit what you are going to do and what others have done or will do.}

% NOTE: Moved as introduction of chapter rather than its own section

Year after year, people increase their total data storage used for obvious reasons. Cameras increase their resolution leading to images and videos taking even more space. With storage being cheap and easily usable, files do not needed to be deleted thus the data accumulates. % FIXME: CITATION NEEDED?
This means that users will require more and more storage throughout their lifetime, and even potentially beyond their lifetime if their decendants want to keep these files. System storage in our hardware devices often increases with new product cycles. Today you can keep hundreds of gigabytes in your pocket at a resonable cost. % FIXME: COMPARE iPHONE FROM LIKE 10 YEARS AGO AND TODAY - STORAGE AVAILIABLE, INCREASED. LOWEST TIER VS HIGHEST TIER
Along with increasing device storage, cloud storage capacity is increasing. For instance Apple's iCloud service allows users to store up to \SI{2}{\tera\byte} of data in the cloud for a few U.S. Dollars per month. % FIXME: CITE??
Even though the cost per month is not a lot, after many months this cost accumulates and you as a user get more and more dependent on this storage, especially as you do not want to spend time looking through all your data and remove some files to save space. % FIXME: FIND SOMETHING THAT USERS NEVER DOWNGRADES THEIR STORAGE - MUST EXIST
With increased pricing or increased space, the total cost will be even higher.

Social media platforms such as Twitter, Flickr, and Facebook have many millions of daily users that post texts and images (for example, of their cats or funny videos). According to Henna Kermani at Twitter, they processed ~200GB of image data every second in 2016\cite{MobileScaleLondona}. A single user posting a few images per day does not significantly change the amount of data processed or saved at all for these tech giants. % - a few gigabytes here or there will probably go unnoticed. % FIXME: PROBABLY?? IS THAT GOOD ENOUGH FOR A THESIS??
The difference between the photos posted on Twitter compared to the ones stored on cloud services such as iCloud is that the images on Twitter are stored for free for the user, for what seems to be indefinitely. While there is no obligation for these services to save it forever, and they do reserve the right to remove any content at any time,  %TODO: Link TOS.
there is also no specified maximum lifespan of these posts. While iCloud and similar services often have a free-tier of storage, Twitter does not have a specified upper limit of how many images or tweets one can make in total, but such constraints can be inflicted on specific users whenever as according to their terms of service. However, there is a limit to the amount of tweets and posts one can make per day. %TODO: Expand and link to API url sent by Chip