\section{Delimitations} % \todo[inline, backgroundcolor=kth-lightblue]{Avgränsningar}
% \todo[inline]{Describe the boundary/limits of your thesis project and what you are explicitly not going to do. This will help you bound your efforts – as you have clearly defined what is out of the scope of this thesis project. Explain the delimitations. These are all the things that could affect the study if they were examined and included in the degree project.}

Due to limitations in time and as the system is only a prototype for a working filesystem and not a production filesystem, some features found in other filesystems are not going to be implemented in FFS. Focus will be to implement a subset of the POSIX standard functions, containing only crucial functions for a simple filesystem. This includes \textit{open}, \textit{read}, \textit{write}, \textit{mkdir}, \textit{rmdir}, \textit{readdir}, and \textit{rename}. However, among other things, file access control is not a necessity and will therefore not be implemented and thus functions such as \textit{chown} and \textit{chmod} are not going to be implemented. The reason is that the goal is to present and evaluate the possibility of creating a filesystem with a variety of different storage subsystems and thus FFS will only aim to implement a minimalistic filesystem. 