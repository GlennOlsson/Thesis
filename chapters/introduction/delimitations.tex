\section{Limitations} % \todo[inline, backgroundcolor=kth-lightblue]{Avgränsningar}
% \todo[inline]{Describe the boundary/limits of your thesis project and what you are explicitly not going to do. This will help you bound your efforts – as you have clearly defined what is out of the scope of this thesis project. Explain the delimitations. These are all the things that could affect the study if they were examined and included in the degree project.}
\label{sec:delim}

Due to limitations in time and as the system is only a prototype for a working filesystem and not a production filesystem, some features found in other filesystems are not implemented in \gls{FFS}. The focus is to implement a subset of the POSIX standard functions, consisting of only those  functions crucial for a simple filesystem, specifically, the \gls{FUSE} functions: \textit{open}, \textit{read}, \textit{write}, \textit{mkdir}, \textit{rmdir}, \textit{readdir}, and \textit{rename}. However, file access control is not a necessity and will therefore not be implemented, thus functions such as \textit{chown} and \textit{chmod} are not going to be implemented. The reason is that the goal is to present and evaluate the \textit{possibility} of creating a cryptographically secure and deniable filesystem with a storage medium based on \glspl{OWS} and thus \gls{FFS} only aims to implement a minimal filesystem.