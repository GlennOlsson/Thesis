
\section{Goals}
% \todo[inline, backgroundcolor=kth-lightblue]{Mål}
% \todo[inline, backgroundcolor=kth-lightblue]{Skilj på syfte och mål. Syftet är att åstakomma en förändring i något. Målen är vad som konkret skall göras för att om möjligt uppnå den önskade förändringen (syfte). }

% \todo[inline]{State the goal/goals of this degree project.}

The project aims to create a secure, mountable filesystem which stores its data for online by taking advantage of the free storage given to users. This can be split into the following subgoals;
\begin{enumerate}
\item to create a mountable filesystem where files can be stored, read and deleted % \todo[inline, backgroundcolor=kth-lightblue]{för att tillfredsställa problemägaren – industrin?}
\item for the system to be secure in the sense that even with access to the uploaded files and the software, the data is not readable without the correct encryption key % \todo[inline, backgroundcolor=kth-lightblue]{för att tillfredsställa ingenjörssamfundet och vetenskapen – akademin) }
\item to analyze the throughput of the filesystem and compare to commercial distributed filesystems
% \item to be able to store a useful\footnote{} amount of data % \todo[inline, backgroundcolor=kth-lightblue]{eventuellt, för att uppfylla kursmålen – du som student}
\end{enumerate}

% \todo[inline]{In addition to presenting the goal(s), you might also state what the deliverables and results of the project are.}
