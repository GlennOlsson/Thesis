
\section{Goals}
% \todo[inline, backgroundcolor=kth-lightblue]{Mål}
% \todo[inline, backgroundcolor=kth-lightblue]{Skilj på syfte och mål. Syftet är att åstakomma en förändring i något. Målen är vad som konkret skall göras för att om möjligt uppnå den önskade förändringen (syfte). }

% \todo[inline]{State the goal/goals of this degree project.}

The project aims to create a secure, deniable filesystem that stores its data on online web services by taking advantage of the storage provided to its users. This can be split into the following subgoals:
\begin{enumerate}
\item to create a mountable filesystem where files and directories can be stored, read, and deleted,
\item for the filesystem to store all the data on online web services rather than on the local disk,
\item for the system to be secure in the sense that even with access to the uploaded files and the software, the data is not readable without the correct decryption key, 
\item to provide the user of the filesystem with plausible deniability of its data in the sense that it is not possible to associate the user with FFS if the filesystem is not mounted,
\item to analyze the write and read speed, storage capacity, and reliability of the filesystem and compare it to commercial cloud-based filesystems and local filesystems, and,
\item to analyze and discuss environmental and ethical aspects of the filesystem.
\end{enumerate}

% \todo[inline]{In addition to presenting the goal(s), you might also state what the deliverables and results of the project are.}
