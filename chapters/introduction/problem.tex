
\section{Problem}
\label{sec:problem}
% \todo[inline, backgroundcolor=kth-lightblue]{svensk: Problemdefinition eller Frågeställning\\
% Lyft fram det ursprungliga problemet om det finns något och definiera därefter
% den ingenjörsmässiga erfarenheten eller/och vetenskapen som kan komma ur
% projektet. }

Current cryptographic filesystems are often based on \mbox{local-disk} solutions. Distributed filessystem services, such as Google Drive, might encrypt your data, but it can be considered unsafe storage as they can leak your data -- since they do not always provide \mbox{end-to-end} encryption. in the case of Google Drive, client side encryption can be enabled by the administrators of a Google Workspace organization. Personal accounts can use Googles cryptographic software development kit Tink to enable client side encryption\,\cite{googleClientsideEncryptionKeys2023}. For both Workspace organizations and personal accounts, client side encryption is not enabled by default. A cryptographic and deniable decentralized \mbox{cloud-based} filesystem where the data is not controlled by any entity other than the user can be important, for instance, for journalists in unsafe countries. Social media services often provide free storage, which makes the social media service providers a potentially good host of the data for such a filesystem as they would not be able to access the unencrypted data nor have any idea how the posts are connected, and the usage of these services for data storage might even go unnoticed due to the existing heavy load of data from regular users of these services. This raises the following questions:
\begin{itemize}
    \item Is it possible to use the storage offered by various social media services to create a cryptographic and deniable filesystem where the data is stored on these \glspl{OWS} through the use of free user accounts? 
    
    \item What are the drawbacks of such a filesystem compared to similar filesystem solutions concerning write and read speed, storage capacity, and reliability?
    
    \item Are there advantages to such a filesystem regarding security and deniability? 
\end{itemize}

% Longer problem statement\\
% If possible, end this section with a question as a problem statement.

% % Research Question
% \subsection{Original problem and definition}
% \label{sec:researchQuestion}
% % \todo[inline, backgroundcolor=kth-lightblue]{Ursprungligt problem och definition}
% Some text

% \subsection{Scientific and engineering issues}% \todo[inline, backgroundcolor=kth-lightblue]{Vetenskaplig och ingenjörsmässig frågeställning}
% some text