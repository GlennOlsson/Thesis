
\section{Problem}
\label{sec:problem}
% \todo[inline, backgroundcolor=kth-lightblue]{svensk: Problemdefinition eller Frågeställning\\
% Lyft fram det ursprungliga problemet om det finns något och definiera därefter
% den ingenjörsmässiga erfarenheten eller/och vetenskapen som kan komma ur
% projektet. }

Current cryptographic filesystems are mainly based on \mbox{local-disk} solutions, and while services such as Google Drive might encrypt your data, it can be considered unsafe storage as they might give out your data. A cryptographic and deniable decentralized \mbox{cloud-based} filesystem where the data is not controlled by any entity other than the user can be of importance, for instance for journalists in unsafe countries. Social media services often provide free storage which makes it a potentially good host of the data in such a filesystem as they would not be able to access the unencrypted data nor have any idea how the posts are connected, and it might even go unnoticed due to their constant heavy load of data from regular users of the services. Is it possible to exploit the storage on various social media services to create a cryptographic and deniable filesystem where the data is stored on these online web services through the use of free user accounts? What are the drawbacks of such a filesystem compared to similar filesystem solutions with regard to write and read speed, storage capacity, and reliability? Are there advantages to such a filesystem in regard to security and deniability? 

% Longer problem statement\\
% If possible, end this section with a question as a problem statement.

% % Research Question
% \subsection{Original problem and definition}
% \label{sec:researchQuestion}
% % \todo[inline, backgroundcolor=kth-lightblue]{Ursprungligt problem och definition}
% Some text

% \subsection{Scientific and engineering issues}% \todo[inline, backgroundcolor=kth-lightblue]{Vetenskaplig och ingenjörsmässig frågeställning}
% some text