
\section{Purpose}
% \todo[inline, backgroundcolor=kth-lightblue]{Syfte}
% \todo[inline, backgroundcolor=kth-lightblue]{Skilj på syfte och mål! Syfte är att förändra något till det bättre. I examensarbetet finns ofta två aspekter på detta. Dels vill problemägaren (företaget) få sitt problem löst till det bättre men akademin och ingenjörssamfundet vill också få nya erfarenheter och vetskap. Beskriv ett syfte som tillfredställer båda dessa aspekter.\\
% Det finns även ett syfte till som kan vara värt att beakta och det är att du som student skall ta examen och att du måste bevisa, i ditt examensarbete, att du uppfyller examensmålen. Dessa mål sammanfaller med kursmålen för examensarbetskursen. 
% }
% \todo[inline]{State the purpose  of your thesis and the purpose of your degree project.\\
% Describe who benefits and how they benefit if you achieve your goals. Include anticipated ethical, sustainability, social issues, etc. related to your project. (Return to these in your reflections in Section~\ref{sec:reflections}.)}

The purpose of this paper is to explore the possibility to store data for free on online services, and to compare the performance of such a system to a professional filesystem service that costs money. The interesting aspect of this is that services such as social medias provides users with essentially infinite amount of storage. Anyone can create any amount of accounts on Twitter and Facebook, and with enough accounts one could potentially store all their data using such a filesystem. The thesis explores the worth to use a free filesystem that is potentially slower than one that costs a few dollars per month. Further, is it ethically defendable to use such a system?

% Being able to advantaging just a few services to create a useable filesystem that can store certain amount of data means that further work can be done to extend the filesystem with even more services and thus achieving even more storage.