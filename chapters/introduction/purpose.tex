
\section{Purpose and motivation}
% \todo[inline, backgroundcolor=kth-lightblue]{Syfte}
% \todo[inline, backgroundcolor=kth-lightblue]{Skilj på syfte och mål! Syfte är att förändra något till det bättre. I examensarbetet finns ofta två aspekter på detta. Dels vill problemägaren (företaget) få sitt problem löst till det bättre men akademin och ingenjörssamfundet vill också få nya erfarenheter och vetskap. Beskriv ett syfte som tillfredställer båda dessa aspekter.\\
% Det finns även ett syfte till som kan vara värt att beakta och det är att du som student skall ta examen och att du måste bevisa, i ditt examensarbete, att du uppfyller examensmålen. Dessa mål sammanfaller med kursmålen för examensarbetskursen. 
% }
% \todo[inline]{State the purpose  of your thesis and the purpose of your degree project.\\
% Describe who benefits and how they benefit if you achieve your goals. Include anticipated ethical, sustainability, social issues, etc. related to your project. (Return to these in your reflections in Section~\ref{sec:reflections}.)}

The purpose of this research is to explore the possibility to create a secure, steganographic cloud-based filesystem that stores data on online services and to compare the performance, benefits and disadvantages of such a filesystem to existing steganographic filesystems and distributed filesystem services. A distributed filesystem service, such as Google Drive, provide data storage for users which can be both free or cost money. Even though Google Drive encrypt the user's data, they control the encryption and decryption keys, and the method of encryption\,\cite{johnsonGoogleDriveSecure2021}. This means that they can give out the user's files and data if faced with legal actions such as subpoenas. It also opens up the possibility of hackers gaining access to the files without the user having any way to control it.

The idea behind FFS is to have a cloud-based filesystem without anyone except the user being able to control the unencrypted data. We aim to exploit the storage web services give their users for free. By encrypting the files locally before uploading them to these services, the user is the only one who have access to the encryption and decryption keys and therefore unencrypted data. Even if the web service would give out the data uploaded by the user, it is not readable without the decryption key.

An interesting aspect of this is that online web services, such as social media, provide users with essentially an infinite amount of storage for free. Anyone can create any number of accounts on Twitter and Facebook without cost, and with enough accounts, one could potentially store all their data using such a filesystem. The thesis explores the use of such a filesystem despite potentially being slower and less dependable than filesystems that utilize other types of storage media, such as distributed filesystem services and local filesystems. Further, is it ethically defendable to create and use such a system?