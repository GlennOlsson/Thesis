
\section{Purpose and motivation}
% \todo[inline, backgroundcolor=kth-lightblue]{Syfte}
% \todo[inline, backgroundcolor=kth-lightblue]{Skilj på syfte och mål! Syfte är att förändra något till det bättre. I examensarbetet finns ofta två aspekter på detta. Dels vill problemägaren (företaget) få sitt problem löst till det bättre men akademin och ingenjörssamfundet vill också få nya erfarenheter och vetskap. Beskriv ett syfte som tillfredställer båda dessa aspekter.\\
% Det finns även ett syfte till som kan vara värt att beakta och det är att du som student skall ta examen och att du måste bevisa, i ditt examensarbete, att du uppfyller examensmålen. Dessa mål sammanfaller med kursmålen för examensarbetskursen. 
% }
% \todo[inline]{State the purpose  of your thesis and the purpose of your degree project.\\
% Describe who benefits and how they benefit if you achieve your goals. Include anticipated ethical, sustainability, social issues, etc. related to your project. (Return to these in your reflections in Section~\ref{sec:reflections}.)}

The purpose of this research is to explore the possibility to create a cryptographically secure and deniable \mbox{cloud-based} filesystem that stores data on an \gls{OWS} and to compare the performance, benefits, and disadvantages of such a filesystem to existing deniable filesystems and distributed filesystem services. A distributed filesystem service, such as Google Drive, provides data storage for users which can be free or cost money. Even though Google Drive encrypts the user's data, they control the encryption and decryption keys, and the method of encryption\,\cite{johnsonGoogleDriveSecure2021}. This means that they can give out the user's files and data if faced with legal actions, such as subpoenas. Because the service providers have the decryption keys, this opens up the possibility of hackers gaining access to the files without the user having any way to prevent this access.

% -- Even using a \mbox{cloud-based} filesystem service as a layer in a stacked filesystem with a cryptographic and deniable layer, the amount of data stored in the system would still be visible to the service provider even if some or even all of that data is noise. It is also an obvious way to store data on a service that provides conventional storage. 

The idea behind \gls{FFS} is to have a decentralized \mbox{cloud-based} filesystem where only the user has access to the unencrypted data. By encrypting and decrypting the files locally before uploading and after downloading them to these services (\mbox{end-to-end} encryption), it is possible to ensure that the user is the only one who has access to the encryption and decryption keys and therefore the unencrypted data. Even if the web service would look at the data uploaded by the user, it is unreadable without the decryption key. An interesting aspect of this is that online web services, such as social media, provide users with essentially an unbounded amount of storage for free. Anyone can create any number of accounts on Twitter and Facebook without cost, and with enough accounts, one could potentially store all their data using such a filesystem. We aim to exploit the storage web services give their users for free. We also want adversaries to be unable to prove the amount of data and the existence of data, even when the images are posted publicly.

There are several deniable filesystems available but these lack certain aspects that \gls{FFS} aims to solve. Some filesystems are based on the local disk of the device in use, such as the physical storage device on a computer or phone, or an external storage device connected to a computer or phone. While these filesystems have advantages compared to \mbox{cloud-based} solutions, such as low latency, they lack accessibility as you need to have the device to access the content on it. It also means that when you want to share or transport the data, you must physically move the device which can lead to problems, such as it could be taken from you or be destroyed. \mbox{Cloud-based} solutions counter this by being available from any location that has internet access to the services used. However, existing \mbox{cloud-based} solutions introduce other disadvantages. One example is CovertFS\,\cite{baliga2007web} where data is stored in images posted on web services. The images are actual images representing something, meaning that there is a limit on how much data can be stored in a steganographic fashion. CovertFS limit this to \SI{4}{\kilo\byte} which means that such a filesystem with a lot of data will require many images which could lead to suspicion from the owners of the web services. \gls{FFS} stores as much data as possible in the images, meaning that fewer images are needed to store a file bigger than \SI{4}{\kilo\byte}. This also means that the images produced by \gls{FFS} does not look like a normal image, but instead have seemingly randomly colored pixels. More examples of similar filesystems will be presented in Chapter~\ref{ch:related_work}. 