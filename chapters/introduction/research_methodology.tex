
\section{Research Methodology}%\todo[inline, backgroundcolor=kth-lightblue]{Undersökningsmetod}
% \todo[inline, backgroundcolor=kth-lightblue]{Här anger du vilken vilken övergripande undersökningsstrategi eller metod du skall använda för att försöka besvara den akademiska frågeställning och samtidigt lösa det e v ursprungliga problemet. Ofta kan man använda ”lösandet av ursprungsproblemet” som en fallstudie kring en akademisk frågeställning. Du undersöker någon intressant fråga i ”skarpt” läge och samlar resultat och erfarenhet ur detta.\\
% Tänk på att företaget ibland måste stå tillbaka i sin önskan och förväntan på projektets resultat till förmån för ny eller kompletterande ingenjörserfarenhet och vetenskap (ditt examensarbete). Det är du som student som bestämmer och löser fördelningen mellan dessa två intressen men se till att alla är informerade. }
% \todo[inline]{Introduce your choice of methodology/methodologies and method/methods – and the reason why you chose them. Contrast them with and explain why you did not choose other methodologies or methods. (The details of the actual methodology and method you have chosen will be given in Chapter~\ref{ch:methods}. Note that in Chapter~\ref{ch:methods}, the focus could be research strategies, data collection, data analysis, and quality assurance.)\\
% In this section you should present your philosophical assumption(s), research method(s), and research approach(es).}

The filesystem created through this thesis will be developed on a Macbook laptop running macOS Monterey, version 12.3.1. It will be written in C++20 and use the Filesystem in Userspace (FUSE) MacOS library\,\cite{HomeMacFUSE} which enables the writing of a filesystem in userspace rather than in kernel space. FUSE is available on other platforms too, such as Linux, but the filesystem will be developed on a Macbook laptop thus macFUSE is chosen. C++ is chosen because the FUSE API is available in C, and C++ version 20 is well established and used. Further details about the development environment will be found in Section~\ref{sec:dev_env}.

The resulting filesystem will be evaluated against other filesystems, both commercial distributed systems, such as Google drive, and an instance of Apple File System (APFS)\,\cite{appleincAppleFileSystem} on the Macbook laptop referenced above. Quantitative data will be gathered from the different filesystems through the use of experiments with the filesystem benchmarking software IOzone\,\cite{IozoneFilesystemBenchmark}. IOzone was chosen because it is, compared to tools such as Fio and Bonnie++, simpler to use while still powerful\,\cite{agarwalComparingIOBenchmarks2018}. We will look at attributes such as the differences in read and write speeds between different filesystems, as well as the speed of random read and random write. However, according to \citeauthor{tarasovBenchmarkingFileSystem2011}, benchmarking filesystems using benchmarking tools is difficult to perform in a standardized way\,\cite{tarasovBenchmarkingFileSystem2011} which will be taken into consideration during the evaluation and when concluding the thesis. Further discussion about this will be found in Section~\ref{sec:iozone}.