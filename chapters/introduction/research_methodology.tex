
\section{Research Methodology}%\todo[inline, backgroundcolor=kth-lightblue]{Undersökningsmetod}
% \todo[inline, backgroundcolor=kth-lightblue]{Här anger du vilken vilken övergripande undersökningsstrategi eller metod du skall använda för att försöka besvara den akademiska frågeställning och samtidigt lösa det e v ursprungliga problemet. Ofta kan man använda ”lösandet av ursprungsproblemet” som en fallstudie kring en akademisk frågeställning. Du undersöker någon intressant fråga i ”skarpt” läge och samlar resultat och erfarenhet ur detta.\\
% Tänk på att företaget ibland måste stå tillbaka i sin önskan och förväntan på projektets resultat till förmån för ny eller kompletterande ingenjörserfarenhet och vetenskap (ditt examensarbete). Det är du som student som bestämmer och löser fördelningen mellan dessa två intressen men se till att alla är informerade. }
% \todo[inline]{Introduce your choice of methodology/methodologies and method/methods – and the reason why you chose them. Contrast them with and explain why you did not choose other methodologies or methods. (The details of the actual methodology and method you have chosen will be given in Chapter~\ref{ch:methods}. Note that in Chapter~\ref{ch:methods}, the focus could be research strategies, data collection, data analysis, and quality assurance.)\\
% In this section you should present your philosophical assumption(s), research method(s), and research approach(es).}

A literature review was carried out to examine existing cryptographic, deniable, and cloud-based filesystems, as well as state-of-the-art security standards. This created a basis for the technologies and security principles used in the produced filesystem, including \gls{FUSE} as the filesystem library. Furthermore, experiments were carried out to gather quantitative data to compare the performance of \gls{FFS} against other relevant filesystems. 
