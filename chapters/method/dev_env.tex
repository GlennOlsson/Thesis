\section{Development environment specification}
\label{sec:dev_env}
Development of FFS was done on a 2016 year model Macbook Pro laptop with \SI{2.6}{\giga\hertz} Quad-Core Intel Core i7 processor and \SI{16}{\giga\byte} \SI{2133}{\mega\hertz} LPDDR3 memory. The storage device of the computer was a \SI{250}{\giga\byte} SSD, and the filesystem used was an encrypted APFS partition. The computer was running macOS Monterey 12.5

FFS was developed using C++20 and compiled using Apple clang version 13.0.0, using target x86\_64\-apple\-darwin21.4.0. FFS uses the ImageMagick Magick++ library\,\cite{ImageMagick2022} for image processing. Version 7.1.0\-29 of Magick++ is used by FFS. macFUSE\,\cite{HomeMacFUSE} version 4.2.5 is used for FFS to use the FUSE API. FUSE API version 26 is used. cURLpp\,\cite{barrette-lapierreCURLpp2022} is a cURL\,\cite{CurlCurl2022} C++ wrapper used by FFS to make HTTP requests. Version 0.8.1 of cURLpp is used by FFS. libOauth\,\cite{Liboauth} version 1.0.3 is used by FFS to sign and encode HTTP request according to the OAuth\,\cite{barrette-lapierreCURLpp2022} standard. Flickcurl\,\cite{beckettFlickcurlLibraryFlickr} version 1.26 is a C library used by FFS to communicate with parts of the Flickr API. Crypto++\,\cite{CryptoLibraryFree} is a C++ library providing cryptographic schemes. FFS uses Crypto++ to encrypt end decrypt the data stored in FFS, and to derive the keys used in the encryption and decryption algorithm. Crypto++ version 8.6 is used by FFS.

FFS was developed for use on a single computer for simplicity, and the version used for the operating system, libraries and tools were the most recent up-to-date versions when development of the filesystem started. To avoid re-writing the source code to handle new API designs, these versions remained the same throughout the development process.