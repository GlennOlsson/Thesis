\section{FFS}
The artifact that was developed as a result of this thesis is the Fejk FileSystem (FFS). It uses an online web service (\gls{OWS}) to store the data but behaved as a mountable filesystem for the users. The filesystem is a proof-of-concept and does not support all functionalities that other filesystems do, such as links or access permissions. The reasoning is that these behaviors are not required for a useable system, and when comparing FFS to distributed filesystems such as Google Drive, many of these other filesystems also often do not support functionality such as links.

\subsection{Design overview}
FFS uses images to store the data of files, directories and the inode table of the filesystem. These images will are uploaded to the OWS, such as Flickr, as image posts. As mentioned in Section~\ref{sec:ows}, there can be limitations to these posts for certain OWSs. To support file sizes bigger than these limitations, bigger files will be split into multiple posts, requiring FFS to keep track of a list of posts. Figure~\ref{fig:ffs_inode_diag} presents the basic outline of FFS and a example content of the filesystem. FFS is based on the idea of inode filesystems and uses a inode table to store information about the files and directories in the filesystem. However, instead of an inode pointing to specific blocks in a disk, the inode table of FFS will instead keep track of the id numbers of the posts on the OWS where the file or directory is located. The inode table entry for each file or directory will also contain metadata about the entry, such as its size and a boolean indicating if the entry is a directory or not.

\begin{figure}[!ht]
	\begin{center}
	  \includegraphics[width=0.8\textwidth]{figures/ffs_inode_diagram.png}
	\end{center}
	\caption{Basic structure of FFS inode-based structure}
	\label{fig:ffs_inode_diag}
\end{figure}

The directories and inode table are represented as classes in C++. Appendix~\ref{app:inode_dir_code} visualizes the main attributes of the \texttt{Directory}, \texttt{InodeTable}, and \texttt{InodeEntry} classes. There can be multiple \texttt{Directory} and \texttt{InodeEntry} objects in the computers' memory and in the filesystem, but there will only exist one \texttt{InodeTable} instance which is relevant. The \texttt{Directory} class is a data structure that stores mappings between filenames and the files' and directories' inode for all files and directories stored in that directory. The \texttt{InodeEntry} is a data structure that keeps track of a file's or directory's information, such as where the data is stored and its metadata, such as size and creation timestamp. The \texttt{InodeTable} stores a mapping between an inode and the files' \texttt{InodeEntry}, and stores all the \texttt{InodeEntry} objects. The \texttt{InodeTable} always has at least one entry which is the root directory. This entry has a constant inode value of 0 for simplicity to look up the root directory. With the help of the root directory, all the files lower in the directory hierarchy can be found. The inode of all files and directories other than the root directory has a unique inode greater than 0. The \texttt{InodeTable} is always the most recent image saved on the OWS, making it easy to find it on the OWS.

To read the content of a known file in a directory has three steps:
\begin{enumerate}
	\item The \texttt{Directory} object of the directory provides the inode of the given filename.
	\item The inode is used to get the \texttt{InodeEntry} from the \texttt{InodeTable}.
	\item Using the inode entry, the the file can be located.
\end{enumerate}
The location of a file or directory is an ordered list of unique IDs of the image posts on the OWS. The data received by downloading these images, decoding them (as described in Subsection~\ref{subsec:file_enc_dec}), and concatenating them, can be read as a file or represented as a \texttt{Directory} object, depending on if the \texttt{InodeEntry} was marked as a file or a directory. 

As directories only know the filenames inode, the \texttt{Directory} object does not have to be saved again (and thus uploaded) when a file or directory in it is edited, for instance adding data. Only the \texttt{InodeEntry}, and thus the \texttt{InodeTable}, needs to be updated with the new post IDs of the new file or directory. This saves computation time as every request to the OWS takes time. However, if the filename is edited or the file or directory is moved to another location, the parent directory of the file or directory would have to be edited, and such its corresponding \texttt{Directory} object has to be updated.

When a new file or directory is created, it is saved in its parent directory with its filename and an inode. The same inode is used in the inode table to keep track of the file's or directory's inode entry. As shown in Appendix~\ref{app:inode_dir_code}, the inode is represented as a unsigned 32-bit integer. The inode is calculated by adding one to the currently greatest inode. This means that new files and directories will always receive a higher greater inode than the ones currently in the inode table. This naïve approach to inode generation does not take in to account that there might be an available inode less than the greatest inode in the inode table (for instance, due to deletion of a previously created file). However, this inode generation approach is fast and will not be a problem until the integer overflows. As the inode is represented using a 32-bit integer, FFS would need to have saved over four billion files before the inode value would overflow. This scenario is not in the scope of this proof-of-concept filesystem.

FFS does not support all filesystem operations that are implementable through FUSE, instead FFS implements a subset of them. The implemented functions are shown in Table~\ref{tbl:fs_impl_op}. The implemented operations are the most vital operations required for a working filesystem\,\cite{kuenningCS135FUSEDocumentation2010}. Operations such as \texttt{chown} provides extended capabilities of the filesystem but these are not required for a proof-of-concept filesystem. The functionality of the filesystem operations implemented by FFS and their implementation details are described in Subsection~\ref{subsec:file_op}. 

\begin{table}[!ht]
	\begin{center}
		\caption{Filesystem operations implementable trough the FUSE API, and wether or not FFS implements them}
		\begin{tabular}{| c | c |}
			
			\hline
			\textbf{Filesystem operation} 	& \textbf{Implemented by FFS}\\
			\hline
			\hline
			\texttt{open} & Yes\\
			\texttt{opendir} & Yes\\
			\texttt{release} & Yes\\
			\texttt{releasedir} & Yes\\
			\texttt{create} & Yes\\
			\texttt{mkdir} & Yes\\
			\texttt{read} & Yes\\
			\texttt{readdir} & Yes\\
			\texttt{write} & Yes\\
			\texttt{rename} & Yes\\
			\texttt{truncate} & Yes\\
			\texttt{ftruncate} & Yes\\
			\texttt{unlink} & Yes\\
			\texttt{rmdir} & Yes\\
			\texttt{getattr} & Yes\\
			\texttt{fgetattr} & Yes\\
			\texttt{statfs} & Yes\\
			\texttt{access} & Yes\\
			\texttt{utimens} & Yes\\
			\texttt{readlink} & No\\
			\texttt{symlink} & No\\
			\texttt{link} & No\\
			\texttt{chmod} & No\\
			\texttt{chown} & No\\
			\texttt{fsync} & No\\
			\texttt{fsyncdir} & No\\
			\texttt{lock} & No\\
			\texttt{bmap} & No\\
			\texttt{setxattr} & No\\
			\texttt{getxattr} & No\\
			\texttt{listxatt} & No\\
			\texttt{ioctl} & No\\
			\texttt{flush} & No\\
			\texttt{poll} & No\\
			\hline

		\end{tabular}
		\label{tbl:fs_impl_op}
	\end{center}
\end{table}

A file, a directory, or the inode table has to be uploaded to the OWS when it is modified to save its current information. As it takes time to make requests to the OWS, FFS is created to make as few requests as possible while still saving the data required. Therefor, only the directory or file that is affected by a change is uploaded to the system, while the ones unaffected can remain the same. The inode table has to be updated with every change of a file or directory as it contains the location of the file or directory.

FFS can be mounted to the local filesystem using FUSE, similar to how you can mount a network drive like an File Transfer Protocol (\gls{FTP}) server. The mounted FFS volume operates similar to any other drive, and can be accessed using, for instance, Apple's Finder or a Z Shell terminal.

\subsection{Cache}
FFS implements a simple in-memory Least Recently Used (\gls{LRU}) cache for the downloaded content. The cache consists of two data structures: 
\begin{itemize}
	\item a Cache Map - a mapping between a post ID and its image data, and
	\item a Cache Queue - a queue keeping track of the cached post IDs.
\end{itemize}
The cache stores a maximum of 20 image posts. The data stored in the cache is the decrypted image data. To avoid FFS to use too much memory, the cache is configured so that images greater than \SI{5}{\mega\byte} are not cached. Each time an image is uploaded or downloaded, it is added to the Cache Map with its post ID as the key. The post ID is also added to the beginning of the Cache Queue. If the Cache Queue exeeds 20 elements, the last elements of the queue is removed, and the corresponding entry in the Cache Map is erased, thus the entry is fully erased from the cache. The queue ensures that the chache is limited to 20 entries, and by using the first in first out valuation method, the queue also ensures that the oldest element in the cache is removed when the cache exeeds the limit. When a file or directory is removed from the filesystem, all its data is also removed from the cache, if it stored there.

Before a post with a specified post ID is downloaded from the OWS, the cache is checked to see if it is storing this post ID. If it is, the stored image is returned. Otherwise, the process continues by downloading the image from the OWS. When the thesis states that a file or directory is downloaded, it is implied that the cache is also checked and the data is possibly returned by the cache instead of requiring to download the data from the OWS.

FFS separately caches both the root directory and the inode table. As both of these data structures are used in many of the filesystem operations, it is important that they can be accessed quickly and not be removed from the cache. Their cache entries are updated when the files are uploaded to the OWS.

\subsection{Encoding and decoding objects}
\label{subsec:file_enc_dec}
Objects that FFS stores, and therefore also encodes and decodes, are: files, directories, and the inode table. All of these objects are stored on the OWS using PNG images with 16 bit RGB color depth. The inode table and the directories are represented as C++ objects in memory during runtime, but are serialized into a binary representation before they are encoded into images. A detailed description of these binary formats is described in Appendix~\ref{app:binary_rep}. The files saved to FFS are also read in to memory in a binary format before being encoded and uploaded to the OWS. 

The input to the image encoder is the binary data do encode as an image. A header (FFS header) is prepended to the binary data, containing among other things, the size of the data and a timestamp of when the data was encoded. The FFS header and the input data is encrypted using authenticated encryption, utilizing GCM and AES. The key used for the encryption is derived using the HKDF function utilizing the SHA-256 hashing algorithm, along with a \SI{64}{\byte} salt vector, re-generated with random data every time new data is being encrypted. The salt is stored with the cipher to ensure that the decryption algorithm uses the same salt to derive the decryption key. The secret used in the HKDF is a password provided by the user. HKDF also uses a initialization vector, re-generated with random data every time new data is being encrypted. The length of the IV is set to 12 bytes. The resulting data from the encryption is the salt, the IV, the encrypted cipher (including the authentication tag). These three data points are concatenated into a string of bytes. This string of bytes is referred to as the Complete Encrypted Data (\gls{CED}).

The dimensions of an FFS image is based on the amount of bytes stored, as described in Section~\ref{sec:data_storage}. The stored data is the CED, prepended with the length of the CED (\gls{LCED}) using 4 bytes. For an image of $X = ceil(\frac{4 + LCED}{6})$ pixels, FFS will set the width $w$ of the image as $w = ceil(\sqrt{X})$. Further, the height $h$ of the image is set as $h = ceil(\frac{X}{w})$. This will require $(w * h) - X$ filler bytes, and will create an image with similar height and width. For certain values of $X$, $h$ will be equal to $w$. For other values of $X$, $h = w-1$. The resulting data encoded in the image is, in order:
\begin{itemize}
	\item 4 bytes representing the LCED,
	\item The CED data, and
	\item Filler bytes
\end{itemize}
The filler bytes are randomized bytes.

The data consisting of the LCED, CED and filler bytes is encoded in to pixel color data for a PNG with 16 RGB bit color depth using the Magick++ library. The result is an image, with a high probability, of what looks like randomized colors for each pixel. This is due to the fact that most pixels are encrypted and therefore the bytes representing this data is seemingly random.

To decode an FFS image, the decoder first interprets the 4 first bytes as the LCED. The salt and IV are retrieved from the CED as they are of known length. The decryption key is derived using the IV and salt, and results in the same key as used in the encryption step because AES is a symmetric cipher algorithm. The remaining bytes of the CED are decrypted using the decryption key. The unencrypted data consists of the FFS header concatenated with the original stored data. The FFS header is asserted to be in the correct format, before the stored binary data is returned from the decryption function. Figure~\ref{fig:file_enc_dec} visualizes the encoder and decoder for all data saved in FFS.

\begin{figure}[!ht]
	\begin{center}
	  \includegraphics[width=1.0\textwidth]{figures/encoder_decoder.png}
	\end{center}
	\caption[Simple visualization of the encoder and decoder of FFS]{Simple visualization of the encoder and decoder of FFS. The input of the encoder is the binary data to store in FFS, eg. a file, and the output is the FFS image to upload to the OWS. The input to the decoder is an FFS image, and the output is the binary data stored on FFS, eg. a file}
	\label{fig:file_enc_dec}
\end{figure}

The encryption and decryption methods used are state-of-the-art solutions as defined and implemented by Crypto++\,\cite{CryptoLibraryFree}. Crypto++ is a well-used and well maintained C++ library for cryptography, and as of writing has no reported CVE security vulnerabilities for the functionality used by FFS\,\cite{CryptoppSecurityVulnerabilities}.

An FFS image has an upper size limit, defined by the OWS used. This limit is defined further down in this section. If the data to be stored in FFS, such as a file, exceeds this limit, it is split into multiple encoded images. These images will have no association with each other and will be encrypted using different salts and IVs. Only the inode table stores the different post IDs in the order they are encoded in. Files and directories stored in FFS can be separated in to multiple images, however the inode table is limited to only one image for simplicity when interacting with the OWS. This introduces a size limit of the inode table, limiting the filesystem further. More details about the limits are found in Subsection~\ref{subsec:ffs_limits}.

\subsection{Online web services}
As FFS is a proof-of-concept filesystem, it only uses one OWS as its storage medium. However, for a production filesystem, multiple OWSs would be beneficial. This would enable features such as redundancy by using replication over multiple OWSs, for instance in case one OWS would stop working.

The initial intention of FFS was to use Twitter as the OWS. Initial research for the thesis found that it was possible to upload a file and download the same file without any data loss. However, it was later found that this was not a reliable conclusion. Some images uploaded to Twitter were converted to another image format when they were stored by Twitter, which meant that the decoder could not decode the data as it expected another image format. Other images where compressed or recoded which led to data loss when downloading the image. As the decoder of FFS images relies on a specific binary representation of the image, this meant that the images could not be decoded into the previously uploaded data. Twitter has previously publicly announced changes to the way they store images\,\cite{nolanobrienUpcomingChangesPNG2018} and even suggested workarounds\,\cite{nolanobrienFeedbackUpcomingChanges2019} for users who are concerned about the potential data loss. However, during research for the thesis, it was concluded that the workarounds mentioned in \cite{nolanobrienFeedbackUpcomingChanges2019} does no longer work on Twitter. For instance, there have been found PNG images less than 900x900px that have been uploaded, have not been able to be downloaded to the same image, which contradicts the workaround mentioned by the Twitter employee. It is possible that further changes have been made to the data management of images on Twitter, however an official announcement has not been found.

Flickr saves the original version of the uploaded image and thus it can be used to download the same image as was uploaded. This also means that a file that is encoded into an FFS-encoded image can be uploaded, downloaded, and decoded into the same file as before. While they do not assure that they will always support original images, they also do not indicate that this would change. Therefore, Flickr can be used at this moment for the proof-of-concept filesystem that FFS is. A free-tier Flickr account is therefore used for FFS. 

Flickr provides an extensive free REST API for non-commercial use. A user can create applications and generate access tokens for the application. These application tokens are later used to request tokens from users who authenticate using Flickr's web interface, and allow the application to do requests for the user. The application will then receive access tokens for the user, which are used to authenticate with the API for the API calls that require authentication.

Flickr provides the ability to search for all the images posted by a user, and to sort this result by time of posting. Every time an image is uploaded to Flickr, it is due to some modification in the filesystem, for instance a write operation to a file or a creation of a new directory. For every modification in the filesystem, the inode table will have to be updated. Therefore, we can ensure that the inode table is always the most recently uploaded image to Flickr by configuring FFS to upload all other images first, for instance the newly written file. This provides FFS with a simple way of querying the inode table from Flickr - by simply requesting the most recently uploaded image by the Flickr account.

While the Flickr API is extensive in its functionality, FFS only uses a few of the provided capabilities. The Flickr API capabilites that FFS utilizes are:
\begin{itemize}
	\item Upload an image and return the post ID,
	\item Query the most recent image by a user, and return the URL and post ID of the original uploaded image,
	\item Get the URL to the original uploaded image given a post ID, and
	\item Remove an image given a post ID.
	\item Get the image data of the image given its URL
\end{itemize}

For instance, to download the original image given a post ID, two requests are required:
\begin{enumerate}
	\item Getting the URL to the original image using the post ID,
	\item Downloading the image from the URL received from the previous request.
\end{enumerate}

For benchmarking purposes, a fake variant of FFS, Fejk FFS (\gls{FFFS}), has also been developed. FFFS uses a Fake OWS (\gls{FOWS}), which stores the data on the local filesystem. The FOWS is used by FFFS similar to how Flickr is used by FFS, by storing encoded images in it. By storing the images on the local filesystem, the filesystem operations duration is shorter as the local filesystem operations are in general faster than the network requests. This makes it easier to conclude how much of the filesystem operation time is affected by the time of the network requests. The time $T$ of an FFS filesystem operation can be modeled like:
$$
	T = t_{ffs} + t_{ows}
$$
where $t_{ffs}$ is the time that FFS takes to, for example for a file read operation;
\begin{itemize}
	\item to find the file in the inode table,
	\item decode and decrypt the image data,
	\item read the specified amount of data, and,
	\item to output the data
\end{itemize}
This time will be approximately consistent for the same request. However, cache misses/hits in the filesystem and process scheduling can fluctuate the value of $t_{ffs}$. $t_{ows}$ is the total time required to complete all requests to the OWS for a filesystem operation. For instance, for a similar read operation as above;
\begin{itemize}
	\item to download all the directories in the file path,
	\item query the Flickr API for URL pointing to the most recently uploaded image,
	\item download the image representing the inode table, and,
	\item to download the images representing the file to read
\end{itemize}
Depending on the OWS, the latency and bandwidth of the internet connection between the user's machine and the OWS's server can differ a lot. Duplicate requests to the same OWS can also differ significantly due to, for instance, server load balancing and a difference in request quantity from other users at the time of the requests. However, for a FOWS, $t_{ows}$ can be replaced by $t_{fows}$ which will have approximately consistent values for duplicate operations, because the local filesystem is not affected as much by load balancing. The local filesystem requests by other applications on the machine can also be influenced and minimized by not using other applications on the machine while running the benchmarking tool to ensure filesystem requests by the FOWS can be handled quickly by the operating system. However, $t_{fows}$ is affected by, among other things, the underlying storage device of the local filesystem and process scheduling which can still fluctuate the value of $t_{fows}$.

Due to limitations in the library \texttt{Flickcurl} used for uploading images to Flickr, the image to be uploaded to Flickr first has to be saved to the local filesystem. \texttt{Flickcurl} reads the file from disk, before uploading it. Therefore, FFS saves a temporary file on the local filesystem when data is uploaded to Flickr. The temporary file is stored in the \texttt{/tmp} directory of the local filesystem, and is removed directly after the file has been uploaded. However, it is not certain that the operating system removes or overwrites the file data on the storage device, and thus there are ways to recover the deleted data, by for instance adversaries\,\cite{llcsysdevlaboratoriesHowRecoverData2022,cedricAPFSDataRecovery2022,santosHowRecoverData2021}. Although, these methods require you to decrypt the APFS volume, requiring the decryption password. Without this password, the data cannot be recovered. Even with the decryption password, it is not certain that the data is recoverable.

\subsection{Implemented filesystem operations}
\label{subsec:file_op}
Following is a detailed description of all the FUSE operations implemented by FFS, and how they are implemented by FFS. Further explanations about the intended functionality of the operations can be found in \,\cite{kuenningCS135FUSEDocumentation2010}. 

The path of a file is sometimes provided for the filesystem operation and traversed by FFS to understand the requested location. An example path is \texttt{/foo/bar/buz.txt} or \texttt{/foo/bar/baz/}. A path is traversed like the following pseudo code:
\begin{lstlisting}[language=python, caption={Pseudocode of traversing a given path, returning the \texttt{Directory} and the filename}, label=lst:traverse_path,breaklines=true]
# Traverse a given path and return the parent directory object
#  and filename of the path
traverse_path(path) -> (Directory, string):
	# Fetches inode table from the OWS
	inode_table := get_inode_table()
	
	split_path := path.split("/")
	# The filename could be either the name of a file 
	#  or the name of a directory
	filename := split_path.last
	dirs := split_path.remove_last()

	# Get the root dir from cache
	curr_dir = cache.get_root_dir()

	# While there are still directories to traverse,
	#  get the next directory in the list from current
	#  directory
	while(!dirs.empty())
		dir_name := dirs.pop_first()
		inode := curr_dir.inode_of(filename=dir_name)
		inode_entry = inode_table.entry_of(inode=inode)
		# Download the image posts defined by the 
		#  post IDs in the inode entry
		curr_dir = download_as_dir(inode_entry)
	
	return (curr_dir, filename)

\end{lstlisting}

By traversing a path, FFS has to fetch all parent directories in the hierarchy. The file or directory with the filename is not fetched during while traversing the path, as it might not be necessary for the operation. This implies that all operations that relies on the path of the file or directory has to download all parent directories of the path. However, the directories in the path could be cached and therefore not require a download from the OWS. Further, \texttt{open}, \texttt{opendir}, and \texttt{create} can associate a file handle with a file or directory, so that certain other operations can use the file handle instead of traversing the string path. This saves time because the path traversing result is saved in the filesystem state.

After every operation that modifies the inode table, the inode table is uploaded to the OWS and cached. Therefore, it is assumed that the inode table is always up to date in memory and on the OWS. This will be true as long as there are not multiple FFS instances working with the same OWS account at the same time. This scenario has undefined behavior as there is no locking implemented for FFS.

All filesystem operations are synchronous unless specified. Further, FUSE is running in single-thread mode meaning that a filesystem operation call must complete before another can begin. This helps limiting the risk of data races as two processes cannot call different operations that, for instance, modify the inode table at the same time.

\subsubsection{open}
Given a path to a file, the file is associated with a file handle. The file handle is used in subsequent operations to avoid traversing the filepath. The file is not downloaded from the OWS, only the parent directories are downloaded during the path traversing as explained above. An \texttt{open} call must, eventually, be followed by a \texttt{release} call. Although, multiple other operation calls can occur between these events.

\subsubsection{release}
Given a file handle, this operation closes the file in the filesystem, disassociating the file handle with the file. The current states of the file and the inode table are saved to the OWS, and the previous versions of the file and inode table are deleted from the OWS. Subsequent operations for the file will require path traversing as the file handle can no longer be used.

The file must have a file handle associated with it before \texttt{release} is called. This requires a preceding \texttt{open} or \texttt{create} call for the file.

\subsubsection{opendir}
Given a path to a directory, the directory and associated with a file handle. The file handle is used in subsequent operations to avoid traversing the filepath. The directory is not downloaded from the OWS, only the parent directories are downloaded during the path traversing as explained above. An \texttt{opendir} call must, eventually, be followed by a \texttt{releasedir} call. Although, multiple other operation calls can occur between these events.

\subsubsection{releasedir}
Given a file handle, this operation closes the directory in the filesystem, disassociating the file handle with the directory. The current states of the directory and the inode table are saved to the OWS, and the previous versions of the directory and inode table are deleted from the OWS. Subsequent operations for the file will require path traversing as the file handle can no longer be used.

The directory must have a file handle associated with it before \texttt{releasedir} is called. This requires a preceding \texttt{opendir} call.

\subsubsection{create}
This operation creates an empty file in the filesystem given a path, and associates a file handle with the file, similar to \texttt{open}. The empty file will not be uploaded to the OWS as it has no data associated with it. A new entry is added to the parent directory with the filename and a generated inode, and the parent directory is uploaded to the OWS. The new posts representing the parent directory in the OWS is associated with the inode entry of the parent directory in the inode table, and the old posts are deleted in the OWS. An new inode entry is also created in the inode table, representing the new, empty, file.

\subsubsection{mkdir}
This operation creates an empty directory in the filesystem given a path. The directory is not uploaded to the OWS as it has no data associated with it. The parent directory is modified so it is uploaded to the OWS, and the old versions of the parent directory is deleted on the OWS. The parent directory entry in the inode table is modified with the new posts, and a new entry is created for the new directory. The inode table is updated in the OWS.

As opposed to \texttt{create} for files, this operation does not associate a file handle with the directory.

\subsubsection{read}
This operation reads a number of bytes, starting from a set offset, from the file specified by the file handle. The data is read into a provided buffer. The full file is downloaded and read into memory, even if just a small part of the file is requested. The file is also cached so that subsequential requests for the same file are faster. 

\subsubsection{readdir}
This operation reads the filenames inside the directory specified by a file handle. The result includes all filenames in the directory, and the special \texttt{"."} and \texttt{".."} directories.

\subsubsection{write}
This operation writes $s$ bytes, starting at the provided offset $o$, to the existing file at the provided file handle. All the data of the current file is read in to memory. Starting from the offset, the new data overwrites the current data of the file, until $s$ bytes have been written. If $o + s$ is greater than the file's size, the file size is set to $o + s$. If $o + s$ is less than the file's size, the data from position $o + s$ and forward remains the same, and the file size is not modified. See Figure~\ref{fig:write_flow} for a visualization of the result of a \texttt{write} operation given different offsets. The parent directory does not have to modified. 

The file and inode table are not updated to the OWS, this occurs instead in the subsequent \texttt{release} call.

\begin{figure}[!ht]
	\begin{center}
	  \includegraphics[width=0.9\textwidth]{figures/write_flow.png}
	\end{center}
	\caption{Visualization of how the write operation handles different offsets.}
	\label{fig:write_flow}
\end{figure}

\subsubsection{rename}
This operation renames a file or directory to a new path. Both the old path and the new path have to be traversed to located the parent directories and the file or directory to rename. The file or directory entry in the old parent directory is removed, and the old parent directory is updated to the OWS. A new entry is created in the new parent directory, with the new filename. The new parent directory is updated to the OWS. The inode entry of the renamed file or directory does not have to be modified. However, as both the old parent directories and the new parent directory are updated in the OWS, their inode entries need to be updated with the new posts. The inode table is updated to the OWS and the old table is removed from the OWS. The old posts associated with the old parent directory and the new parent directory are removed from the OWS.

The new path could be in the same directory as the file or directory currently is in. This will not affect the process mentioned above, however the path will only have to be traversed once.

\subsubsection{truncate}
This operation truncates or extends the file in the given path, to the provided size $s$ . The full current file is downloaded into memory. The data of the current file is read into a new buffer until either the file is fully read, or until $s$ bytes have been read. If the current file's size is smaller than $s$, the remaining amount of bytes are added as the NULL character. The new file data is uploaded to the OWS, and the old data is removed from the OWS. The inode table entry is updated with the new posts and uploaded to the OWS. The old inode table is removed from the OWS.

\subsubsection{ftruncate}
This operation is similar to \texttt{truncate}, but is called from a user context which means it has a file handle associated with it. The operation truncates or extends the file in the given file handle, to the provided integer $s$. The full current file is downloaded into memory. The data of the current file is read into a new buffer until either the file is fully read, or until $s$ bytes have been read. If the current file's size is smaller than $s$, the remaining amount of bytes are added as the NULL character.

The file and inode table are not updated to the OWS, this occurs instead in the subsequent \texttt{release} call.

\subsubsection{unlink}
This operation removes a file given the filepath. The file is removed from the parent directory, and the parent directory is updated to the OWS. The old parent directory data is removed on the OWS. The removed file's entry in the inode table is also removed, and the inode table updates the entry for the parent directory with its new posts. The inode table is then updated on the OWS and the old inode table is removed on the OWS. Finally, the data of the removed file is removed from the OWS. The last step is not necessary for a working filesystem; however, to save space on the OWS, this is done. If the OWS permits unlimited images and sizes, this step could be omitted to execution save time.

\subsubsection{rmdir}
Similar to \texttt{unlink}, this operation removes the directory at the path. The directory and all its subdirectories are traversed, and the post IDs of these files and directories are recorded for deletion in the OWS later. Following, the entry of the removed directory is removed from the parent directory. The inode entry for the removed directory is removed. The parent directory is updated to the OWS, and the inode table is updated with the new posts of the parent directory. Following, the inode table is updated to the OWS. The old parent directory and the old inode table are removed from the OWS.

The operation also starts a new thread, where all the posts of files and subdirectories inside the removed directory, are removed from the OWS. This occurs to save space on the OWS, and a separate thread is used to save computation time for subsequent file operations. There is no data race involved as the API is thread safe, and the posts are no longer associated with any data structures on the main thread.

\subsubsection{getattr}
This operations returns attributes about a file or directory given a path. This includes permissions, number of entries (if the provided path points to a directory), and timestamps of creation, last access and last modification. However, as mentioned previously, FFS does not implement all features, such as permissions. Instead of keeping track of a file's or directory's permissions, all calls to valid path will return full read, write, and execute permissions for everyone. However, the timestamps are stored in the inode table of FFS. The file or directory pointed to by the path does not need to be downloaded, all the metadata that FFS stores is accessible through the inode entry in the inode table.

\subsubsection{fgetattr}
This operation is the similar to \texttt{getattr}, but is called from a user program context meaning that the file has a file handle associated with it. Other than skipping the path traverse step, this operation returns the equivalent information as \texttt{getattr}.

\subsubsection{statfs}
This operation returns metadata information about FFS. This includes, among other things, the maximum filename size and the filesystem ID. The operation has a short computation time as it does not have to download or upload any files. The only variable information is read from the inode table which is stored in memory and thus does not have to be downloaded from the OWS.

\subsubsection{access}
This operation, given a path, returns wether or not the path can be accessed. As long as the path is valid, this always returns that it can be accessed.

\subsubsection{utimens}
This operation updates the last access timestamp, the last modified timestamp, or both, of the file or directory at the given path. The file or directory does not have to be downloaded. However, the inode entry for the file's or directory's inode is updated with the new timestamps if they are newer than the previous timestamps but not greater than the current time since epoch. The new state of the inode table is updated to the OWS, and the old version is removed from the OWS.

% \subsection{Unimplemented filesystem operations}
% FIXME: Is this needed? ^ 

\subsection{FFS limitations}
\label{subsec:ffs_limits}
% TODO: Limitations, both in number of bytes, files etc
% Also what it lacks, such as filesystem operations and speed(!)
FFS has numerous of limitations due to both implementation decisions and OWS limits. As Flickr allows a free-tier user account to store up to 1\,000 images of up to \SI{200}{\mega\byte} per image, this allows storage of up to \SI{200}{\giga\byte} of images on per account on Flickr. However, as the inode table is required to be stored on the filesystem, a maximum of 999 images can be used to save file and directory data. This limits the filesystem to a maximum of 999 files and directories when utilizing one free-tier account on Flickr. 

While Flicker supports each image to be up to \SI{200}{\mega\byte}, it is not possible to use the full \SI{200}{\mega\byte} as the file data to store. The image includes, among other things, a PNG header, other PNG attributes, and the CED which in total is of greater size than the unencrypted data. To ensure that the pixel color data along with the PNG header and other PNG attributes does not exceed the limit of \SI{200}{\mega\byte}, FFS limits the pixel color data size to allow at least \SI{10}{\mega\byte} for the PNG header and other PNG attributes, meaning that the pixel color data can be a maximum of \SI{190}{\mega\byte}. The cryptographic variables IV, salt, and the authentication tag are stored in the CED using 12, 16, and 64 bytes respectively, for a total of 92 bytes. The size limit means that these 92 bytes, along with the encrypted cipher text, cannot exceed \SI{190}{\mega\byte}, meaning that the encrypted cipher text cannot exceed $190\,000\,000 - 92 = $\SI{189\,999\,908}{\byte}. However, as AES is a block cipher producing cipher blocks of 16 bytes, the resulting cipher text must be a divisible of 16. The largest encrypted cipher text that FFS allows is therefore $floor(\frac{189\,999\,906}{16})*16 = 189\,999\,904$ bytes. Due to plain text padding, the unencrypted plain text can be a maximum of one byte less than this value\,\cite{z.z.coderAnswerSizeData2010}, meaning that the plain text can be a maximum of \SI{189\,999\,903}{\byte}. For simplicity, this is rounded down to \SI{189}{\mega\byte}, leaving almost \SI{11}{\mega\byte} in total for the PNG header and other PNG attributes. \SI{189}{\mega\byte} is set as the maximum amount of data FFS will store per image. Data greater than \SI{189}{\mega\byte} is split into multiple encoded images. For instance, a file of \SI{200}{\mega\byte} will be stored as \SI{189}{\mega\byte} in one image, and \SI{11}{\mega\byte} in another. 

\SI{189}{\mega\byte} of usable data per images gives FFS a maximum storage capacity of \SI{188.811}{\giga\byte} using one free-tier account on Flickr. Each file with data requires at least one image, thus there can be a maximum of 998 non-empty files and directories in the filesystem, excluding the root directory. However, there could also be just one single file of \SI{188.811}{\giga\byte} stored in the filesystem.

The inode table also keeps information about empty files and directories even though they store no data on the OWS. The inode of a file or directory is an unsigned 32-bit integer, meaning that the inode table could theoretically store up to over four billion files and directories. However, due to the constraints mentioned above, most of these files and directories would have to be empty as Flickr limits the amount of images stored. An empty file requires \SI{37}{\byte} in the inode table. As the inode table is limited to one single image on the OWS, the inode table is limited to a maximum size of \SI{189}{\mega\byte}. Further, the size of the inode table is \SI{4}{\byte} plus the size of each entry, and one of these entries is the root directory. Even if a file is empty, it is still stored with its filename and inode in its parent directory. A non-empty directory in the inode table requires approximately (depending on the post ID length generated by the OWS) \SI{12}{\byte} per file or directories it contains. The maximum number of empty files and directories $X$ that the inode table can store is therefore, approximately:
$$
	X = ceil(\frac{189\,000\,000 - 4 - (12 * X)}{37}) + 1, X = 3\,857\,143
$$
The additional directory is the root directory. Thus, the maximum number of files and directories that the inode table can store is close to four million, however this requires all files and directories, except the root directory, to be empty. These calculations are based on a single free-tier Flickr account. However, a future expansion of FFS could include multiple user accounts, and multiple services. This could increase the limits on the filesystem.

Limits to the file sizes also depend on the machine where FFS is mounted. When a file is read or written to, the complete file is read into memory. This requires the computer to provide at least as much memory as the size of the file. However, even if the computer has less memory available, more memory can often be provided through memory swap on the hard disk. Apple ensures that the swapped data is securely encrypted on the hard disk\,\cite{appleinc.WhatSecureVirtual}. However, using memory swap puts a constraint on the storage of the computer to be sufficient. Also, as FFS temporarily saves the data on the local filesystem before it is uploaded to Flickr, the storage device must have sufficient storage available. For instance, a file larger than the available storage on the local filesystem cannot be saved to FFS. If the local filesystem has no available storage, very few filesystem operations can be performed on FFS as any operation that modify the inode table requires the new inode table to be saved to the local filesystem before it is uploaded to Flickr. 

A limitation of FFS that is not possible to quantify is the bandwidth and latency of the network connection from the user to Flickr. The connection can vary significantly depending on for instance the network load in a given moment and the geographic location of the user. A slow network connection is not something FFS can solve, but is left as an exercise for the reader.