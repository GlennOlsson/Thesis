\section{Filesystem benchmarking}
\label{sec:iozone}
IOzone is a filesystem benchmarking tool that is used to measure performance and analyze a filesystem\,\cite{IozoneFilesystemBenchmark}. It is built for, among other platforms, Apple's macOS where FFS will be built, run, and tested. However, as mentioned previously, filesystem benchmarking is more complicated than one might imagine. Different filesystems might perform differently on small and big file sizes among other things, which means that we can never compare benchmarking outputs as just single numbers. We must instead compare different aspects of the filesystems. In \citeyear{tarasovBenchmarkingFileSystem2011} \citeauthor{tarasovBenchmarkingFileSystem2011} presents a paper where they criticize several papers due to their lack of scientific and honest filesystem benchmarking\,\cite{tarasovBenchmarkingFileSystem2011}. The problem of benchmarking a filesystem is all the different components that are involved when interacting with a filesystem. For instance, they mention how benchmarking the in- and output (\gls{I/O}) of the filesystem, such as bandwidth and latency, is different from benchmarking on-disk operations, such as the performance of file read and write operations. The benchmarking tools can for instance rarely affect or determine how the filesystem handles caching and pre-fetching. This means that benchmarking the read and write performance of different filesystems can be misleading as they might handle this differently, meaning that the result could be different depending on for instance the distance between the files on the disk. Two files could be adjacent on the disk on one filesystem and therefore one could be pre-fetched into the cache when the other one is read. Considerations about such factors must be present when analyzing the results of the benchmarking.

\citeauthor{tarasovBenchmarkingFileSystem2011} also lists several different filesystem benchmarking tools available and used by the papers they reviewed, and how well the tools can analyze certain aspects of a filesystem\,\cite{tarasovBenchmarkingFileSystem2011}. IOZone is listed as being compatible with multiple of the different benchmarking types and as it is simpler to use\,\cite{agarwalComparingIOBenchmarks2018} and still maintained. Due to these factors, IOZone was chosen as the benchmarking tool for FFS.