\section{Related filesystems}
In 2007, \citeauthor{baliga2007web} presented an idea of a covert filesystem that hides the file data in images and uploads them to web services, named CovertFS\,\cite{baliga2007web}. The paper lacks implementation of the filesystem but they present an implementation plan which includes using FUSE. They limit the filesystem such that each image posted will only store a maximum of \SI{4}{\kilo\byte} of steganographic file data and the images posted on the web services will be actual images. This is different from the idea of FFS where the images will be purely the encrypted file data and will therefore not be an image that represents anything but will instead look like random color noise. An implementation of CovertFS has been attempted by \citeauthor{sosaSuperSecretFile2007} which also used Tor to further anonymize the users\,\cite{sosaSuperSecretFile2007}.

% TODO: Paragraph about tweetfs

In 2006, \citeauthor{jonesGoogleHackUse2006} creates GmailFS - a mountable filesystem that uses Google's Gmail to store the data\,\cite{jonesGoogleHackUse2006, jonesGmailFilesystemImplementation2006}. The filesystem was written in Python using FUSE and was presented well before the introduction of Google Drive in 2012. Today, Gmail and Google Drive share their storage quota and GmailFS has since its launch become redundant. GMail Drive is another example of a Gmail-based filesystem and it is influenced by GmailFS\,\cite{viksoeViksoeDkGMail2004}. GMail Drive has been declared dead by its author since 2015.

Timothy \citeauthor{petersDEFYDeniableFile2014} created a deniable filesystem using a log-based structure in \citeyear{petersDEFYDeniableFile2014}\cite{petersDEFYDeniableFile2014}. The filesystem of my project could be seen as a deniable system in the sense that the data is not stored on the device, and if the filesystem is not mounted it could be hard to prove that the user has access to the data, even if someone were to find the web service account. The deniable system developed by \citeauthor{petersDEFYDeniableFile2014} was developed using FUSE\,\cite{Libfuse2021} which we also will be using.

\citeauthor{badulescuCryptfsStackableVnode1998} created Cryptfs, a stackable Vnode filesystem that encrypted the underlying, potentially unencrypted, filesystem\,\cite{badulescuCryptfsStackableVnode1998}. By making the filesystem stackable, any layer can be added on top of any other, and the abstraction occurs by each Vnode layer communicating with the one beneath. There is a potential to further stack additional layers by using tools such as FiST\,\cite{FiSTStackableFile}. This approach enables one to create not only an encrypted file system but also to provide redundancy by replicating data to different
underlying file systems. If these file systems are independent, then potentially this increases availability and reliability.