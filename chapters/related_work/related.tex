\section{Related filesystems}
% FIXME: Chip: "The discussion of other FUSE based filesystems seems really minimal."
Multiple steganographic filesystems have been presented previously but many of these are focused on filesystems for physical storage disks that the user has access too. For instance, Timothy Peters created DEFY, a deniable filesystem using a log-based structure in 2014\,\cite{petersDEFYDeniableFile2014}. DEFY was built to be used exclusively on Solid State Drives (SSD) found in mobile devices to provide a steganographic filesystem which could be used on Android phones. Further examples of local disk-based filesystems can be found in \cite{andersonSteganographicFileSystem1998, mcdonaldStegFSSteganographicFile2000, domingo-ferrerSharedSteganographicFile2008, hanMultiuserSteganographicFile2010}, among other papers. However, this paper aims to create a filesystem which is not based on a physical disk but rather a cloud-based steganographic filesystem which uses online web services as its storage medium. 

In 2007, \citeauthor{baliga2007web} presented an idea of a covert filesystem that hides the file data in images and uploads them to web services, named CovertFS\,\cite{baliga2007web}. The paper lacks implementation of the filesystem but they present an implementation plan which includes using FUSE. They limit the filesystem such that each image posted will only store a maximum of \SI{4}{\kilo\byte} of steganographic file data and the images posted on the web services will be actual images. This is different from the idea of FFS where the images will be purely the encrypted file data and will therefore not be an image that represents anything but will instead look like random color noise. An implementation of CovertFS has been attempted by \citeauthor{sosaSuperSecretFile2007} which also used Tor to further anonymize the users\,\cite{sosaSuperSecretFile2007}.

In \citeyear{szczypiorskiStegHashNewMethod2016}, \citeauthor{szczypiorskiStegHashNewMethod2016} introduced the idea of StegHash - a way to hide steganographic data on Open Social Networks (OSN) by connecting multimedia files, such as images and videos, with hashtags\,\cite{szczypiorskiStegHashNewMethod2016}. Specifically, images were posted to Twitter and Instagram along with certain permutations of hashtags which pointed to other posts through the use of a custom designed secret transition generator. StegHash managed to store short messages with 10 bytes of hidden data with a 100\% success rate, while longer messages with up to 400 bytes of hidden data had a success rate of 80\%. \citeauthor{bieniaszSocialStegDiscApplicationSteganography2017} later presented SocialStegDisc which was a filesystem application of the idea presented with StegHash\,\cite{bieniaszSocialStegDiscApplicationSteganography2017}. Multiple posts could be required to store a single file and each post referenced the next post like a linked list, which means that you only need the root post to read all the data. This is unlike the idea of FFS where a table will be kept to keep track of which posts store a certain file, and in what order they should be concatenated.

TweetFS is a filesystem created by Robert Winslow which stores the data on Twitter\,\cite{winslowTweetfsTweetfsMaster}, created in 2011. It was created as a proof of concept to show that it is possible to store filedata on Twitter. The filesystem uses sequential text posts to store the data. The filesystem is not mounted to the operating system but instead the user interacts with a Python script  through the command line. This makes the filesystem less convenient from a user perspective, compared to a mounted filesystem where the files can be browsed using a user interface or command line. There are two commands available: \texttt{upload} and \texttt{download} which uploads and downloads files or directories, respectively. Names and permissions of files and directories are maintained throughout the upload and download process.

In 2006, \citeauthor{jonesGoogleHackUse2006} created GmailFS - a mountable filesystem that uses Google's Gmail to store the data\,\cite{jonesGoogleHackUse2006, jonesGmailFilesystemImplementation2006}. The filesystem was written in Python using FUSE and was presented well before the introduction of Google Drive in 2012. Today, Gmail and Google Drive share their storage quota and GmailFS has since become redundant. GMail Drive is another example of a Gmail-based filesystem and it was influenced by GmailFS\,\cite{viksoeViksoeDkGMail2004}. GMail Drive has been declared dead by its author since 2015.

\citeauthor{badulescuCryptfsStackableVnode1998} created Cryptfs, a stackable Vnode filesystem that encrypted the underlying, potentially unencrypted, filesystem\,\cite{badulescuCryptfsStackableVnode1998}. By making the filesystem stackable, any layer can be added on top of any other, and the abstraction occurs by each Vnode layer communicating with the one beneath. There is a potential to further stack additional layers by using tools such as FiST\,\cite{FiSTStackableFile}. This approach enables one to create not only an encrypted file system but also to provide redundancy by replicating data to different
underlying filesystems. If these filesystems are independent, then potentially this increases availability and reliability. The filesystem developed as a result of this thesis aims to achieve stackability through the use of FUSE. 
% FIXME: Chip: "While you mention CryptFS and FiST you do not say why you do not build your systems to use a stackable filesystem (as this would make adding redundancy much easier)"
% 	Stackable filesystem - kernel module between VFS and the actual filesystem (eg ext4)
%		According to "To fuse or not to fuse..." they build a stackable filesystem in FUSE
%			Maybe using the kernel API??
% 		Read 2.1 in this article, seems appropriate


% Create table comparing these fs and FFS