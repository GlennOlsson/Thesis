\section{Steganography and deniable filesystems}
\label{sec:SteganographyAndDeniableFS}
Steganography is the art of hiding information in plain sight and has been around for ages. For example today, a major part of steganography is hiding malicious code in images, called stegomalware or stegoware. Stegomalware is an increasing problem and in a sample set of \mbox{real-life} stegomalware, over 40\% of the cases used images to store the malicious code\,\cite{stichtingcuingfoundationSIMARGLStegwarePrimer2020}.While \gls{FFS} will not include malicious code in its images, the problem of stegomalware  has fostered the development of steganography detection techniques in (for instance) social media platforms. 

Twitter has been exposed for allowing steganographic images that can easily contain any type of file\,\cite{TwitterImagesCan}. David Buchanan created a simple python script of only \num{100} lines of code that can encode zip files, mp3 files, and any imaginable file in an image of the user's choosing\,\cite{buchananTweetablepolyglotpng2022}. He presents multiple examples of this technique on his Twitter profile\footnote{\url{https://twitter.com/David3141593}}. The fact that his images are public might be evidence that Twitter's steganography detection software is imperfect. However, it is also possible that Twitter has explicitly chosen to not remove these posts.

Other examples of steganographic data storage on \glspl{OSN} include the paper by \citeauthor{ningSecretMessageSharing2014} where the authors build a system for private communication on top of public \mbox{photo-sharing} web services\,\cite{ningSecretMessageSharing2014}. Due to the web services processing of uploaded multimedia, the authors first researched how the integrity of steganographic data could be maintained after being uploaded to these services. Following this, they presented an approach that ensured the integrity of the hidden messages in the uploaded images, while maintaining a low likelihood of discovery from steganographic analysis. \citeauthor{beatoUndetectableCommunicationOnline2014} also explored the idea of undetectable communications over \glspl{OSN}\,\cite{beatoUndetectableCommunicationOnline2014}. While they did not carry out an implementation, they present an idea where messages are encoded together with a cover object and a cryptographic key to produce a steganographic message which is then posted to \glspl{OSN}. They presented a \mbox{web-based} user interface client with a \gls{PHP} server backend as the method users would use to create and share their secret messages.

A deniable filesystem is a system that does not expose files stored on this system without credentials, \ie not providing information about how many files are stored, their sizes, their content, or even if there exist any files in the filesystem\,\cite{andersonSteganographicFileSystem1998}. A \mbox{rubber-hose} filesystem is a filesystem where if an adversary is only given one key out of $n$ keys in total, they cannot prove that more data exists and the filesystem is therefore deniable. This is known as a \mbox{rubber-hose} filesystem because of the idea behind \mbox{rubber-hose} cryptanalysis where an adversary beats the user with a \mbox{rubber-hose} to extract the encryption key. The adversary should have no way of knowing how many keys are used to encrypt the data. Steganographic methods can be used to hide the data for a deniable filesystem, and some deniable filesystems are also steganographic filesystems. However, deniability can be accomplished using techniques other than steganography. This thesis proposes a deniable filesystem that, while storing the data in images, is not steganographic as it does \textbf{not} hide the data in the images, but rather simply uses these images as the storage medium.