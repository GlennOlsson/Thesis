\section{Steganography and deniable filesystems}
Steganography is the art of hiding information in plain sight and has been around for ages. Today, a major part of steganography is hiding malicious code in for instance images, called stegomalware or stegoware. Stegomalware is an increasing problem and in a sample set of examined \mbox{real-life} stegomalware, over 40\% of the cases used images to store the malicious code\,\cite{stichtingcuingfoundationSIMARGLStegwarePrimer2020}. While \gls{FFS} will not include malicious code in its images, this stegomalware problem has fostered the development of detection techniques of steganography in for instance social media, and it is \mbox{well-researched}. 

Twitter has been exposed to allowing steganographic images that can easily contain any type of file\,\cite{TwitterImagesCan}. David Buchanan created a simple python script of only $100$ lines of code that can encode zip files, mp3 files, and any file imaginable in an image of the user's choosing\,\cite{buchananTweetablepolyglotpng2022}. He presents multiple examples of this technique on his Twitter profile\footnote{\url{https://twitter.com/David3141593}}. The fact that the images are available publicly might be evidence that Twitter's steganography detection software is not perfect. However, it is also possible that Twitter has chosen to not remove these posts.

Other examples of steganographic data storage on \gls{OSNs} include the paper by \citeauthor{ningSecretMessageSharing2014} where the authors build a system for private communication on public \mbox{photo-sharing} web services\,\cite{ningSecretMessageSharing2014}. Due to the web services processing of uploaded multimedia, they first researched how the integrity of steganographic data could be maintained after being uploaded to these services. Following this, they presented an approach that ensured the integrity of the hidden messages in the uploaded images, while also maintaining a low likelihood of discovery from the steganographic analysis. \citeauthor{beatoUndetectableCommunicationOnline2014} also explored the idea of undetectable communications over \gls{OSNs}\,\cite{beatoUndetectableCommunicationOnline2014}. While they did not carry out an implementation, they present an idea where messages are encoded together with a cover object and a cryptographic key to produce a steganographic message which is then posted to the \gls{OSNs}. They presented a \mbox{web-based} user interface client with a PHP server backend as the method the users would use to create and share their secret messages.

\label{sec:rubber_hose}
A deniable filesystem is a system that does not expose files stored on this system without credentials - neither how many files are stored, their sizes, their content, or even if there exist any files in the filesystem\,\cite{andersonSteganographicFileSystem1998}. This is also known as a rubber hose filesystem because of the characteristic that the data only can be proven to exist with the correct encryption key which only is accessible if the person is tortured and beaten with a rubber hose because of its simplicity and immediacy compared to the complexity of breaking the key by computational techniques. Steganographic methods can be used to hide the data for a deniable filesystem, and some deniable filesystems are also steganographic filesystems. However, deniability can be accomplished using other techniques than steganography. This thesis proposes a deniable filesystem that, while storing the data in images, is not steganographic as it does not hide the data in the images, but rather use the images as the storage medium.