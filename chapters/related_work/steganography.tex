\section{Steganography and deniable filesystems}
Steganography is the art of hiding information in plain sight, and has been around for ages. Today, a major part of steganography is hiding malicious code in for instance images, called stegomalware. Stegomalware is an increasing problem and in a sample set of images over 40\% of real-life stegomalware was found in digital images\cite{SIMARGLStegwarePrimer}. While FFS does will not include malicious code in its images, this stegomalware problem has fostered the development of detection techniques of steganography in for instance social media, and it is well researched. 

Twitter has been exposed to allowing steganographic images that contains any type of file easily\cite{TwitterImagesCan}. David Buchanan created a simple python script of only 100 rows of code that can encode zip-files, mp3-files, and really any file imaginable in an image of the user's choosing\cite{buchananTweetablepolyglotpng2022}. He presents multiple examples of this technique on his Twitter profile\footnote{\url{https://twitter.com/David3141593}}. The fact that the images available for the public's eye is evidence that Twitter's steganography detection software might not be perfect. However, it is also possible that Twitter has chosen to not remove these posts.

A steganographic, or deniable, filesystem is a system that does not expose files stored on this system without credentials - neither how many files are stored, their sizes, their content, or even if there exist any files on the filesystem\cite{petersDEFYDeniableFile2014}. This is also known as a rubberhose filesystem because of the characteristic that the data really only can be proven to exist with the correct encryption key which only is accessible if the person is tortured and beaten with a rubber-hose because of its simplicity and immediacy compared to the complexity of breaking the key by computational techniques.

% TODO: Mention spam bots and that detection of that is a problem not solved, but worked upon