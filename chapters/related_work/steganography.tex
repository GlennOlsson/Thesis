\section{Steganography and deniable filesystems}
Steganography is the art of hiding information in plain sight and has been around for ages. Today, a major part of steganography is hiding malicious code in for instance images, called stegomalware or stegoware. Stegomalware is an increasing problem and in a sample set of examined real-life stegomalware, over 40\% of the cases used images to store the malicious code\,\cite{stichtingcuingfoundationSIMARGLStegwarePrimer2020}. While FFS does will not include malicious code in its images, this stegomalware problem has fostered the development of detection techniques of steganography in for instance social media, and it is well researched. 

Twitter has been exposed to allowing steganographic images that contain any type of file easily\,\cite{TwitterImagesCan}. David Buchanan created a simple python script of only $100$ lines of code that can encode zip-files, mp3-files, and any file imaginable in an image of the user's choosing\,\cite{buchananTweetablepolyglotpng2022}. He presents multiple examples of this technique on his Twitter profile\footnote{\url{https://twitter.com/David3141593}}. The fact that the images are available for the public's eye might be evidence that Twitter's steganography detection software is not perfect. However, it is also possible that Twitter has chosen to not remove these posts.

Other examples of steganographic data storage on Open Social Networks (OSNs) include the paper presented by \citeauthor{ningSecretMessageSharing2014} where the authors build a system for private communication on public photo-sharing web services\,\cite{ningSecretMessageSharing2014}. Due to the web services processing of uploaded multimedia, they first researched how the integrity of steganographic data could be maintained after being uploaded to these services. Following, they presented an approach which ensured the integrity of the hidden messages in the uploaded images, while also maintaining a low-likelihood of discovery from steganographic analysis. \citeauthor{beatoUndetectableCommunicationOnline2014} also explores the idea of undetectable communications over OSNs in another paper\,\cite{beatoUndetectableCommunicationOnline2014}. While implementation is not carried out, they present an idea where messages are encoded together with a cover object and a cryptographic key to produce a steganographic message which is then posted to the OSN. A web-based user interface client with a PHP server backend is presented as the method the users would use to create and share their secret messages.

A steganographic, or deniable, filesystem is a system that does not expose files stored on this system without credentials - neither how many files are stored, their sizes, their content, or even if there exist any files in the filesystem\,\cite{andersonSteganographicFileSystem1998}. This is also known as a rubber hose filesystem because of the characteristic that the data really only can be proven to exist with the correct encryption key which only is accessible if the person is tortured and beaten with a rubber hose because of its simplicity and immediacy compared to the complexity of breaking the key by computational techniques.

% TODO: Mention spam bots and that detection of that is a problem not solved, but worked upon