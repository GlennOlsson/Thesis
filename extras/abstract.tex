% \todo[inline, backgroundcolor=kth-lightgreen]{All theses at KTH are \textbf{required} to have an abstract in both \textit{English} and \textit{Swedish}.}

% \todo[inline, backgroundcolor=kth-lightgreen]{GLENNE HARRIE OLLESSON Exchange students many want to include one or more abstracts in the language(s) used in their home institutions to avoid the need to write another thesis when returning to their home institution.}

% \todo[inline]{Keep in mind that most of your potential readers are only going to read your \texttt{title} and \texttt{abstract}. This is why it is important that the abstract give them enough information that they can decide is this document relevant to them or not. Otherwise the likely default choice is to ignore the rest of your document.\\

% A abstract should stand on its own, i.e., no citations, cross references to the body of the document, acronyms must be spelled out, \ldots .\\

% Write this early and revise as necessary. This will help keep you focused on what you are trying to do.}

Many \gls{OWS}s today, such as Flickr and Twitter, provide users with the possibility to post images which are stored on the platform for free. This thesis explores the idea of creating a cryptographically secure filesystem which stores its data on an online web service using encoded and encrypted images. More data can usually be stored in image posts than in a text posts on \gls{OWS}s. The filesystem, named \gls{FFS}, provides users with free, deniable, and cryptographic storage by exploiting the storage provided by these online web services. The thesis compares the performance of \gls{FFS} against two other filesystems. It can be concluded that \gls{FFS} has limitations in mainly speed, making it unviable as a \mbox{general-purpose} usage, such as a substitute to the local filesystem on a computer. However, its portability and security makes it relevant for certain scenarios.
% TODO: Add when it could be relevant
% TODO: Add more about environmental etc.