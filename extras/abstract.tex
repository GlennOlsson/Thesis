% \todo[inline, backgroundcolor=kth-lightgreen]{All theses at KTH are \textbf{required} to have an abstract in both \textit{English} and \textit{Swedish}.}

% \todo[inline, backgroundcolor=kth-lightgreen]{GLENNE HARRIE OLLESSON Exchange students many want to include one or more abstracts in the language(s) used in their home institutions to avoid the need to write another thesis when returning to their home institution.}

% \todo[inline]{Keep in mind that most of your potential readers are only going to read your \texttt{title} and \texttt{abstract}. This is why it is important that the abstract give them enough information that they can decide is this document relevant to them or not. Otherwise the likely default choice is to ignore the rest of your document.\\

% A abstract should stand on its own, i.e., no citations, cross references to the body of the document, acronyms must be spelled out, \ldots .\\

% Write this early and revise as necessary. This will help keep you focused on what you are trying to do.}
Today there are free online services that can be used to store files of arbitrary types and sizes, such as Google Drive. However, these services are often limited by a certain total storage size. The goal of this thesis is to create a filesystem that can store arbitrary amount and types of data i.e. without any real limit to the storage size. This is to be achieved by taking advantage of online webpages, such as Twitter, where text and files can be posted on free accounts with no apparent limit on storage size. The aim is to have a filesystem that behaves similar to any other filesystem but where the actual data is stored for free on various websites.