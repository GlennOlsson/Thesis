% \todo[inline, backgroundcolor=kth-lightblue]{Alla avhandlingar vid KTH \textbf{måste ha} ett abstrakt på både \textit{engelska} och \textit{svenska}.\\
% Om du skriver din avhandling på svenska ska detta göras först (och placera det som det första abstraktet) - och du bör revidera det vid behov.}

% \todo[inline]{If you are writing your thesis in English, you can leave this until the draft version that goes to your opponent for the written opposition. In this way you can provide the English and Swedish abstract/summary information that can be used in the announcement for your oral presentation.\\

% If you are writing your thesis in English, then this section can be a summary targeted at a more general reader. However, if you are writing your thesis in Swedish, then the reverse is true – your abstract should be for your target audience, while an English summary can be written targeted at a more general audience.\\

% This means that the English abstract and Swedish sammnfattning  
% or Swedish abstract and English summary need not be literal translations of each other.
% }
% \todo[inline, backgroundcolor=kth-lightred]{Do not use the \textbackslash glspl\{\} macro in an abstract that is not in English, as my programs do not know how to generate plurals in other languages. Instead you will need to spell these terms out or give the proper plural form.}

% \todo[inline, backgroundcolor=kth-lightgreen]{The abstract in the language used for the thesis should be the first abstract, while the Summary/Sammanfattning in the other language can follow}


% Many \gls{OWS}s today, such as Flickr and Twitter, provide users with the possibility to post images which are stored on the platform for free. This thesis explores the idea of creating a cryptographically secure filesystem which stores its data on an online web service using encoded and encrypted images. More data can usually be stored in image posts than in a text posts on \gls{OWS}s. The filesystem, named \gls{FFS}, provides users with free, deniable, and cryptographic storage by exploiting the storage provided by these online web services. The thesis analyzes and compares the performance of \gls{FFS} against two other filesystems and a version of \gls{FFS} that does not use an \gls{OWS}. It can be concluded that \gls{FFS} has limitations in performance, making it unviable as a \mbox{general-purpose} filesytem, such as a substitute to the local filesystem on a computer. However, it provides security benefits compared to other \mbox{cloud-based} filesystems such as \mbox{end-to-end} encryption, authenticated encryption, and plausible deniability of the data. Furthermore, being a \mbox{cloud-based} filesystem, \gls{FFS} can be mounted on any computer with the same operating system, given the correct secrets. 

Flertalet webbtjänster idag, så som Flickr och Twitter, tillhandahåller möjligheten att lägga upp bilder som lagras på tjänsten utan kostnad. Denna avhandling utforskar en ide om att skapa ett kryptografiskt säkert filsystem som lagrar sin data på webbtjänster med hjälp av kodade och krypterade bilder. Mer data kan ofta lagras i bildinlägg än i textinlägg på webbtjänsterna. Filsystemet, döpt Fejk Filesystem (FFS), tillhandahåller användare med gratis, förnekningsbar, och kryptografisk lagring genom att utnyttja det lagringsutrymme som tillhandahålls av dessa webbtjänster. Avhandlingen analyserar och jämför prestandan hos FFS med två andra filsystem, samt en version av FFS som inte använder en webbtjänst. En slutsats som kan dras är att FFS har begränsningar i prestanda, vilket gör det ohållbart som ett generellt-ändamålsfilssystem, till exempel som ersättare för det lokala filsystemet på en dator. Däremot tillhandahåller det säkerhetsförmåner jämfört med andra molnbaserade filsystem, så som början-till-slutetkryptering, autentiserad kryptering, samt sannolik förnekningsbarhet av datan. Dessutom, som ett molnbaserat filsystem kan FFS monteras på vilken dator som helst med samma operativsystem, givet de korrekta nycklarna. 