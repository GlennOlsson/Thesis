%%%%%%%%%%%%%%%%%%%%%%%%%%%%%% Packages %%%%%%%%%%%%%%%%%%%%%%%%%%%%%%
% The following is for use with the KTH cover when not using XeLaTeX or LuaLaTeX
%\ifxeorlua\relax
%\else
\usepackage[scaled]{helvet}
%\fi

%% The following are needed for generating the DiVA page(s)
\usepackage[force-eol=true]{scontents}              %% Needed to save lang, abstract, and keywords
\usepackage{pgffor}                 %% includes the foreach loop

%% Basic packages

%% Links
\usepackage{url}                %% Support for breaking URLs

%% Colorize
%\usepackage{color}
\PassOptionsToPackage{dvipsnames, svgnames}{xcolor}
\usepackage{xcolor}

\usepackage[normalem]{ulem}
\usepackage{soul}
\usepackage{xspace}
\usepackage{braket}

% to support units and decimal aligned columns in tables
\usepackage[locale=US]{siunitx}

\usepackage{balance}
\usepackage{stmaryrd}
\usepackage{booktabs}
\usepackage{graphicx}	        %% Support for images
\usepackage{multirow}	        %% Support for multirow columns in tables
\usepackage{tabularx}		    %% For simple table stretching
\usepackage{mathtools}
\usepackage{algorithm} 
\usepackage{algorithmic}  
\usepackage{amsmath}
\usepackage[linesnumbered,ruled,vlined,algo2e]{algorithm2e}
% can't use both algpseudocode and algorithmic packages
%\usepackage[noend]{algpseudocode}
%\usepackage{subfig}  %% cannot use both subcaption and subfig packages
\usepackage{optidef}
\usepackage{float}		        %% Support for more flexible floating box positioning
\usepackage{pifont}

%% some additional useful packages
% to enable rotated figures
\usepackage{rotating}	    	%% For text rotating
\usepackage{array}		        %% For table wrapping
\usepackage{mdwlist}            %% various \mbox{list-related} commands
\usepackage{setspace}           %% For \mbox{fine-grained} control over line spacing


\usepackage{enumitem}           %% to allow changes to the margins of descriptions


%% If you are going to include source code (or code snippets)
\usepackage{listings}		    %% For source code listing
%%\usepackage[cache=false]{minted} %% For source code highlighting
%%\usemintedstyle{borland}

\usepackage{bytefield}          %% For packet drawings


\setlength {\marginparwidth }{2cm} %leave some extra space for todo notes
\usepackage{todonotes}
\usepackage{notoccite} % do not number captions based on their appearance in the TOC


% Footnotes
\usepackage{perpage}
\usepackage[perpage,para,symbol]{footmisc} %% use symbols to ``number'' footnotes and reset which symbol is used first on each page


%% Various useful packages
%%----------------------------------------------------------------------------
%%   pcap2tex stuff
%%----------------------------------------------------------------------------
\usepackage{tikz}
\usepackage{colortbl}
\usetikzlibrary{arrows,decorations.pathmorphing,backgrounds,fit,positioning,calc,shapes}
\usepackage{pgfmath}	% --math engine
\newcommand\bmmax{2}
\usepackage{bm} % bold math


% \documentclass{beamer}
\usepackage{enumitem}
\setlist[itemize]{noitemsep, nolistsep}
\setlist[enumerate]{noitemsep, nolistsep}

%% Managing titles
% \usepackage[outermarks]{titlesec}
%%%%%%%%%%%%%%%%%%%%%%%%%%%%%%%%%%%%%%%%%%%%%%%%%%%%%%%%%%%%%%%%%%%%%%
%\captionsetup[subfloat]{listofformat=parens}

% to include PDF pages
%\usepackage{pdfpages}


\usepackage{csquotes}               %% Recommended by biblatex
% to provide a float barrier use:
\usepackage{placeins}

\usepackage{comment}  %% Provides a comment environment
\usepackage{refcount}   %% to be able to get an expandable \getpagerefnumber

% for experiments with new cover
\usepackage{eso-pic}
\usepackage[absolute,overlay]{textpos}

% when the package is used, it draws boxes on the page showing the text, footnote, header, and margin regions of the page
%\usepackage{showframe}  
%\usepackage{printlen} % defines the printlength command to print out values of latex variable

\usepackage{bytefield}

% \usepackage[section]{placeins}

\usepackage{needspace}

\newcommand{\floor}[1]{\lfloor #1 \rfloor}
\newcommand{\ceil}[1]{\lceil #1 \rceil}

\DeclareSourcemap{
	\maps[datatype=bibtex]{
		\map{
			\step[fieldset=month, null]
			\step[fieldsource=date, match=\regexp{([0-9]{4})-([0-9]{2})(-[0-9]{2})?}, replace={$1}]
		}
	}
}