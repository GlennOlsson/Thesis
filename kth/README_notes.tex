\documentclass{article}

% Language setting
\usepackage[english]{babel}

% Set page size and margins
\usepackage[a4paper,top=2cm,bottom=2cm,left=3cm,right=3cm,marginparwidth=1.75cm]{geometry}

% Useful packages
\usepackage{enumitem}

\usepackage[colorlinks=true, allcolors=blue]{hyperref}
\usepackage{cleveref} % Must be after the hyperref package
 
\title{README and notes about the template}
\author{Gerald Q. Maguire Jr.}

\begin{document}
\maketitle

\begin{abstract}
This document describes the thesis template that I have developed for use at KTH Royal Institute of Technology and provides some background about why it is the way that it is. It is important to note that the template is \textbf{not prescriptive}, as not every thesis will have all of the parts that the template shows. However, if there is something that you decide to leave out, you should make a conscious decision to do so and you should consider the impact this may have on your thesis being approved by the examiner.

Fundamental to the design of the template are several key factors:
\begin{itemize}
    \item Helping students be successful in their degree project,
    \item Helping students produce a high-quality thesis,
    \item There are several thousand theses written each year by KTH students.
    \item Every approved thesis will be entered into DiVA (independent of whether the full text is made available via DiVA), and
    \item Supporting all of the (relevant) phases of the degree project process.
\end{itemize}

This template is \textbf{not} designed for use by Interactive Media Technology (TIMTM) and Media Management (TMMTM) students - as students in these two programmes are using a different structure for their reports (there is another template available for them).

This document is a work in progress.
\end{abstract}

\section{Introduction}
This template evolved (radically) from an earlier thesis template and the direction of this evolution was based on the earlier DOCX template that was developed over many years for use with students for whom I was the examiner and/or supervisor. The suggested structure and contents of the thesis reflect my experience as an examiner for more that 570 degree projects and the experience I have had as a teacher and examiner for the course II2202 Research Methodology and Scientific Writing. The template reflects my interest as a member of KTH's Language Committee on facilitating the parallel use of English and Swedish at KTH, as well as supporting other languages. The latter aspects reflects my experience with double degree students, who often need to have at least the abstract of their thesis available in the language(s) of their home university. The thesis also reflect my experience in entering the metadata for hundreds of theses into DiVA and announcing a very large number of degree project seminars.

\Cref{sec:expectedUsers} describes a number of different groups of users and how the template is relevant to them.

There were several major thoughts influencing the design of this template:
\begin{enumerate}[leftmargin=*, label=\textbf{Thought \arabic*}, ref={Thought \arabic*}]
    \item \label{thought:helpStudent} The template should help a student be successful in their degree project and help them produce a high-quality thesis in conjunction with their degree project.
    
    \item \label{thought:inDiVA} Every approved thesis will have at least its meta data entered into DiVA.
    
    \item \label{thought:reducingDataEntry} Redundant data entry should be minimized in order to increase consistency.
    
    \item \label{thought:volume} There are several thousand theses each year. In fact, theses are the 2\textsuperscript{nd} most common type of publication at KTH.
    
    \item \label{thought:process} The template should help support all of the (relevant) phases of the degree project process.
\end{enumerate}

\section{Deliminations}

This template is \textbf{not} designed for use by Interactive Media Technology (TIMTM) and Media Management (TMMTM) students - as students in these two programmes are using a different structure for their reports (there is another template available for them).

Additionally, I have been told by one of my colleagues in applied mathematics that theses in this areas generally do not follow the IMRAD structure.

\section{Expected users and their differences}
\label{sec:expectedUsers}
There are four different sets of users who are relevant to this template:
\begin{enumerate}[leftmargin=*,label=\textbf{Users \arabic*}, ref={Users \arabic*}]
    \item \label{users:authors} Author or Authors (see \Cref{sec:authors}),
    \item \label{users:others} Those working together with the author(s) during the degree project process (see \Cref{sec:examinerAdvisorsOpponent}),
    \item \label{users:admins} Administrative staff working with the document after it has been approved by the examiner (see \Cref{sec:adminStaff}), and
    \item \label{users:readers} The (hopefully) many (human) readers of the final document (see \Cref{sec:readers}).
    \item \label{users:searchEngines} The (hopefully) many computers reading the metadata and the full text of the final document (see \Cref{sec:searchEngines}).
\end{enumerate}

Each of these different sets of users have different needs and perspectives. The following subsections describe these needs and perspectives.

\subsection{Author or Authors}
\label{sec:authors}
One of the hardest problems an author faces is getting started writing, i.e., the blank sheet of paper - empty file barrier. The template provides an non-blank starting point. Additionally, the template provide some initial structure, basically an  Introduction, Methods, Results, and Discussion (IMRAD) structure, so that there are hints of were to place material. Moreover, there are places (and notes) about material that the student should consider adding; for example, the ``required reflections'' section in the final chapter.

The template also provides some examples of commonly occurring types of content, so that one can easily find examples of how to include a figure, table, code listing, etc. These examples are not meant to be exhaustive and quite often the student will probably need to learn new \LaTeX\ commands in the course of writing their thesis.

\subsection{Those working in parallel with the authors(s) during the degree project}
\label{sec:examinerAdvisorsOpponent}

\subsection{Administrative staff}
\label{sec:adminStaff}

\subsection{(Human) Readers of the thesis}
\label{sec:readers}

\subsection{Machines reading the metadata or full text of the thesis}
\label{sec:searchEngines}
 

%\bibliographystyle{alpha}
%bibliography{sample}

\end{document}