
%%
%% forked from https://gits-15.sys.kth.se/giampi/kthlatex kthlatex-0.2rc4 on 2020-02-13
%% expanded upon by Gerald Q. Maguire Jr.
%% This template has been adapted by Anders Sjögren to the University
%% Engineering Program in Computer Science at KTH ICT. Adaptation is the
%% translation of English headings into Swedish as the addition of Swedish
%% text. Original body text is deliberately left in English.

% Make it possible to conditionally depend on the TeX engine used
\RequirePackage{ifxetex}
\RequirePackage{ifluatex}
\newif\ifxeorlua
\ifxetex\xeorluatrue\fi
\ifluatex\xeorluatrue\fi

%% Conventions for todo notes:
% \todo[inline]{Comments/directions/... in English}
% \todo[inline, backgroundcolor=kth-lightblue]{Text på svenska}
% \todo[inline, backgroundcolor=kth-lightgreen]{English descriptions about formatting}

%% The template is designed to handle a thesis in English or Swedish
% set the default language to english or swedish by passing an option to the documentclass - this handles the inside tile page
% To optimize for digital output (this changes the color palette add the option: digitaloutput
% To use bibtex or biblatex - include one of these as an option
%\documentclass[english, bibtex]{templates/kth/kththesis}
\documentclass[english, biblatex]{templates/kth/kththesis}

% \usepackage[style=numeric,sorting=none,backend=biber]{biblatex}
\ifbiblatex
    %\usepackage[language=english,bibstyle=authoryear,citestyle=authoryear, maxbibnames=99]{biblatex}
    %  \usepackage[natbib=true,bibstyle=authoryear,citestyle=authoryear, maxbibnames=99,language=english,backend=bibtex]{biblatex}
    %% GQMJr the min and max were backward
    \usepackage[natbib,maxcitenames=11,mincitenames=3,style=ieee,citestyle=numeric-comp,sorting=none,backend=biber,maxbibnames=99]{biblatex}
    \addbibresource{bibliography.bib}
    \addbibresource{additional_bib_entries.bib}
    %\DeclareLanguageMapping{norsk}{norwegian}
\else
    % The line(s) below are for BibTeX
    \bibliographystyle{bibstyle/myIEEEtran}
    %\bibliographystyle{apalike}
\fi


% include a variety of packages that are useful
\input{extras/includes}
\usepackage[perpage,para,symbol]{footmisc}
\input{templates/kth/lib/kthcolors}

% \glsdisablehyper
% \makeglossaries 
% \makenoidxglossaries
% %%% Local Variables:
%%% mode: latex
%%% TeX-master: t
%%% End:

% The form of the entries in this file is \newacronym{label}{acronym}{phrase}
%                                      or \newacronym[options]{label}{acronym}{phrase}
% see "User Manual for glossaries.sty" for the  details about the options, one example is shown below
% note the specification of the long form plural in the line below
% \newacronym[longplural={Debugging Information Entities}]{DIE}{DIE}{Debugging Information Entity}
% %
% % The following example also uses options
% \newacronym[plural={OSes}, firstplural={operating systems (OSes)}]{OS}{OS}{operating system}

% % note the use of a non-breaking dash in long text for the following acronym
% \newacronym{IQL}{IQL}{Independent Q‑Learning}

% \newacronym{LAN}{LAN}{Local Area Network}
% % note the use of a non-breaking dash in the following acronym
% \newacronym{WiFi}{Wi-Fi}{Wireless Fidelity}

% \newacronym{WLAN}{WLAN}{Wireless Local Area Network}
% \newacronym{UN}{UN}{United Nations}
% \newacronym{SDG}{SDG}{Sustainable Development Goal}

\newacronym{FFS}{FFS}{Fejk Filesystem}
\newacronym{FFFS}{FFFS}{Fejk Fejk Filesystem}
\newacronym{OWS}{OWS}{Online Web Service}
\newacronym{FOWS}{FOWS}{Fake Online Web Service}
\newacronym{CED}{CED}{Complete Encrypted Data}
\newacronym{LCED}{LCED}{Length of Complete Encrypted Data}
\newacronym{FUSE}{FUSE}{Filesystem in Userspace}
\newacronym{AES}{AES}{Advanced Encryption Standard}
\newacronym{GCM}{GCM}{Galois/Counter Mode}
\newacronym{PCD}{PCD}{Pixel Color Data}
\newacronym{RGB}{RGB}{Red Green Blue}
\newacronym{HMAC}{HMAC}{Hashed Message Authentication Code}
\newacronym{HKDF}{HKDF}{Hashed Message Authentication Code based Key Derivation Functions}
\newacronym{SHA}{SHA}{Secrure Hash Algorithms}
\newacronym{IV}{IV}{Initialization Vector}
\newacronym{ADD}{ADD}{Additional authentication data}
\newacronym{PBKDF}{PBKDF}{Password-Based Key Derivation Function}
\newacronym{NIST}{NIST}{U.S. National Institute of Standards and Technology}
\newacronym{APFS}{APFS}{Apple Filesystem}
\newacronym{SSD}{SSD}{Solid-State drive}
\newacronym{VFS}{VFS}{Virtual Filesystem}
\newacronym{OSN}{OSN}{Open Social Networks}
\newacronym{GCSF}{GCSF}{Google Conduce Sistem de Fișiere}
\newacronym{LRU}{LRU}{Least Recently Used}
\newacronym{I/O}{I/O}{In- and output}
\newacronym{EASCII}{EASCII}{Extended ASCII}
\newacronym{OSN}{OSN}{Open Social Networks}
% \gls{}
                %load the acronyms file

\input{templates/kth/lib/defines}  % load some additional definitions to make writing more consistent

% The following is needed in conjunction with generating the DiVA data with abstracts and keywords using the scontents package and a modified listings environment
%\usepackage{listings}   %  already included
\ExplSyntaxOn
\newcommand\typestoredx[2]{\expandafter\__scontents_typestored_internal:nn\expandafter{#1} {#2}}
\ExplSyntaxOff
\makeatletter
\let\verbatimsc\@undefined
\let\endverbatimsc\@undefined
\lst@AddToHook{Init}{\hyphenpenalty=50\relax}
\makeatother

\lstnewenvironment{verbatimsc}
    {
    \lstset{%
        basicstyle=\ttfamily\tiny,
        backgroundcolor=\color{white},
        %basicstyle=\tiny,
        %columns=fullflexible,
        columns=[l]fixed,
        language=[LaTeX]TeX,
        %numbers=left,
        %numberstyle=\tiny\color{gray},
        keywordstyle=\color{red},
        breaklines=true,                 % sets automatic line breaking
        breakatwhitespace=true,          % sets if automatic breaks should only happen at whitespace
        %keepspaces=false,
        breakindent=0em,
        %fancyvrb=true,
        frame=none,                     % turn off any box
        postbreak={}                    % turn off any hook arrow for continuation lines
    }
}{}


%% definition of new command for bytefield package
% \newcommand{\colorbitbox}[3]{%
% 	\rlap{\bitbox{#2}{\color{#1}\rule{\width}{\height}}}%
% 	\bitbox{#2}{#3}}

%% Acronyms
% note that nonumberlist - removes the cross references to the pages where the acronym appears
% note that nomain - does not produce a main glossary, this only acronyms will be in the glossary
% note that nopostdot - will present there being a period at the end of each entry
\usepackage[acronym, section=section, nonumberlist, nomain, nopostdot]{glossaries}
%% GQMJr changed to not pass the automake option - for Overleaf
\usepackage[]{glossaries-extra}
%\usepackage[automake]{glossaries-extra}
\ifinswedish
    %\usepackage{glossaries-swedish}
\fi

% Because backref is not compatible with biblatex
\ifbiblatex
    \usepackage[plainpages=false]{hyperref}
\else
    \usepackage[
    backref=page,
    pagebackref=false,
    plainpages=false,
                            % PDF related options
    unicode=true,           % Unicode encoded PDF strings
    bookmarks=true,         % generate bookmarks in PDF files
    bookmarksopen=false,    % Do not automatically open the bookmarks in the PDF reading program
    pdfpagemode=UseNone,    % None, UseOutlines, UseThumbs, or FullScreen
    ]{hyperref}
    \usepackage{backref}
    %
    % Customize list of backreferences.
    % From https://tex.stackexchange.com/a/183735/1340
    \renewcommand*{\backref}[1]{}
    \renewcommand*{\backrefalt}[4]{%
    \ifcase #1%
          \or [Page~#2.]%
          \else [Pages~#2.]%
    \fi%
    }
\fi
\usepackage[all]{hypcap}	%% prevents an issue related to hyperref and caption linking


\input{templates/kth/lib/includes-after-hyperref}

% \glsdisablehyper
\makeglossaries
%%% Local Variables:
%%% mode: latex
%%% TeX-master: t
%%% End:

% The form of the entries in this file is \newacronym{label}{acronym}{phrase}
%                                      or \newacronym[options]{label}{acronym}{phrase}
% see "User Manual for glossaries.sty" for the  details about the options, one example is shown below
% note the specification of the long form plural in the line below
% \newacronym[longplural={Debugging Information Entities}]{DIE}{DIE}{Debugging Information Entity}
% %
% % The following example also uses options
% \newacronym[plural={OSes}, firstplural={operating systems (OSes)}]{OS}{OS}{operating system}

% % note the use of a non-breaking dash in long text for the following acronym
% \newacronym{IQL}{IQL}{Independent Q‑Learning}

% \newacronym{LAN}{LAN}{Local Area Network}
% % note the use of a non-breaking dash in the following acronym
% \newacronym{WiFi}{Wi-Fi}{Wireless Fidelity}

% \newacronym{WLAN}{WLAN}{Wireless Local Area Network}
% \newacronym{UN}{UN}{United Nations}
% \newacronym{SDG}{SDG}{Sustainable Development Goal}

\newacronym{FFS}{FFS}{Fejk Filesystem}
\newacronym{FFFS}{FFFS}{Fejk Fejk Filesystem}
\newacronym{OWS}{OWS}{Online Web Service}
\newacronym{FOWS}{FOWS}{Fake Online Web Service}
\newacronym{CED}{CED}{Complete Encrypted Data}
\newacronym{LCED}{LCED}{Length of Complete Encrypted Data}
\newacronym{FUSE}{FUSE}{Filesystem in Userspace}
\newacronym{AES}{AES}{Advanced Encryption Standard}
\newacronym{GCM}{GCM}{Galois/Counter Mode}
\newacronym{PCD}{PCD}{Pixel Color Data}
\newacronym{RGB}{RGB}{Red Green Blue}
\newacronym{HMAC}{HMAC}{Hashed Message Authentication Code}
\newacronym{HKDF}{HKDF}{Hashed Message Authentication Code based Key Derivation Functions}
\newacronym{SHA}{SHA}{Secrure Hash Algorithms}
\newacronym{IV}{IV}{Initialization Vector}
\newacronym{ADD}{ADD}{Additional authentication data}
\newacronym{PBKDF}{PBKDF}{Password-Based Key Derivation Function}
\newacronym{NIST}{NIST}{U.S. National Institute of Standards and Technology}
\newacronym{APFS}{APFS}{Apple Filesystem}
\newacronym{SSD}{SSD}{Solid-State drive}
\newacronym{VFS}{VFS}{Virtual Filesystem}
\newacronym{OSN}{OSN}{Open Social Networks}
\newacronym{GCSF}{GCSF}{Google Conduce Sistem de Fișiere}
\newacronym{LRU}{LRU}{Least Recently Used}
\newacronym{I/O}{I/O}{In- and output}
\newacronym{EASCII}{EASCII}{Extended ASCII}
\newacronym{OSN}{OSN}{Open Social Networks}
% \gls{}
                %load the acronyms file

% insert the configuration information with author(s), examiner, supervisor(s), ...
%% Information for inside title page
\authorsLastname{Olsson}
\authorsFirstname{Glenn}
\email{glennol@kth.se}
\kthid{u18orpa8}
\authorsSchool{\schoolAcronym{EECS}}

%second author information

%External cooperation information
% \hostcompany{Cal Poly}
\hostorganization{Cal Poly}   % if there was a host organization

%Supervisor(s) information
\supervisorAsLastname{Ghasemirahni}
\supervisorAsFirstname{Hamid}
\supervisorAsEmail{hamidgr@kth.se}
\supervisorAsKTHID{u1fz5jtv}
\supervisorAsSchool{\schoolAcronym{None}}
\supervisorAsDepartment{Computer Science}
\supervisorBsLastname{Peterson}
\supervisorBsFirstname{Zachory}
\supervisorBsEmail{znjpeterson@gmail.com}
\supervisorBsOrganization{Cal Poly}

%Examiner information
\examinersLastname{Maguire Jr}
\examinersFirstname{Gerald Quentin}
\examinersEmail{maguire@kth.se}
\examinersKTHID{u1d13i2c}
\examinersSchool{\schoolAcronym{None}}
\examinersDepartment{Computer Science}

%Date

\date{\today}

%course and program information
\courseCycle{2}
\courseCode{DA231X}
\edprogram{Master's Programme, Computer Science, 120 credits}
\programcode{TCSCM}
\degreeName{Degree of Master (120 credits)}
%%%%% for DiVA's National Subject Category information
%%% Enter one or more 3 or 5 digit codes
%%% See https://www.scb.se/contentassets/3a12f556522d4bdc887c4838a37c7ec7/standard-for-svensk-indelning--av-forskningsamnen-2011-uppdaterad-aug-2016.pdf
%%% See https://www.scb.se/contentassets/10054f2ef27c437884e8cde0d38b9cc4/oversattningsnyckel-forskningsamnen.pdf
%%%%
%%%% Some examples of these codes are shown below:
% 102 Data- och informationsvetenskap (Datateknik)    Computer and Information Sciences
% 10201 Datavetenskap (datalogi) Computer Sciences
% 10202 Systemvetenskap, informationssystem och informatik (samhällsvetenskaplig inriktning under 50804)
% Information Systems (Social aspects to be 50804)
% 10203 Bioinformatik (beräkningsbiologi) (tillämpningar under 10610)
% Bioinformatics (Computational Biology) (applications to be 10610)
% 10204 Människa-datorinteraktion (interaktionsdesign) (Samhällsvetenskapliga aspekter under 50803) Human Computer Interaction (Social aspects to be 50803)
% 10205 Programvaruteknik Software Engineering
% 10206 Datorteknik Computer Engineering
% 10207 Datorseende och robotik (autonoma system) Computer Vision and Robotics (Autonomous Systems)
% 10208 Språkteknologi (språkvetenskaplig databehandling) Language Technology (Computational Linguistics)
% 10209 Medieteknik Media and Communication Technology
% 10299 Annan data- och informationsvetenskap Other Computer and Information Science
%%%
% 202 Elektroteknik och elektronik Electrical Engineering, Electronic Engineering, Information Engineering
% 20201 Robotteknik och automation Robotics
% 20202 Reglerteknik Control Engineering
% 20203 Kommunikationssystem Communication Systems
% 20204 Telekommunikation Telecommunications
% 20205 Signalbehandling Signal Processing
% 20206 Datorsystem Computer Systems
% 20207 Inbäddad systemteknik Embedded Systems
% 20299 Annan elektroteknik och elektronik Other Electrical Engineering, Electronic Engineering, Information Engineering
%% Example for a thesis in Computer Science and Computer Systems

%National Subject Categories information
\nationalsubjectcategories{10201}

% Added by Glenn
\subjectArea{Computer Science and Engineering}
\courseCredits{120}

\title{FFS: A cryptographic cloud-based deniable filesystem through exploitation of online web services}
% \title{FFS: A cryptographic \mbox{cloud-based} deniable filesystem through exploitation of online web services}
\subtitle{Store your sensitive data in plain sight}

% give the alternative title - i.e., if the thesis is in English, then give a Swedish title
\alttitle{FFS: Ett kryptografiskt molnbaserat steganografiskt filsystem genom utnyttjande av onlinebaserade webtjänster}
% \alttitle{FFS: Ett kryptografiskt molnbaserat förnekningsbart filsystem genom utnyttjande av onlinebaserade webtjänster}
\altsubtitle{Lagra din känsliga data öppet}
% alternative, if the thesis is in Swedish, then give an English title
%\alttitle{This is the English translation of the title}
%\altsubtitle{This is the English translation of the subtitle}

% Enter the English and Swedish keywords here for use in the PDF meta data _and_ for later use
% following the respective abstract.
% Try to put the words in the same order in both languages to facilitate matching. For example:
\EnglishKeywords{Filesystem, Fejk FileSystem, \mbox{Cloud-based} filesystem, Steganograhpic filesystem}
\SwedishKeywords{Filsystem, Fejk FileSystem, Molnbaserat filsystem, Steganografiskt filsystem}

%%%%% For the oral presentation
%% Add this information once your examiner has scheduled your oral presentation
\presentationDateAndTimeISO{2023-03-20 16:00}
\presentationLanguage{eng}
\presentationRoom{via Zoom https://kth-se.zoom.us/j/63695419495}
\presentationAddress{Isafjordsgatan 22 (Kistagången 16)}
\presentationCity{Stockholm}

% When there are multiple opponents, separate their names with '\&'
% Opponent's information
\opponentsNames{Arnar Pétursson}

% Once a thesis is approved by the examiner, add the TRITA number
% for entering the TRITA number for a thesis
\trita{TRITA-EECS-EX}{2022:00}

% Put the title, author, and keyword information into the PDF meta information
\input{templates/kth/lib/pdf_related_includes}


% the custom colors and the commands are defined in defines.tex    
\hypersetup{
	colorlinks  = true,
	breaklinks  = true,
	linkcolor   = \linkscolor,
	urlcolor    = \urlscolor,
	citecolor   = \refscolor,
	anchorcolor = black
}
%% GQMJr increased header height to avoid warniing from fancyhdr
\setlength{\headheight}{25.89253pt}
\begin{document}
%\selectlanguage{swedish}
%
\selectlanguage{english}

%%% Set the numbering for the title page to a numbering series not in the preface or body
\pagenumbering{alph}
\kthcover
\titlepage
% document/book information page
\bookinfopage

% Frontmatter includes the abstracts and table-of-contents
\frontmatter
\setcounter{page}{1}
\begin{abstract}
% The first abstract should be in the language of the thesis.
% Abstract fungerar på svenska också.
  \markboth{\abstractname}{}
\begin{scontents}[store-env=lang]
eng
\end{scontents}
%%% The contents of the abstract (between the begin and end of scontents) will be saved in LaTeX format
%%% and output on the page(s) at the end of the thesis with information for DiVA facilitating the correct
%%% entry of the meta data for your thesis.
%%% These page(s) will be removed before the thesis is inserted into DiVA.
\begin{scontents}[store-env=abstracts,print-env=true]
% \todo[inline, backgroundcolor=kth-lightgreen]{All theses at KTH are \textbf{required} to have an abstract in both \textit{English} and \textit{Swedish}.}

% \todo[inline, backgroundcolor=kth-lightgreen]{GLENNE HARRIE OLLESSON Exchange students many want to include one or more abstracts in the language(s) used in their home institutions to avoid the need to write another thesis when returning to their home institution.}

% \todo[inline]{Keep in mind that most of your potential readers are only going to read your \texttt{title} and \texttt{abstract}. This is why it is important that the abstract give them enough information that they can decide is this document relevant to them or not. Otherwise the likely default choice is to ignore the rest of your document.\\

% A abstract should stand on its own, i.e., no citations, cross references to the body of the document, acronyms must be spelled out, \ldots .\\

% Write this early and revise as necessary. This will help keep you focused on what you are trying to do.}
Today there are free online services that can be used to store files of arbitrary types and sizes, such as Google Drive. However, these services are often limited by a certain total storage size. The goal of this thesis is to create a filesystem that can store arbitrary amount and types of data, i.e. without any real limit to the total storage size. This is to be achieved by taking advantage of online webpages, such as Flickr, where text and files can be posted on free accounts with no apparent limit on storage size. The aim is to have a filesystem that behaves similar to any other filesystem but where the actual data is stored for free on various websites. 

% TODO: Chip - One of the thinks that was learned is that Twitter could not provide a reliable way of storing data, since the method of encoding images was changes - thus the data could not be recovered from the re-coded images. This lead to a change to using Flickr. However, there remains the risks that they too will change their encoding, hence the stored data will no longer be recoverable.Given the performance results of measurements of the file system this is an nearly completely unacceptable level of performance. Additionally, it is bad from a sustainable development point of view as it is nearly efficient or economically viable.
% Many \glspl{OWS} today, such as Flickr and Twitter, provide users with the possibility to post images that are stored on the platform for free. This thesis explores creating a cryptographically secure filesystem that stores its data on an online web service by encoding the encrypted data as images. Images have been selected as the target as more data can usually be stored in image posts than text posts on \glspl{OWS}. The proposed filesystem, named \gls{FFS}, provides users with free, deniable, and cryptographic storage by exploiting the storage provided by online web services. The thesis analyzes and compares the performance of \gls{FFS} against two other filesystems and a version of \gls{FFS} that does not use an \gls{OWS}. While \gls{FFS} has performance limitations that make it non-viable as a general-purpose filesystem, such as a substitute for the local filesystem on a computer; however, \gls{FFS} provides security benefits compared to other cloud-based filesystems specifically by providing end-to-end encryption, authenticated encryption, and plausible deniability of the data. Furthermore, being a cloud-based filesystem, \gls{FFS} can be mounted on any computer with the same operating system, given the correct secrets. 
\end{scontents}
% \todo[inline, backgroundcolor=kth-lightgreen]{The following are some notes about what can be included (in terms of LaTeX) in your abstract.}
% Choice of typeface with \textbackslash textit, \textbackslash textbf, and \textbackslash texttt:  \textit{x}, \textbf{x}, and \texttt{x}

% Text superscripts and subscripts with \textbackslash textsubscript and \textbackslash textsuperscript: A\textsubscript{x} and A\textsuperscript{x}

% Some useful symbols: \textbackslash textregistered, \textbackslash texttrademark, and \textbackslash textcopyright. For example, 
% copyright symbol: \textbackslash textcopyright Maguire 2022, and some superscripts: \textbackslash textsuperscript\{99m\}Tc, A\textbackslash textsuperscript\{*\}, A\textbackslash textsuperscript\{\textbackslash textregistered\}, and A\textbackslash texttrademark : \textcopyright Maguire 2022, and some superscripts: \textsuperscript{99m}Tc, A\textsuperscript{*}, A\textsuperscript{\textregistered}, and A\texttrademark. Another example: H\textbackslash textsubscript\{2\}O: H\textsubscript{2}O

% Simple environment with begin and end: itemize and enumerate and within these \textbackslash item

% The following macros can be used: \textbackslash eg, \textbackslash Eg, \textbackslash ie, \textbackslash Ie, \textbackslash etc, and \textbackslash etal: \eg, \Eg, \ie, \Ie, \etc, and \etal

% The following macros for numbering with lower case roman numerals: \textbackslash first, \textbackslash second, \textbackslash third, \textbackslash fourth, \textbackslash fifth, \textbackslash sixth, \textbackslash seventh, and \textbackslash eighth: \first, \second, \third, \fourth, \fifth, \sixth, \seventh, and \eighth.

% Equations using \textbackslash( xxxx \textbackslash) or \textbackslash[ xxxx \textbackslash] can be used in the abstract. For example: \( (C_5O_2H_8)_n \)
% or \[ \int_{a}^{b} x^2 \,dx \]


% Even LaTeX comments can be handled, for example: \% comment at end

\subsection*{Keywords}
\begin{scontents}[store-env=keywords,print-env=true]
% If you set the EnglishKeywords earlier, you can retrieve them with:
\InsertKeywords{english}
% If you did not set the EnglishKeywords earlier then simply enter the keywords here:
% comma separate keywords, such as: Canvas Learning Management System, Docker containers, Performance tuning
\end{scontents}

% Choose the most specific keyword from those used in your domain, see for example: the ACM Computing Classification System ({\small \url{https://www.acm.org/publications/computing-classification-system/how-to-use})},
% the IEEE Taxonomy ({\small \url{https://www.ieee.org/publications/services/thesaurus-thank-you.html}}), PhySH (Physics Subject Headings)\linebreak[4] ({\small \url{https://physh.aps.org/}}), \ldots or keyword selection tools such as the  National Library of Medicine's Medical Subject Headings (MeSH)  ({\small \url{https://www.nlm.nih.gov/mesh/authors.html}}) or Google's Keyword Tool ({\small \url{https://keywordtool.io/}})\\

% \textbf{Mechanics}:
% \begin{itemize}
%   \item The first letter of a keyword should be set with a capital letter and proper names should be capitalized as usual.
%   \item Spell out acronyms and abbreviations.
%   \item Avoid "stop words" - as they generally carry little or no information.
%   \item List your keywords separated by commas (",").
% \end{itemize}    
% Since you should have both English and Swedish keywords - you might think of ordering them in corresponding order (\ie, so that the n\textsuperscript{th} word in each list correspond) - this makes it easier to mechanically find matching keywords.
\end{abstract}
\cleardoublepage
\babelpolyLangStart{swedish}
\begin{abstract}
    \markboth{\abstractname}{}
\begin{scontents}[store-env=lang]
swe
\end{scontents}
\begin{scontents}[store-env=abstracts,print-env=true]
% \todo[inline, backgroundcolor=kth-lightblue]{Alla avhandlingar vid KTH \textbf{måste ha} ett abstrakt på både \textit{engelska} och \textit{svenska}.\\
% Om du skriver din avhandling på svenska ska detta göras först (och placera det som det första abstraktet) - och du bör revidera det vid behov.}

% \todo[inline]{If you are writing your thesis in English, you can leave this until the draft version that goes to your opponent for the written opposition. In this way you can provide the English and Swedish abstract/summary information that can be used in the announcement for your oral presentation.\\

% If you are writing your thesis in English, then this section can be a summary targeted at a more general reader. However, if you are writing your thesis in Swedish, then the reverse is true – your abstract should be for your target audience, while an English summary can be written targeted at a more general audience.\\

% This means that the English abstract and Swedish sammnfattning  
% or Swedish abstract and English summary need not be literal translations of each other.
% }
% \todo[inline, backgroundcolor=kth-lightred]{Do not use the \textbackslash glspl\{\} macro in an abstract that is not in English, as my programs do not know how to generate plurals in other languages. Instead you will need to spell these terms out or give the proper plural form.}

% \todo[inline, backgroundcolor=kth-lightgreen]{The abstract in the language used for the thesis should be the first abstract, while the Summary/Sammanfattning in the other language can follow}


% Many \gls{OWS}s today, such as Flickr and Twitter, provide users with the possibility to post images which are stored on the platform for free. This thesis explores the idea of creating a cryptographically secure filesystem which stores its data on an online web service using encoded and encrypted images. More data can usually be stored in image posts than in a text posts on \gls{OWS}s. The filesystem, named \gls{FFS}, provides users with free, deniable, and cryptographic storage by exploiting the storage provided by these online web services. The thesis analyzes and compares the performance of \gls{FFS} against two other filesystems and a version of \gls{FFS} that does not use an \gls{OWS}. It can be concluded that \gls{FFS} has limitations in performance, making it unviable as a \mbox{general-purpose} filesytem, such as a substitute to the local filesystem on a computer. However, it provides security benefits compared to other \mbox{cloud-based} filesystems such as \mbox{end-to-end} encryption, authenticated encryption, and plausible deniability of the data. Furthermore, being a \mbox{cloud-based} filesystem, \gls{FFS} can be mounted on any computer with the same operating system, given the correct secrets. 

Flertalet webbtjänster idag, så som Flickr och Twitter, tillhandahåller möjligheten att lägga upp bilder som lagras på tjänsten utan kostnad. Denna avhandling utforskar en ide om att skapa ett kryptografiskt säkert filsystem som lagrar sin data på webbtjänster med hjälp av kodade och krypterade bilder. Mer data kan ofta lagras i bildinlägg än i textinlägg på webbtjänsterna. Filsystemet, döpt Fejk Filesystem (FFS), tillhandahåller användare med gratis, förnekningsbar, och kryptografisk lagring genom att utnyttja det lagringsutrymme som tillhandahålls av dessa webbtjänster. Avhandlingen analyserar och jämför prestandan hos FFS med två andra filsystem, samt en version av FFS som inte använder en webbtjänst. En slutsats som kan dras är att FFS har begränsningar i prestanda, vilket gör det ohållbart som ett generellt-ändamålsfilssystem, till exempel som ersättare för det lokala filsystemet på en dator. Däremot tillhandahåller det säkerhetsförmåner jämfört med andra molnbaserade filsystem, så som början-till-slutetkryptering, autentiserad kryptering, samt sannolik förnekningsbarhet av datan. Dessutom, som ett molnbaserat filsystem kan FFS monteras på vilken dator som helst med samma operativsystem, givet de korrekta nycklarna. 
%Flertalet webbtjänster idag, så som Flickr och Twitter, tillhandahåller möjligheten att lägga upp bilder som lagras på tjänsten utan kostnad. Denna avhandling utforskar en ide om att skapa ett kryptografiskt säkert filsystem som lagrar sin data på webbtjänster med hjälp av kodade och krypterade bilder. Mer data kan ofta lagras i bildinlägg än i textinlägg på webbtjänsterna. Filsystemet, döpt Fejk Filesystem (FFS), tillhandahåller användare med gratis, förnekningsbar, och kryptografisk lagring genom att utnyttja det lagringsutrymme som tillhandahålls av dessa webbtjänster. Avhandlingen analyserar och jämför prestandan hos FFS med två andra filsystem, samt en version av FFS som inte använder en webbtjänst. En slutsats som kan dras är att FFS har begränsningar i prestanda, vilket gör det ohållbart som ett generellt-ändamålsfilssystem, till exempel som ersättare för det lokala filsystemet på en dator. Däremot tillhandahåller det säkerhetsförmåner jämfört med andra molnbaserade filsystem, så som början-till-slutetkryptering, autentiserad kryptering, samt sannolik förnekningsbarhet av datan. Dessutom, som ett molnbaserat filsystem kan FFS monteras på vilken dator som helst med samma operativsystem, givet de korrekta nycklarna. 
\end{scontents}
\subsection*{Nyckelord}
\begin{scontents}[store-env=keywords,print-env=true]
% SwedishKeywords were set earlier, hence we can use alternative 2
\InsertKeywords{swedish}
\end{scontents}

\end{abstract}
\babelpolyLangStop{swedish}

\cleardoublepage

\section*{Acknowledgments }
\markboth{Acknowledgments}{}

\noindent
Thanks to my mom, dad, and the rest of my family for their constant support. 

To the people who said it could not be done - it is done.

% \begin{itemize}
%     \item 
% \end{itemize}


\acknowlegmentssignature

\fancypagestyle{plain}{}
\renewcommand{\chaptermark}[1]{ \markboth{#1}{}} 
\tableofcontents
  \markboth{\contentsname}{}

\cleardoublepage
\listoffigures

\cleardoublepage

\listoftables
\cleardoublepage
\lstlistoflistings % \todo[inline, backgroundcolor=kth-lightgreen]{If you have listings in your thesis. If not, then remove this preface page.}
\cleardoublepage
% Align the text expansion of the glossary entries
\newglossarystyle{mylong}{%
  \setglossarystyle{long}%
  \renewenvironment{theglossary}%
     {\begin{longtable}[l]{@{}p{\dimexpr 2cm-\tabcolsep}p{0.8\hsize}}}% <-- change the value here
     {\end{longtable}}%
 }
%\glsaddall
%\printglossaries[type=\acronymtype, title={List of acronyms}]
\printglossary[style=mylong, type=\acronymtype, title={List of acronyms and abbreviations}]
%\printglossary[type=\acronymtype, title={List of acronyms and abbreviations}]

%% The following label is essential to know the page number of the last page of the preface
%% It is used to computer the data for the "For DIVA" pages
\label{pg:lastPageofPreface}
% Mainmatter is where the actual contents of the thesis goes
\mainmatter
\glsresetall
\renewcommand{\chaptermark}[1]{\markboth{#1}{}}
\selectlanguage{english}


\chapter{Introduction}
\label{ch:introduction}

To keep files and data secure we often use encrypted filesystems. However, while these filesystems hide the content of the data, they often do not conceal the existence of data. For instance, using snapshots of the filesystems from different moments in time, it could be possible to notice a difference in the data stored and therefore that data exists and where it is located. Snapshots could even reveal user passwords\,\cite{hanMultiuserSteganographicFile2010}.

Deniable filesystems are intended to make the data deniable, meaning that the user is supposed to be able to plausibly deny the existence of data. This is often accomplished through the use of digital steganography. There are many reasons why this is important. For instance, in 2011, a Syrian man recorded videos of attacks on civilians carried out by Syrian security forces, which he wanted to share with the world\,\cite{westheadHowSyrianRefugee2012}. By cutting his arm, he was able to hide a memory card inside the wound and smuggled it out of the country. However, if he would have used methods such as an encrypted deniable filesystem, the border control may not have been able to discover even the existence of data, even if they would have found the memory card. By only encrypting the data, the border control would have been able to see that he was trying to hide data and make him reveal the decryption key, either by legal measures or by force, which is why he smuggled it out.

There exist multiple deniable filesystems that are designed to combat this problem on physical devices, such as memory cards. However, even just carrying a memory card might subject you to suspicion of hiding data, no matter how the filesystem is designed. Another solution to hiding the data is therefore to hide it somewhere else, for instance online through the use of a cloud-based filesystem service, such as Google Drive. Someone searching your body and devices, at for instance an airport or border control, might not realize that you are using a cloud-based filesystem service to hide your data. Although, more thorough investigations of a person might reveal user accounts used on the service, leading to legal processes where the service is forced to disclose your data. Even if you encrypt the data you upload to such a service, you can still be forced to reveal the decryption keys. What we want to achieve is a combination of a deniable filesystem and a cloud-based filesystem, where the data is stored using digital cryptographic and steganographic methods but without any company or person other than the user controlling the actual data. To accomplish this, we can store the data on online social media platforms.

Social media platforms such as Twitter and Flickr have many millions of daily users that post texts and images (for example, of their cats or funny videos). According to Henna Kermani at Twitter, they processed ~\SI{200}{\giga\byte} of image data every second in 2016\,\cite{MobileScaleLondona}. The photos posted on Twitter, as opposed to the ones stored on cloud services such as Google Drive, are stored for free on the service for the user, for what seems to be an indefinite period. There is also no specified limit on how many images or tweets one can make. Although, as stated in their terms of service, such limits can be imposed on specific users whenever Twitter wishes, and tweets can be removed at any point in time\,\cite{twitterTwitterTermsService2021}.

This project intends to create a cryptographic and deniable cloud-based filesystem called the \textit{Fejk FileSystem} (\gls{FFS}) which takes advantage of free online web services, such as Twitter and Flickr, for the actual storage. The idea is to save the user's files by posting an encrypted version of the file as images and text posts on these web services. The intention is not to create a revolutionary fast and usable filesystem but instead to explore how well it is possible to utilize the storage that Twitter and similar services provide their users for free, as a cryptographic and deniable cloud-based filesystem. Additionally, the performance and limits of this filesystem will be analyzed and compared to alternative filesystems, such as Google Drive, to compare the advantages and disadvantages of the developed filesystem compared to professional filesystems. The security of the filesystem will also be discussed, as well as an analysis of the steganographic capability of the developed filesystem.

% \section{Project Overview}

This project intends to create a filesystem called \textit{Fejk FileSystem} (FFS) which takes advantage of online web services, such as Twitter, for the actual storage. The idea is to save the user's files by posting or sending an encrypted version of the file as posts or private messages on these web services. The intention is not to create a revolutionary fast and usable filesystem but instead to explore how well it is possible to utilize the storage that Twitter and similar services provides for free as a filesystem. However, the performance and limits of this filesystem will be analyzed and compared to existing alternatives, such as Google Drive, to compare the benefits of this free storage compared to a professional system that might cost money. The security of the filesystem will also be discussed, as well as an analysis of the steganographic capability of the developed filesystem. 
% Moved the content to this page instead


\section{Problem}
\label{sec:problem}
% \todo[inline, backgroundcolor=kth-lightblue]{svensk: Problemdefinition eller Frågeställning\\
% Lyft fram det ursprungliga problemet om det finns något och definiera därefter
% den ingenjörsmässiga erfarenheten eller/och vetenskapen som kan komma ur
% projektet. }

Is it possible to create a secure and deniable cloud-based filesystem that store the data on various online web services through the use of the free user accounts offered by these online web services? What are the drawbacks of such a filesystem compared to similar filesystem solutions with regards to write and read speed, storage capacity, and reliability? Are there advantages to such a system in regards to security and deniability? 

% Longer problem statement\\
% If possible, end this section with a question as a problem statement.

% % Research Question
% \subsection{Original problem and definition}
% \label{sec:researchQuestion}
% % \todo[inline, backgroundcolor=kth-lightblue]{Ursprungligt problem och definition}
% Some text

% \subsection{Scientific and engineering issues}% \todo[inline, backgroundcolor=kth-lightblue]{Vetenskaplig och ingenjörsmässig frågeställning}
% some text


\section{Purpose and motivation}
% \todo[inline, backgroundcolor=kth-lightblue]{Syfte}
% \todo[inline, backgroundcolor=kth-lightblue]{Skilj på syfte och mål! Syfte är att förändra något till det bättre. I examensarbetet finns ofta två aspekter på detta. Dels vill problemägaren (företaget) få sitt problem löst till det bättre men akademin och ingenjörssamfundet vill också få nya erfarenheter och vetskap. Beskriv ett syfte som tillfredställer båda dessa aspekter.\\
% Det finns även ett syfte till som kan vara värt att beakta och det är att du som student skall ta examen och att du måste bevisa, i ditt examensarbete, att du uppfyller examensmålen. Dessa mål sammanfaller med kursmålen för examensarbetskursen. 
% }
% \todo[inline]{State the purpose  of your thesis and the purpose of your degree project.\\
% Describe who benefits and how they benefit if you achieve your goals. Include anticipated ethical, sustainability, social issues, etc. related to your project. (Return to these in your reflections in Section~\ref{sec:reflections}.)}

The purpose of this research is to explore the possibility to create a secure, steganographic cloud-based filesystem that stores data on \gls{OWS}s and to compare the performance, benefits, and disadvantages of such a filesystem to existing steganographic filesystems and distributed filesystem services. A distributed filesystem service, such as Google Drive, provide data storage for users which can be both free and cost money. Even though Google Drive encrypts the user's data, they control the encryption and decryption keys, and the method of encryption\,\cite{johnsonGoogleDriveSecure2021}. This means that they can give out the user's files and data if faced with legal actions such as subpoenas. It also opens up the possibility of hackers gaining access to the files without the user having any way to control them.

% -- Even using a cloud-based filesystem service as a layer in a stacked filesystem with a cryptographic and deniable layer, the amount of data stored in the system would still be visible to the service provider even if some or even all of that data is noise. It is also an obvious way to store data on a service that provides conventional storage. 

The idea behind \gls{FFS} is to have a decentralized cloud-based filesystem where only the user has access to the unencrypted data. By encrypting and decrypting the files locally before uploading and after downloading them to these services (end-to-end encryption), it is possible to ensure that the user is the only one who has access to the encryption and decryption keys and therefore the unencrypted data. Even if the web service would look at the data uploaded by the user, it is unreadable without the decryption key. An interesting aspect of this is that online web services, such as social media, provide users with essentially an unbounded amount of storage for free. Anyone can create any number of accounts on Twitter and Facebook without cost, and with enough accounts, one could potentially store all their data using such a filesystem. We aim to exploit the storage web services give their users for free. As the file data is stored in the open but only accessible by the user, and as \gls{FFS} can be unmounted to hide its existence, it is steganographic. 

There are several steganographic filesystems available but these lack certain aspects that \gls{FFS} aims to solve. Some filesystems are based on the local disk of the device in use, such as the physical storage device on a computer or phone, or an external storage device connected to a computer or phone. While these filesystems have advantages compared to cloud-based solutions, such as latency, they lack accessibility as you need to have the device to access the content on it. It also means that when you want to share or transport the data, you must physically move the device which can mean problems as it could for instance be taken from you or be destroyed. Cloud-based solutions counter this by being available from any location that has internet access to the services used. However, existing cloud-based solutions introduce other disadvantages. One example is CovertFS\,\cite{baliga2007web} where data is stored in images posted on web services. The images are actual images representing something, meaning that there is a limit on how much steganographic data can be stored. CovertFS limit this to \SI{4}{\kilo\byte} which means that such a filesystem with a lot of data will require many images which could lead to suspicion from the owners of the web services. \gls{FFS} stores as much data as possible in the images, meaning that less images are needed to store a file bigger than \SI{4}{\kilo\byte}. It also means that the images produced by \gls{FFS} do not look like a normal image, but instead has seemingly randomly colored pixels. More examples of similar filesystems will be presented in Chapter~\ref{ch:related_work}. 


\section{Goals}
% \todo[inline, backgroundcolor=kth-lightblue]{Mål}
% \todo[inline, backgroundcolor=kth-lightblue]{Skilj på syfte och mål. Syftet är att åstakomma en förändring i något. Målen är vad som konkret skall göras för att om möjligt uppnå den önskade förändringen (syfte). }

% \todo[inline]{State the goal/goals of this degree project.}

The project aims to create a secure, mountable filesystem that stores its data via online web services by taking advantage of the storage provided to its users. This can be split into the following subgoals;
\begin{enumerate}
\item to create a free mountable filesystem where files can be stored, read, and deleted, % \todo[inline, backgroundcolor=kth-lightblue]{för att tillfredsställa problemägaren – industrin?}
\item for the system to be secure in the sense that even with access to the uploaded files and the software, the data is not readable without the correct decryption key, and, % \todo[inline, backgroundcolor=kth-lightblue]{för att tillfredsställa ingenjörssamfundet och vetenskapen – akademin) }
\item to analyze the write and read speed, storage capacity, and reliability of the filesystem and compare it to commercial distributed filesystems.
% \item to be able to store a useful\footnote{} amount of data % \todo[inline, backgroundcolor=kth-lightblue]{eventuellt, för att uppfylla kursmålen – du som student}
\end{enumerate}

A side effect of such a filesystem that creates posts that are, while encrypted, publicly available is a steganographic filesystem in the sense that the data is hidden in plain sight. Therefore, an additional subgoal is to achieve and analyze the deniability of the filesystem. This could make the filesystem useful for persons who need to hide their data, such as spies, journalists, and political actors where freedom of speech is non-existent.

% \todo[inline]{In addition to presenting the goal(s), you might also state what the deliverables and results of the project are.}



\section{Research Methodology}%\todo[inline, backgroundcolor=kth-lightblue]{Undersökningsmetod}
% \todo[inline, backgroundcolor=kth-lightblue]{Här anger du vilken vilken övergripande undersökningsstrategi eller metod du skall använda för att försöka besvara den akademiska frågeställning och samtidigt lösa det e v ursprungliga problemet. Ofta kan man använda ”lösandet av ursprungsproblemet” som en fallstudie kring en akademisk frågeställning. Du undersöker någon intressant fråga i ”skarpt” läge och samlar resultat och erfarenhet ur detta.\\
% Tänk på att företaget ibland måste stå tillbaka i sin önskan och förväntan på projektets resultat till förmån för ny eller kompletterande ingenjörserfarenhet och vetenskap (ditt examensarbete). Det är du som student som bestämmer och löser fördelningen mellan dessa två intressen men se till att alla är informerade. }
% \todo[inline]{Introduce your choice of methodology/methodologies and method/methods – and the reason why you chose them. Contrast them with and explain why you did not choose other methodologies or methods. (The details of the actual methodology and method you have chosen will be given in Chapter~\ref{ch:methods}. Note that in Chapter~\ref{ch:methods}, the focus could be research strategies, data collection, data analysis, and quality assurance.)\\
% In this section you should present your philosophical assumption(s), research method(s), and research approach(es).}

% TODO: Fix comments here

The filesystem created through this thesis be written in C++11 and the FUSE MacOS library\cite{HomeMacFUSE} which enables us to write a filesystem in user space rather than kernel space. The produced filesystem will be evaluated against other filesystems, both commercial distributed systems, such as Google drive, but also an APFS filesystem on a Macbook laptop. Quantitative data will be gathered from the different filesystems through the use of experiments with the filesystem benchmarking software Iozone\cite{IozoneFilesystemBenchmark}. We will look at attributes such as the difference in speed of read and write, as well as the speed of random read and random write. 

\todo[inline]{Do I need to motivate the use of Iozone, as compared to Fio or FFSB? Should the attributes I will look at be motivated as well?}

% Quantitative data will be gathered from the filsystem developed as a result of this thesis, as well as different existing filesystems it will be compared against, such as Google Drive. This data will be gathered by experiments using filesystem benchmarking tools, such as fio\footnote{\url{https://fio.readthedocs.io/en/latest/fio_doc.html}}, where different variables of the benchmarking tools can be tested. 

% TODO: WHAT DO ADD?? FUUUCK

\section{Delimitations} % \todo[inline, backgroundcolor=kth-lightblue]{Avgränsningar}
% \todo[inline]{Describe the boundary/limits of your thesis project and what you are explicitly not going to do. This will help you bound your efforts – as you have clearly defined what is out of the scope of this thesis project. Explain the delimitations. These are all the things that could affect the study if they were examined and included in the degree project.}

Due to limitations in time and as the system is only a prototype for a working filesystem and not a production filesystem, some features found in other filesystems are not going to be implemented in FFS. This includes for instance file access control and symbolic links. The reason is that the goal is to present and evaluate the possibility of creating such a filesystem with a variety of different storage subsystems. The features are more limited to those that are useful in a regular filesystem. However FFS will only aim to implement a minimalistic filesystem. %TODO: This includes reading, deleting and writing files and directories and support for multiple levels of directory hierarchy etc etc

\section{Structure of the thesis} % \todo[inline, backgroundcolor=kth-lightblue]{ Rapportens disposition}
Chapter~\ref{ch:background} presents relevant background information about xxx.  Chapter~\ref{ch:methods} presents the methodology and method used to solve the problem. …


\cleardoublepage


\section{Ditt fula fan}
Hej hopp, gummisnopp

Balle balle, min skalle

\cleardoublepage

\chapter{Related work}
\label{ch:related_work}
The research area of creating filesystems to improve security, reliability, and deniability is not new and has been extensively worked on previously. This chapter presents prior work related to this thesis. This includes other filesystems that share similarities \gls{FFS}, for example, the idea of unconventional storage media and steganography. \Cref{sec:SteganographyAndDeniableFS} introduces the concepts of steganography and deniable filesystems. \Cref{sec:rel_crypto} discuses why this work use \gls{AES} rather than introducing yet another cryptographic system. \Cref{sec:rel_fs} discusses related filesystems. \Cref{sec:iozone} discusses the particular filesystem benchmarking system that has been used. Finally, \Cref{sec:relatedWorkSummary} gives a summary of the information provided in this chapter.



\section{Steganography and deniable filesystems}
\label{sec:SteganographyAndDeniableFS}
Steganography is the art of hiding information in plain sight and has been around for ages. For example today, a major part of steganography is hiding malicious code in images, called stegomalware or stegoware. Stegomalware is an increasing problem and in a sample set of \mbox{real-life} stegomalware, over 40\% of the cases used images to store the malicious code\,\cite{stichtingcuingfoundationSIMARGLStegwarePrimer2020}.While \gls{FFS} will not include malicious code in its images, the problem of stegomalware  has fostered the development of steganography detection techniques in (for instance) social media platforms. 

Twitter has been exposed for allowing steganographic images that can easily contain any type of file\,\cite{TwitterImagesCan}. David Buchanan created a simple python script of only \num{100} lines of code that can encode zip files, mp3 files, and any imaginable file in an image of the user's choosing\,\cite{buchananTweetablepolyglotpng2022}. He presents multiple examples of this technique on his Twitter profile\footnote{\url{https://twitter.com/David3141593}}. The fact that his images are public might be evidence that Twitter's steganography detection software is imperfect. However, it is also possible that Twitter has explicitly chosen to not remove these posts.

Other examples of steganographic data storage on \glspl{OSN} include the paper by \citeauthor{ningSecretMessageSharing2014} where the authors build a system for private communication on top of public \mbox{photo-sharing} web services\,\cite{ningSecretMessageSharing2014}. Due to the web services processing of uploaded multimedia, the authors first researched how the integrity of steganographic data could be maintained after being uploaded to these services. Following this, they presented an approach that ensured the integrity of the hidden messages in the uploaded images, while maintaining a low likelihood of discovery from steganographic analysis. \citeauthor{beatoUndetectableCommunicationOnline2014} also explored the idea of undetectable communications over \glspl{OSN}\,\cite{beatoUndetectableCommunicationOnline2014}. While they did not carry out an implementation, they present an idea where messages are encoded together with a cover object and a cryptographic key to produce a steganographic message which is then posted to \glspl{OSN}. They presented a \mbox{web-based} user interface client with a \gls{PHP} server backend as the method users would use to create and share their secret messages.

A deniable filesystem is a system that does not expose files stored on this system without credentials, \ie not providing information about how many files are stored, their sizes, their content, or even if there exist any files in the filesystem\,\cite{andersonSteganographicFileSystem1998}. A \mbox{rubber-hose} filesystem is a filesystem where if an adversary is only given one key out of $n$ keys in total, they cannot prove that more data exists and the filesystem is therefore deniable. This is known as a \mbox{rubber-hose} filesystem because of the idea behind \mbox{rubber-hose} cryptanalysis where an adversary beats the user with a \mbox{rubber-hose} to extract the encryption key. The adversary should have no way of knowing how many keys are used to encrypt the data. Steganographic methods can be used to hide the data for a deniable filesystem, and some deniable filesystems are also steganographic filesystems. However, deniability can be accomplished using techniques other than steganography. This thesis proposes a deniable filesystem that, while storing the data in images, is not steganographic as it does \textbf{not} hide the data in the images, but rather simply uses these images as the storage medium.

\section{Cryptography}
\label{sec:rel_crypto}
Some papers choose to invent their encryption methods rather than using established standards. \citeauthor{chumanEncryptionThenCompressionSystemsUsing2019} proposes a \mbox{scrambling-based} encryption scheme for images that splits the picture into multiple rectangular blocks that are randomly rotated and inverted, both horizontally and vertically, along with shuffling of the color components\,\cite{chumanEncryptionThenCompressionSystemsUsing2019}. This is used to demonstrate the security and integrity of images sent over insecure channels. The paper uses Twitter and Facebook to exhibit this. Despite its improvement and compatibility of a common image format, such as bitstream compliance, due to its \mbox{well-proven} security \gls{FFS} will use \gls{AES} as its encryption method. 

\section{Related filesystems}
\label{sec:rel_fs}
% FIXME: Chip: "The discussion of other FUSE based filesystems seems really minimal."
Multiple steganographic filesystems have been presented previously but many of these are focused on filesystems for physical storage disks to that the user has access. For instance, Timothy Peters created DEFY, a deniable filesystem using a log-based structure in 2014\,\cite{petersDEFYDeniableFile2014}. DEFY was built to be used exclusively on Solid State Drives (\gls{SSD}) found in mobile devices to provide a steganographic filesystem that could be used on Android phones. Further examples of local disk-based filesystems can be found in \cite{andersonSteganographicFileSystem1998, mcdonaldStegFSSteganographicFile2000, domingo-ferrerSharedSteganographicFile2008, hanMultiuserSteganographicFile2010}, among other papers. However, this paper aims to create a filesystem that is not based on a physical disk but rather a cloud-based steganographic filesystem that uses online web services as its storage medium. 

In 2007, \citeauthor{baliga2007web} presented an idea of a covert filesystem that hides the file data in images and uploads them to web services, named CovertFS\,\cite{baliga2007web}. The paper lacks implementation of the filesystem but they present an implementation plan which includes using FUSE. They limit the filesystem such that each image posted will only store a maximum of \SI{4}{\kilo\byte} of steganographic file data and the images posted on the web services will be actual images. This is different from the idea of FFS where the images will be purely the encrypted file data and will therefore not be an image that represents anything but will instead look like random color noise. An implementation of CovertFS has been attempted by \citeauthor{sosaSuperSecretFile2007} which also used Tor to further anonymize the users\,\cite{sosaSuperSecretFile2007}.

In \citeyear{szczypiorskiStegHashNewMethod2016}, \citeauthor{szczypiorskiStegHashNewMethod2016} introduced the idea of StegHash - a way to hide steganographic data on Open Social Networks (\gls{OSN}) by connecting multimedia files, such as images and videos, with hashtags\,\cite{szczypiorskiStegHashNewMethod2016}. Specifically, images were posted to Twitter and Instagram along with certain permutations of hashtags that pointed to other posts through the use of a custom-designed secret transition generator. StegHash managed to store short messages with 10 bytes of hidden data with a 100\% success rate, while longer messages with up to 400 bytes of hidden data had a success rate of 80\%. \citeauthor{bieniaszSocialStegDiscApplicationSteganography2017} later presented SocialStegDisc which was a filesystem application of the idea presented with StegHash\,\cite{bieniaszSocialStegDiscApplicationSteganography2017}. Multiple posts could be required to store a single file and each post referenced the next post like a linked list, which means that you only need the root post to read all the data. This is unlike the idea of FFS where a table will be kept to keep track of which posts store a certain file, and in what order they should be concatenated, similar to the idea of an inode table. SocialStegDisc lacks actual implementation of the filesystem but similar to CovertFS presents the idea of a social media-based filesystem.

TweetFS is a filesystem created by Robert Winslow that stores the data on Twitter\,\cite{winslowTweetfsTweetfsMaster}, created in 2011. It was created as a proof of concept to show that it is possible to store file data on Twitter. The filesystem uses sequential text posts to store the data. The filesystem is not mounted to the operating system, instead, the user interacts with a Python script  through the command line. This makes the filesystem less convenient from a user perspective, compared to a mounted filesystem where the files can be browsed using a user interface or command line. There are two commands available: \texttt{upload} and \texttt{download} which upload and download files or directories, respectively. Names and permissions of files and directories are maintained throughout the upload and download process. The tweets are not encrypted but are enciphered into English words which makes them look like nonsense paragraphs, similar to what we mentioned in Section~\ref{sec:data_storage} about how arbitrary data can be encoded as plain text. This makes the filesystem less secure than an encrypted version as it can be read by anyone with access to the decoder. However, it does introduce a steganographic element to the filesystem.

In 2006, \citeauthor{jonesGoogleHackUse2006} created GmailFS - a mountable filesystem that uses Google's Gmail to store the data\,\cite{jonesGoogleHackUse2006, jonesGmailFilesystemImplementation2006}. The filesystem was written in Python using FUSE and was presented well before the introduction of Google Drive in 2012. It does not support encryption as the plain file data is stored in emails. Today, Gmail and Google Drive share their storage quota and GmailFS has since become redundant as Google Drive is an easier filesystem to use. GMail Drive is another example of a Gmail-based filesystem and it was influenced by GmailFS\,\cite{viksoeViksoeDkGMail2004}. GMail Drive has been declared dead by its author since 2015.

Google Conduce Sistem de Fișiere (\gls{GCSF}) is a filesystem that stores its data on Google Drive, built using FUSE\,\cite{puscassergiudanGCSFVIRTUALFILE2018,puscasHarababurelGcsf2022}. On the other hand, Google Drive provides a desktop application\,\cite{googleInstallSetGoogle} that presents a mounted volume in the local filesystem, representing the user's Google Drive filesystem. The mountable volume provided by the desktop application does not always sync the stored data directly, but might instead store it locally until a later time. To enable direct synchronization of the data to Google Drive, GCSF interacts with the Google Drive REST API rather than the mounted filesystem volume. One benefit of always synchronizing the data with Google Drive is that the duration of a filesystem operation can be measured easily. For instance, a write operation on a file in GCSF will not complete before the new file data has been completely stored on Google Drive. Therefore, the duration from the start of the filesystem operation until its end includes the time it takes to upload the file. On the other hand, the duration of a filesystem operation on the mountable volume provided by the Google Drive Desktop application does not always include the time it takes to upload the file, this can occur at a later time. One difference between GCSF and the idea of FFS is that GCSF does not encrypt the data stored in the filesystem. While the data is, as mentioned previously, encrypted by Google Drive, the encryption keys are controlled by Google Drive, not the user of GCSF. The data stored on GCSF is also stored as its original files in Google Drive, not as images as FFS intends to store the data. The Google Drive filesystem architecture is utilized by GCSF, for instance by using its directory hierarchy structure. This allows GCSF to avoid creating its own inode table and directory structures, as Google Drive provides the functionality these structures similarly provide FFS, through the Google Drive API. The development of GCSF started in 2018\,\cite{puscasHarababurelGcsf2022}, and the repository in GitHub has around $2\,300$ starts as of writing. 

Another Google Drive-based filesystem is google-drive-ocamlfuse\,\cite{stradaGoogledriveocamlfuse2022}, developed for Linux using FUSE. The project is well received online. The repository has around $6\,700$ stars on GitHub at the time of writing and there are multiple articles online about the project\,\cite{guoanInstallGoogleDrive2021,sneddonMountYourGoogle2017,aminUseGoogleDrive2021}. The filesystem is well developed and, as of writing, well maintained. The filesystem supports filesystem operations such as symbolic links, Unix ownership, and multiple account support. According to the author of GCSF, GCSF tends to be faster than google-drive-ocamlfuse for certain operations, including reading cached files\,\cite{shubhamharnalShortGCSFTends2018,harababurelShowHNGoogle2018,}. google-drive-ocamlfuse has no native support of macOS but is focused on Linux. 

\citeauthor{zadokCryptfsStackableVnode1998} created Cryptfs, a stackable Vnode filesystem that encrypted the underlying, potentially unencrypted, filesystem\,\cite{zadokCryptfsStackableVnode1998}. By making the filesystem stackable, any layer can be added on top of any other, and the abstraction occurs by each Vnode layer communicating with the one beneath. There is a potential to further stack additional layers by using tools such as FiST\,\cite{FiSTStackableFile}. This approach enables one to create not only an encrypted filesystem but also to provide redundancy by replicating data to different underlying filesystems. If these filesystems are independent, then this potentially increases availability and reliability. FFS aims to achieve stackability through the use of FUSE. 
% FIXME: Chip: "While you mention CryptFS and FiST you do not say why you do not build your systems to use a stackable filesystem (as this would make adding redundancy much easier)"
% 	Stackable filesystem - kernel module between VFS and the actual filesystem (eg ext4)
%		According to "To fuse or not to fuse..." they build a stackable filesystem in FUSE
%			Maybe using the kernel API??
% 		Read 2.1 in this article, seems appropriate


\section{Filesystem benchmarking}
IOzone is a filesystem benchmarking tool which is used to measures performance and analyze a filesystem\,\cite{IozoneFilesystemBenchmark}. It is built for, among other platforms, Apple's macOS where the filesystem will be built, run, and tested. 


\section{Summary} % \todo[inline, backgroundcolor=kth-lightblue]{Sammanfattning}
% \todo[inline, backgroundcolor=kth-lightblue]{Det är trevligt om detta kapitel
%   avslutas med en sammanfattning. Till exempel kan du inkludera en tabell som
%   sammanfattar andras idéer och fördelar och nackdelar med varje - så som
%   senare kan du jämföra din lösning till var och en av dessa. Detta kommer
%   också att hjälpa dig att definiera de variabler som du kommer att använda
%   för din utvärdering.}

% \todo[inline]{It is nice to have this chapter conclude with a summary. For
%   example, you can include a table that summarizes other people's ideas and
%   benefits and drawbacks with each - so as later you can compare your solution
%   to each of them. This will also help you define the variables that you will
%   use for your evaluation.}

As presented, different filesystems provide different features and drawbacks. 
In Table~\ref{tbl:fs_comp} we display a summary of characteristics and features 
of some filesystems mentioned above and how FFS compares. As can be seen, FFS 
mainly lacks certain filesystem operations which are not the focus of
FFS as it is a proof of concept. 

\begin{table}[!ht]
	\begin{center}
		\begin{tabular}{ l || c | c | c | c | c | c }
			
			\hline
\hspace{1mm} 						& \textbf{ext4} 	& \textbf{Google drive} & \textbf{DEFY} 	& \textbf{TweetFS} 	& \textbf{FFS}\\
			
			\hline
			\hline
			
Mountable							& 	X 				& 	X					& 	X 				& 	-				& 	X\\
Read/Write/Remove file				& 	X 				& 	X					& 	X 				& 	X				& 	X\\
Read/Write/Remove directory 		& 	X 				& 	X					& 	X 				& 	X				& 	X\\
Hard links 							& 	X 				& 	-					& 	X 				& 	-				& 	-\\
Soft links 							& 	X 				& 	-					& 	X 				& 	-				& 	-\\
File and directory access control 	& 	X 				& 	X					& 	- 				& 	X				& 	-\\

Encrypted							&	X				&	X$^{*}$				&	X				&	-				&	X\\
Steganographic						&	-				&	-					&	X				&	X				&	X\\
Cloud-based							&	-				&	X					&	-				&	X				&	X\\

			\hline		

		\multicolumn{3}{l}{$^{*}$\footnotesize{As mentioned, the user has no control over this encryption}} \\

		\end{tabular}
		\caption{Comparison between features present in related filesystems and FFS. X means that the feature is supported and - means that it is not supported}
		\label{tbl:fs_comp}
	\end{center}

\end{table}


\cleardoublepage


\chapter{Method or Methods}
\label{ch:methods}
% \todo[inline, backgroundcolor=kth-lightblue]{Metod eller Metodval}
% \todo[inline]{This chapter is about Engineering-related
  content, Methodologies and Methods.  Use a self-explaining title.\\The
  contents and structure of this chapter will change with your choice of
  methodology and methods.}



% \todo[inline]{Describe the engineering-related contents (preferably with models) and the research methodology and methods that are used in the degree project.\\
% Give a theoretical description of the scientific or engineering methodology are you going to use and why have you chosen this method. What other methods did you consider and why did you reject them.\\
% In this chapter, you describe what engineering-related and scientific skills you are going to apply, such as modeling, analyzing, developing, and evaluating engineering-related and scientific content. The choice of these methods should be appropriate for the problem . Additionally, you should be consciousness of aspects relating to society and ethics (if applicable). The choices should also reflect your goals and what you (or someone else) should be able to do as a result of your solution - which could not be done well before you started.}

The purpose of this chapter is to provide an overview of the research method
used in this thesis. Section~\ref{sec:researchProcess} describes the research
process. Section~\ref{sec:researchParadigm} details the research
paradigm. Section~\ref{sec:dataCollection} focuses on the data collection
techniques used for this research. Section~\ref{sec:experimentalDesign}
describes the experimental design. Section~\ref{sec:assessingReliability}
explains the techniques used to evaluate the reliability and validity of the
data collected. Section~\ref{sec:plannedDataAnalysis} describes the method
used for the data analysis. Finally, Section~\ref{sec:evaluationFramework}
describes the framework selected to evaluate xxx.

% \todo[inline, backgroundcolor=kth-lightblue]{Vilka vetenskaplig eller ingenjörs-metodik ska du använda och varför har du valt den här metoden. Vilka andra metoder gjorde du övervägde du och varför du avvisar dem.
% Vad är dina mål? (Vad ska du kunna göra som ett resultat av din lösning - vilken inte kan göras i god tid innan du började)
% Vad du ska göra? Hur? Varför? Till exempel, om du har implementerat en artefakt vad gjorde du och varför? Hur kommer du utvärdera den.
% Syftet med detta kapitel är att ge en översikt över forsknings metod som
% används i denna avhandling. Avsnitt~\ref{sec:researchProcess} beskriver forskningsprocessen. Avsnitt~\ref{sec:researchParadigm} beskriver forskningsparadigmen detaljerat. Avsnitt~\ref{sec:dataCollection} fokuserar på datainsamlingstekniker som används för denna forskning. Avsnitt~\ref{sec:experimentalDesign} beskriver experimentell
% design. Avsnitt~\ref{sec:assessingReliability} förklarar de tekniker som används för att utvärdera
% tillförlitligheten och giltigheten av de insamlade uppgifterna. Avsnitt~\ref{sec:plannedDataAnalysis}
% beskriver den metod som används för dataanalysen. Slutligen, Avsnitt~\ref{sec:evaluationFramework}
% beskriver ramverket som valts för att utvärdera xxx.\\
% Ofta kan man koppla ett antal följdfrågor till undersökningsfrågan och problemlösningen t ex\\
% (1) Vilken process skall användas för konstruktion av lösningen och vilken process skall kopplas till denna för att svara på undersökningsfrågan?\\
% (2) Hur och vilket resultat (storheter) skall presenteras både för att redovisa svar på undersökningsfrågan (resultatkapitlet i denna rapport) och redovisa resultat av problemlösningen (prototypen, ofta dokument som bilagor men vilka dokument och varför?).\\
% (3) Vilken teori/teknik skall väljas och användas både för undersökningen (taxonomi, matematik, grafer, storheter mm)  och  problemlösning (UML, UseCases, Java mm) och varför?\\
% (4) Vad behöver du som student leverera för att uppnå hög kvaliet (minimikrav) eller mycket hög kvalitet på examensarbetet?\\
% (5) Frågorna kopplar till de följande underkapitlen.\\
% (6) Resonemanget bygger på att studenter på hing-programmet ofta skall konstruera något åt problemägaren och att man till detta måste koppla en intressant ingenjörsfråga. Det finns hela tiden en dualism mellan dessa aspekter i exjobbet.
% }

\section{Research Process}
\label{sec:researchProcess}
% \todo[inline, backgroundcolor=kth-lightblue]{Undersökningsrocess och utvecklingsprocess}

Figure~\ref{fig:researchprocess} shows the steps conducted in order to carry out this research. 

% \todo[inline, backgroundcolor=kth-lightblue]{Figur~\ref{fig:researchprocess} visar de steg som utförs för att genomföra\\
% Beskriv, gärna med ett aktivitetsdiagram (UML?), din undersökningsprocess och utvecklingsprocess.  Du måste koppla ihop det akademiska intresset (undersökningsprocess) med ursprungsproblemet (utvecklingsprocess)
% denna forskning.\\
% Aktivitetsdiagram från t ex UML-standard}


 
\begin{figure}[!ht]
  \begin{center}
    \includegraphics[width=0.5\textwidth]{figures/researchprocess.png}
  \end{center}
  \caption{Research Process}
  \label{fig:researchprocess}
\end{figure}
% \todo[inline, backgroundcolor=kth-lightblue]{Forskningsprocessen}

\section{Research Paradigm}
\label{sec:researchParadigm}
% \todo[inline, backgroundcolor=kth-lightblue]{Undersökningsparadigm\\
% Exempelvis\\
% Positivistisk (vad/hur fungerar det?) kvalitativ fallstudie med en deduktivt (förbestämd) vald ansats och ett induktivt(efterhand uppstår dataområden och data) insamlade av data och erfarenheter.}


\section{Data Collection}
\label{sec:dataCollection}
% \todo[inline, backgroundcolor=kth-lightblue]{Datainsamling\\
% (Detta bör också visa att du är medveten om de sociala och etiska frågor som
% kan vara relevanta för dina data insamlingsmetod.)}
% \todo[inline]{This should also show that you are aware of the social and ethical concerns that might be relevant to your data collection method.)}



\subsection{Sampling}
% \todo[inline, backgroundcolor=kth-lightblue]{Stickprovsundersökning}

\subsection{Sample Size}
% \todo[inline, backgroundcolor=kth-lightblue]{Provstorleken}

\subsection{Target Population}
% \todo[inline, backgroundcolor=kth-lightblue]{Målgruppen}

\section{Experimental design/Planned Measurements}
\label{sec:experimentalDesign}
% \todo[inline, backgroundcolor=kth-lightblue]{Experimentdesign/Mätuppställning}
% \todo[inline, backgroundcolor=kth-lightblue]{Testmiljö/testbädd/modell\\
% Beskriv allt att någon annan skulle behöva återskapa din testmiljö / testbädd / modell / …}
\subsection{Test environment/test bed/model} % \todo[inline]{Describe everything that someone else would need to reproduce your test environment/test bed/model/… .}


\subsection{Hardware/Software to be used}
% \todo[inline, backgroundcolor=kth-lightblue]{Hårdvara / programvara som ska användas}


\section{Assessing reliability and validity of the data collected}
\label{sec:assessingReliability}
% \todo[inline, backgroundcolor=kth-lightblue]{Bedömning av validitet och reliabilitet hos använda metoder och insamlade data }


\subsection{Validity of method}
\label{sec:validtyOfMethod}
% \todo[inline, backgroundcolor=kth-lightblue]{Giltigheten av metoder\\
%   Har dina metoder gett dig de rätta svaren och lösningarna? Var metoderna korrekta?}

% \todo[inline]{How will you know if your results are valid?}

\subsection{Reliability of method}
\label{sec:reliabilityOfMethod}
% \todo[inline, backgroundcolor=kth-lightblue]{Tillförlitlighet av för metoder\\
% Hur bra är dina metoder, finns det bättre metoder? Hur kan du förbättra dem?}
% \todo[inline]{How will you know if your results are reliable?}

\subsection{Data validity}
\label{sec:dataValidity}
% \todo[inline, backgroundcolor=kth-lightblue]{Giltigheten av uppgifter\\
% Hur vet du om dina resultat är giltiga? Är ditt resultat rättvisande?}

\subsection{Reliability of data}
\label{sec:reliabilityOfData}
% \todo[inline, backgroundcolor=kth-lightblue]{Tillförlitlighet av data\\
% Hur vet du om dina resultat är tillförlitliga? Hur bra är dina resultat?}


\section{Planned Data Analysis}
\label{sec:plannedDataAnalysis}
% \todo[inline, backgroundcolor=kth-lightblue]{Metod för analys av data}


\subsection{Data Analysis Technique}
\label{sec:dataAnalysisTechnique}
% \todo[inline, backgroundcolor=kth-lightblue]{Dataanalysteknik}

\subsection{Software Tools}
\label{sec:softwareTools}
% \todo[inline, backgroundcolor=kth-lightblue]{Mjukvaruverktyg}


\section{Evaluation framework}
\label{sec:evaluationFramework}
% \todo[inline, backgroundcolor=kth-lightblue]{Utvärdering och ramverk\\
% Metod för utvärdering, jämförelse mm. Kopplar till kapitel~\ref{ch:resultsAndAnalysis}.}

\section{System documentation}
\label{sec:systemDocumentation}
% \todo[inline, backgroundcolor=kth-lightblue]{Systemdokumentation\\
% Med vilka dokument och hur skall en konstruerad prototyp dokumenteras? Detta blir ofta bilagor till rapporten och det som problemägaren till det ursprungliga problemet (industrin) ofta vill ha.\\
% Bland dessa bilagor återfinns ofta, och enligt någon angiven standard, kravdokument, arkitekturdokument, designdokumnet, implementationsdokument, driftsdokument, testprotokoll mm.}
% \todo[inline]{If this is going to be a complete document consider putting it in as an appendix, then just put the highlights here.}



% \cleardoublepage

% \chapter{What you did}
\label{ch:whatYouDid}


\cleardoublepage


\chapter{Results and Analysis}
\label{ch:resultsAndAnalysis}

% TODO: Introduce chapter 

% Say that benchmarking "_suggests_ that FFS is slower than blah blah..."

\cleardoublepage


\chapter{Discussion}
\label{ch:discussion}
% TODO: Introduce chapter

% \todo[inline, backgroundcolor=kth-lightblue]{Diskussion\\
% Förbättringsförslag?}
% \todo[inline]{This can be a separate chapter or a section
%   in the previous chapter.}


\cleardoublepage

\chapter{Conclusions and Future work}
\label{ch:conclusionsAndFutureWork}
\Cref{sec:conclusions} presents the conclusions from the discussion \Cref{ch:discussion} while \Cref{sec:futureWork} suggests future work.

% \todo[inline, backgroundcolor=kth-lightblue]{Slutsats och framtida arbete}

% \todo[inline]{Add text to introduce the subsections of this chapter.}


\section{Conclusions}
\label{sec:conclusions}
% \todo[inline, backgroundcolor=kth-lightblue]{Slutsatser}
% \todo[inline]{Describe the conclusions (reflect on the whole introduction given in Chapter 1).}
  
% \todo[inline]{Discuss the positive effects and the drawbacks.\\
% Describe the evaluation of the results of the degree project.\\
% Did you meet your goals?\\
% What insights have you gained?\\
% What suggestions can you give to others working in this area?\\
% If you had it to do again, what would you have done differently?}

% \todo[inline, backgroundcolor=kth-lightblue]{Uppfyllde du dina mål?\\
% Vilka insikter har du fått?\\
% Vilka förslag kan du ge till andra som arbetar inom detta område?
% Om du skulle göra detta igen, vad skulle du ha gjort annorlunda?}

\gls{FFS} is a cryptographic and deniable \mbox{cloud-based} filesystem with free storage through exploiting online web services. Compared to other filesystems, \gls{FFS} is slow and is not suitable as a \mbox{multi-purpose} filesystem, for instance as a hard drive for a computer. It performed poorly even compared to another \mbox{cloud-based} filesystem, \gls{GCSF}. However, one key difference between these two filesystems is that \gls{FFS} manages the cryptography of the filesystem, while \gls{GCSF} delegates this task to Google Drive. This provides security benefits for \gls{FFS}, but might also contribute to the slower computation time. The results also show that even when removing the dependency of an internet connection is \gls{FFS} performing poorly, especially for the read operations compared to \gls{GCSF} and \gls{APFS}. The write operations of \gls{FFFS} perform better than \gls{GCSF} and \gls{FFS}. The read operations of \gls{FFS} and \gls{FFFS} are more similar than the write operations, however, \gls{FFFS} outperforms \gls{FFS} for every read operation test leading to the conclusion that the internet connection and the \gls{OWS} influence the file operations significantly. With better read performance than write performance, \gls{FFS} is best suited as a \mbox{many-read-few-write} filesystem.

While the filesystem is slow, it provides security aspects such as \mbox{end-to-end} encryption and deniability. As long as the filesystem is not mounted to the computer, it is not possible to prove how much data is stored on \gls{FFS}, or even prove that data is stored on \gls{FFS}. However, it is possible to get a upper-limit amount of data stored. \mbox{End-to-end} cryptography provides the user with confidential data. Further, by using authenticated encryption, \gls{FFS} provides the user with proof of the authenticity of the data it stores. 

%TODO: Possible to get an estimate of how much data is stored?

% 
\section{Limitations}
\label{sec:limitations}
% \todo[inline, backgroundcolor=kth-lightblue]{Begränsande faktorer\\
% Vad gjorde du som begränsade dina ansträngningar? Vilka är begränsningarna i dina resultat?}
% \todo[inline]{What did you find that limited your
%   efforts? What are the limitations of your results?}




\section{Future work}
\label{sec:futureWork}
% \todo[inline, backgroundcolor=kth-lightblue]{Vad du har kvar ogjort?\\
% Vad är nästa självklara saker som ska göras?\\
% Vad tips kan du ge till nästa person som kommer att följa upp på ditt arbete?
% \todo[inline]{Describe valid future work that you or someone else could or should do.\\
% Consider: What you have left undone? What are the next obvious things to be done? What hints can you give to the next person who is going to follow up on your work?
% }

% }

% Due to the breadth of the problem, only some of the initial goals have been
% met. In these section we will focus on some of the remaining issues that
% should be addressed in future work. ...

% \subsection{What has been left undone?}
% \label{what-has-been-left-undone}

% The prototype does not address the third requirment, i.e., a yearly
% unavailability of less than 3 minutes, this remains an open problem. ...

% \subsubsection{Cost analysis}

% The current prototype works, but the performance from a cost perspective makes
% this an impractical solution. Future work must reduce the cost of this
% solution, to do so a cost analysis needs to first be done. ...

% \subsubsection{Security}

% A future research effort is needed to address the security holes that results
% from using a self-signed certificate. Page filling text mass. Page filling
% text mass. ...


% \subsection{Next obvious things to be done}

% In particular, the author of this thesis wishes to point out xxxxxx remains as
% a problem to be solved. Solving this problem is the next thing that should be
% done. ...

% 
\section{Reflections}
\label{sec:reflections}
% \todo[inline, backgroundcolor=kth-lightblue]{Reflektioner}
% \todo[inline, backgroundcolor=kth-lightblue]{Vilka är de relevanta ekonomiska, sociala, miljömässiga och etiska aspekter av ditt arbete?}
% \todo[inline]{What are the relevant economic, social,
%   environmental, and ethical aspects of your work?
% }



One of the most important results is the reduction in the amount of
energy required to process each packet while at the same time reducing the
time required to process each packet.

The thesis contributes to the \gls{UN}\enspace\glspl{SDG} numbers 1 and 9 by
xxxx. 

\noindent\rule{\textwidth}{0.4mm}

\cleardoublepage
% Print the bibliography (and make it appear in the table of contents)
\renewcommand{\bibname}{References}
\addcontentsline{toc}{chapter}{References}

\ifbiblatex
    %\typeout{Biblatex current language is \currentlang}
    \printbibliography%[heading=bibintoc]
\else
    \bibliography{bibliography}
\fi

\cleardoublepage

\input{appendices/dir_inode_code_appendix}

%% The following label is necessary for computing the last page number of the body of the report to include in the "For DIVA" information
\label{pg:lastPageofMainmatter}

\clearpage
\kthbackcover
\fancyhead{}  % Do not use header on this extra page or pages
\section*{For DIVA}
\lstset{numbers=none} %% remove any list line numbering
\divainfo{pg:lastPageofPreface}{pg:lastPageofMainmatter}
\end{document}
